\documentclass[[12pt,twoside]{book}
\usepackage{_my_document_style}

\begin{document}

%
\def\myAltitudeM{9000.000000}
\def\myAltitudeFt{29527.559055}
\def\myMassKg{79000.000000}
\def\myMassLb{174165.187126}
\def\myWingSurfaceMTSD{124.600000}
\def\myWingSurfaceFTSD{4400.207473}
\def\myMach{0.592691}
\def\mySpeedKts{350.000000}
\def\mySpeedMS{180.055556}
\def\mySpeedKmH{648.200000}
\def\myGravitationalAcceleration{9.810000}
\def\myDensityKGMC{0.466348}
\def\myDynamicPressurePa{7559.504874}
\def\myDynamicPressureBar{0.075595}
\def\myLiftCoefficient{0.822782}
\def\myAspectRatio{9.460000}
\def\myOswaldFactor{0.850000}
\def\myKpolar{0.146309}
\def\myDragCoefficientZero{0.019000}
\def\myDragCoefficient{0.118047}
\def\myDragN{111189.964083}
\def\myDragKgf{11338.220910}
\def\myDragLbf{24996.498309}

%
\begin{myExampleX}{Aircraft aerodynamic coefficients } {\ding{46}}% \ \Keyboard\ % 
\label{example:Equivalent:Airspeed:Basic:B}
%

\noindent
In this numerical example, we carry out the calculation of an aerodynamic coefficient starting
from a force and subsequently the calculation of an aerodynamic force known the coefficient.
An aircraft with characteristics similar to those of a Boeing ~ 777-300 ER has mass
$m=\SI[round-precision=0]{\myMassKg}{kg}=\SI[round-precision=0]{\myMassLb}{lb}$
(that is a weight equal to about 80 \% of the, \emph{Maximum Take-Off Weight}, MTOW)
and a wing surface
$S=\SI[round-precision=1]{\myWingSurfaceMTSD}{\meter^2}=\SI[round-precision=1]{\myWingSurfaceFTSD}{\feet^2}$.
For a flight speed of
$V=\SI[round-precision=2]{\mySpeedKts}{kts}=\SI[round-precision=1]{\mySpeedKmH}{\kilo\meter/\hour}$,
at the altitude of
$h=\SI[round-precision=0]{\myAltitudeM}{\meter}$,
for which
$\rho=\SI[round-precision=3]{\myDensityKGMC}{\kilogram\,\meter^{-3}}$,
%(cfr.~esempio~\ref{example:Atmosphere:Basic:A}),
we  have a dynamic flight pressure
\[
\begin{split}
\bar{q} = \frac{1}{2} \rho V^2 
  & {}=
  \num{0.5} \cdot \SI[round-precision=3]{\myDensityKGMC}{\kilogram\,\meter^{-3}}
    \, (\SI[round-precision=2]{\mySpeedMS}{\meter/\second}\big)^2
\\
  & {}=
    { 
      \SI[round-precision=3,scientific-notation=true]{\myDynamicPressurePa}{\newton\,\meter^{-2}} 
    }
    = 
    { 
      \SI[round-precision=3,scientific-notation=true]{\myDynamicPressureBar}{ bar } 
    }
\end{split}
\]
corresponding to a Mach number $M=\SI[round-precision=2]{\myMach}{}$.
If we take a balanced flight, in which lift $L$ equals weight
 $W=mg$, we have a lift coefficient
$C_L$ that is
\[
C_L=\frac{L}{\bar{q}S}=\frac{m g}{\bar{q}S}
  = \frac{
      \SI[round-precision=0]{\myMassKg}{\kilogram} \cdot \SI{9.81}{\meter\,\second^{-2}}
    }{
      \SI[round-precision=3,scientific-notation=true]{\myDynamicPressurePa}{\newton\,\meter^{-2}}
      \cdot \SI[round-precision=1]{\myWingSurfaceMTSD}{\meter^2} 
    }
  = { \SI[round-precision=3]{\myLiftCoefficient}{} }
\]
%
For a similar aircraft, under the given conditions, a polar given by the following law can be assumed:
%############################

\[
C_D = C_{D0} + \frac{C_L^2}{\pi \AR_\Wing e} = C_{D0} + K\,C_L^2
  = \SI[round-precision=4]{\myDragCoefficientZero}{}
    + \SI[round-precision=4]{\myKpolar}\, C_L^2
\]
 where $\AR_\Wing=\SI[round-precision=2]{\myAspectRatio}$,
$e=\SI[round-precision=2]{\myOswaldFactor}{}$,
K=1/(\AR_\Wing \, e  \pi)=\SI[round-precision=4]{\myKpolar}.

%
\noindent
Therefore, the flight drag coefficient
is
\[
C_D
  = \SI[round-precision=4]{\myDragCoefficientZero}{}
    + 
      \SI[round-precision=4]{\myKpolar} 
        \cdot \big( \SI[round-precision=3]{\myLiftCoefficient}{} \big)^2
  =\SI[round-precision=4]{\myDragCoefficient}{}
 \]
%
If it is assumed that resistance is equal to thrust,
the propulsive force necessary for balanced flight is
\[
\begin{split}
T & {}= D = C_D \, \bar{q} S
\\
  & {}= \SI[round-precision=4]{\myDragCoefficient}{}
    \cdot \SI[round-precision=3,scientific-notation=true]{\myDynamicPressureBar}{\newton\,\meter^{-2}}
    \cdot \SI[round-precision=1]{\myWingSurfaceMTSD}{\meter^2}
  = { \SI[round-precision=3,scientific-notation=true]{\myDragN}{\newton} }
\\
  & {}= { \SI[round-precision=3,scientific-notation=true]{\myDragKgf}{\kilopond} }
  = { \SI[round-precision=3,scientific-notation=true]{\myDragLbf}{\lbf} }
\end{split}
\]
\end{myExampleX}

\end{document}