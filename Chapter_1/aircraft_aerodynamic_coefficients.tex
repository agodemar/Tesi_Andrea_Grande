\documentclass[[12pt,twoside]{book}
\usepackage{_my_document_style}

\begin{document}

%
\def\myInitialValueA{999}
\def\myISAAirGasConstNMTKGK{287}
\def\myISAAirAdiabaticIndex{1.4}
\def\myISAAirTemperatureSeaLevelK{288.16}
\def\myISAAirDensitySeaLevelKGMTcubed{1.225}
\def\myISAAirDensitySeaLevelSLUGFTcubed{0.0023769}
\def\myAltitudeMT{3000}
\def\myAltitudeFT{9842.52}
\def\myISALapseRateKMT{-0.0065}
\def\myAirDensityKGMTcubed{0.9091}
\def\myAirDensitySLUGFTcubed{0.0017639}
\def\myAirDensityRatio{0.7421}
\def\myAirDynamicViscosiyMTSecKG{1.694e-05}
\def\myAirTemperatureK{268.66}
\def\myAirSoundSpeedMTSec{328.55378}
\def\myAirSoundSpeedFTSec{1077.93236}
\def\myPrepareForStackNull{-1e+307}
\def\myInitialValueB{999}
\def\myMach{0.147}
\def\myFlightSpeedMTSec{48.368}
\def\myFlightSpeedFTSec{158.68767}
\def\myFlightSpeedKMH{174.1}
\def\myFlightSpeedKTS{94.02}
\def\myFlightSpeedEASMTSec{41.66667}
\def\myFlightSpeedEASFTSec{136.70166}
\def\myFlightSpeedEASKMH{150}
\def\myFlightSpeedEASKTS{80.99}
\def\myFlightQbarPA{1063.36806}
\def\myFlightQbarBAR{0.011}
\def\myInitialValueB{999}
\def\myUBodyMTSec{48.368}
\def\myUBodyFTSec{158.68767}
\def\myUBodyKMH{174.1}
\def\myUBodyKTS{94.02}
\def\myVBodyMTSec{0}
\def\myVBodyFTSec{0}
\def\myVBodyKMH{0}
\def\myVBodyKTS{0}
\def\myWBodyMTSec{0}
\def\myWBodyFTSec{0}
\def\myWBodyKMH{0}
\def\myWBodyKTS{0}

%
\begin{myExampleX}{Aircraft aerodynamic  coefficients } {\ding{46}}% \ \Keyboard\ % 
\label{example:Equivalent:Airspeed:Basic:B}
%

\noindent
In this numerical example, we carry out the calculation of an aerodynamic coefficient starting
from a force and subsequently the calculation of an aerodynamic force known the coefficient.

An aircraft with characteristics similar to those of a Boeing ~ 777-300ER has mass
$m=\SI[round-precision=0]{\myMassKG}{\kilogram}=\SI[round-precision=0]{\myMassLB}{\lbm}$
(that is a weight equal to about 80 \% of the maximum take-off weight, \emph{Maximum Take-Off Weight}, MTOW)
and a wing surface
$S=\SI[round-precision=1]{\myAreaWingMTsquared}{\meter^2}=\SI[round-precision=1]{\myAreaWingFTsquared}{\feet^2}$.
For a flight at speed
$V=\SI[round-precision=2]{\myFlightSpeedKTS}{kts}=\SI[round-precision=1]{\myFlightSpeedKMH}{\kilo\meter/\hour}$,
at the altitude of
$h=\SI[round-precision=3]{\myAltitudeMT}{\meter}$,
for which
$\rho=\SI[round-precision=3]{\myAirDensityKGMTcubed}{\kilogram\,\meter^{-3}}$,
%(cfr.~esempio~\ref{example:Atmosphere:Basic:A}),
we  have a dynamic flight pressure
\[
\begin{split}
\bar{q} = \frac{1}{2} \rho V^2 
  & {}=
  \num{0.5} \cdot \SI[round-precision=3]{\myAirDensityKGMTcubed}{\kilogram\,\meter^{-3}}
    \, (\SI[round-precision=2]{\myFlightSpeedMTSec}{\meter/\second}\big)^2
\\
  & {}=
    \mathunderline{mydarkblue}{ 
      \SI[round-precision=3,scientific-notation=true]{\myDynamicPressurePA}{\newton\,\meter^{-2}} 
    }
    = \mathunderline{mydarkblue}{ 
      \SI[round-precision=3,scientific-notation=true]{\myDynamicPressureBAR}{ bar } 
    }
\end{split}
\]
corresponding to a Mach number $M=\SI[round-precision=2]{\myMachNumber}{}$.
If we take a balanced flight, in which lift $L$ equals weight
 $W=mg$, we have a lift coefficient
$C_L$ equal to
\[
C_L=\frac{L}{\bar{q}S}=\frac{m g}{\bar{q}S}
  = \frac{
      \SI[round-precision=0]{\myMassKG}{\kilogram} \cdot \SI{9.81}{\meter\,\second^{-2}}
    }{
      \SI[round-precision=3,scientific-notation=true]{\myDynamicPressurePA}{\newton\,\meter^{-2}}
      \cdot \SI[round-precision=1]{\myAreaWingMTsquared}{\meter^2} 
    }
  = \mathunderline{mydarkblue}{ \SI[round-precision=3]{\myLiftCoefficient}{} }
\]

For a similar aircraft, under the given conditions, a polar given by the following law can be assumed:
%############################
\ExplSyntaxOn
\def\myKPolar{\calcnum[round-precision=4]{\myAspectRatioWing * \myInducedDragFactorWing}}
\ExplSyntaxOff
%############################
\[
C_D = C_{D0} + \frac{C_L^2}{\pi \AR_\Wing e} = C_{D0} + K\,C_L^2
  = \SI[round-precision=4]{\myDragCoefficientZero}{}
    + \calcSI[round-precision=4]{\myAspectRatioWing * \myInducedDragFactorWing}{} \, C_L^2
\]
where $\AR_\Wing=\SI[round-precision=2]{\myAspectRatioWing}{}$,
$e=\SI[round-precision=2]{\myInducedDragFactorWing}{}$,
% $K=1/(\AR_\Wing \, e , pi ) = \SI[round-precision=4]{\myKPolar}{}$.
$K=1/(\AR_\Wing \, e)=\myKPolar$.

Therefore, the flight drag coefficient
is
\[
C_D
  = \SI[round-precision=4]{\myDragCoefficientZero}{}
    + 
      % \SI[round-precision=4]{\myKPolar}{} 
      \myKPolar
      \cdot \big( \SI[round-precision=3]{\myLiftCoefficient}{} \big)^2
  =\SI[round-precision=4]{\myDragCoefficient}{}
\]


If it is assumed that resistance is equal to thrust,
the propulsive force necessary for balanced flight is
\[
\begin{split}
T & {}= D = C_D \, \bar{q} S
\\
  & {}= \SI[round-precision=4]{\myDragCoefficient}{}
    \cdot \SI[round-precision=3,scientific-notation=true]{\myDynamicPressureBAR}{\newton\,\meter^{-2}}
    \cdot \SI[round-precision=1]{\myAreaWingMTsquared}{\meter^2}
  = \mathunderline{mydarkblue}{ \SI[round-precision=3,scientific-notation=true]{\myDragN}{\newton} }
\\
  & {}= \mathunderline{mydarkblue}{ \SI[round-precision=3,scientific-notation=true]{\myDragKGF}{\kilopond} }
  = \mathunderline{mydarkblue}{ \SI[round-precision=3,scientific-notation=true]{\myDragLBF}{\lbf} }
\end{split}
\]
\end{myExampleX}

\end{document}