\documentclass[[12pt,twoside]{book}
\usepackage{_my_document_style}

\begin{document}

%
\def\myInitialValueA{999}
\def\myISAAirGasConstNMTKGK{287}
\def\myISAAirAdiabaticIndex{1.4}
\def\myISAAirTemperatureSeaLevelK{288.16}
\def\myISAAirDensitySeaLevelKGMTcubed{1.225}
\def\myISAAirDensitySeaLevelSLUGFTcubed{0.0023769}
\def\myAltitudeMT{3000}
\def\myAltitudeFT{9842.52}
\def\myISALapseRateKMT{-0.0065}
\def\myAirDensityKGMTcubed{0.9091}
\def\myAirDensitySLUGFTcubed{0.0017639}
\def\myAirDensityRatio{0.7421}
\def\myAirDynamicViscosiyMTSecKG{1.694e-05}
\def\myAirTemperatureK{268.66}
\def\myAirSoundSpeedMTSec{328.55378}
\def\myAirSoundSpeedFTSec{1077.93236}
\def\myPrepareForStackNull{-1e+307}
\def\myInitialValueB{999}
\def\myMach{0.147}
\def\myFlightSpeedMTSec{48.368}
\def\myFlightSpeedFTSec{158.68767}
\def\myFlightSpeedKMH{174.1}
\def\myFlightSpeedKTS{94.02}
\def\myFlightSpeedEASMTSec{41.66667}
\def\myFlightSpeedEASFTSec{136.70166}
\def\myFlightSpeedEASKMH{150}
\def\myFlightSpeedEASKTS{80.99}
\def\myFlightQbarPA{1063.36806}
\def\myFlightQbarBAR{0.011}
\def\myInitialValueB{999}
\def\myUBodyMTSec{48.368}
\def\myUBodyFTSec{158.68767}
\def\myUBodyKMH{174.1}
\def\myUBodyKTS{94.02}
\def\myVBodyMTSec{0}
\def\myVBodyFTSec{0}
\def\myVBodyKMH{0}
\def\myVBodyKTS{0}
\def\myWBodyMTSec{0}
\def\myWBodyFTSec{0}
\def\myWBodyKMH{0}
\def\myWBodyKTS{0}

%
\begin{myExampleX}{characteristics of the air at a certain altitude}{\ding{46}}% \ \Keyboard\ %
\label{example:Equivalent:Airspeed:Basic:B}
%
\noindent
Let us calculate air characteristics,based on the ISA model, at a flying altitude of $h=\SI[round-precision=0]{\myAltitudeM}{m}$.
% 
Moreover,it is demanded to determine the true airspeed at a Mach $M=\SI[round-precision=2]{\myMachnumber}{m}$ and the Reynolds number per unit of lenght $\frac{ Re }{l_\text{rif} }$ ; the given altitude is inside the stratosphere,therefore,as the ISA model requires,we have a temperature gradient $LR=\SI[round-precision=4]{\myTemperaturegradient}{K/m}$.
Consequently,the temperature at the flying altitude is
\[
 T = {T_\text{SL}- \text{LR} h   =\SI[round-precision=2]{\myTemperaturesealevelK}{K}+(\SI[round-precision=4]{\myTemperaturegradient}{K/m}) \cdot\SI[round-precision=0]{\myAltitudeM}{m} } = \SI[round-precision=1]{\myTemperatureK}{K}
 \]
and the related speed sound  is $a={\sqrt{\gamma_\text{SL}{R_\text{SL}}{T}}}=\sqrt{\SI[round-precision=1]{\myAdiabaticindex}\cdot\ \SI[round-precision=0]{\myAirspecificconstant} {N m Kg^{-1} K^{-1} } \cdot\ \SI[round-precision=2]{\myTemperatureK} {K}} = \SI[round-precision=2]{\mySoundspeed} {m/s}$ 

\noindent
%
For the given flying Mach number and altitude ,we have a flight velocity
\[
 V = {a }~{M} =\SI[round-precision=2]{\mySoundspeed}{m/s} \cdot\ \SI[round-precision=2]{\myMachnumber}= ~\SI[round-precision=2]{\mySpeedMs}{m/s} = \SI[round-precision=2]{\mySpeedkmh}{km/h} 
 \]
the relative density value is \[
 \sigma = \frac{ \rho }{\rho_\text{SL} } = \biggl( \frac{T }{T_\text{SL} }  \biggr) ^ {-\Bigl( \frac{ g }{RL\cdot\ R_\text{air}} +1 \Bigr)}  
 \]
 \[
 = \biggl(\frac{ \SI[round-precision=2]{\myTemperatureK}{K} }{\SI[round-precision=2]{\myTemperaturesealevelK}{k} }\biggr) ^ {-\Bigl( \frac{\SI[round-precision=2]{\myGravitationalacceleration}{m/s^2} }{\SI[round-precision=4]{\myTemperaturegradient}{K/m}~\cdot\ \SI[round-precision=0]{\myAirspecificconstant}{N m Kg^{-1} K^{-1}}} +1 \Bigr)} =\SI[round-precision=3]{\myRelativedensity}
 \]  

Therefore the air density at the flight altitude is
\[
 \rho = {\rho_\text{SL} }~{\sigma} =\SI[round-precision=3]{\myDensitysealevel}{kg~m^{-3}} \cdot\ \SI[round-precision=3]{\myRelativedensity} =~ \SI[round-precision=3]{\myDensity}{kg~m^{-3}}
\]

This value and the flight speed value allow the calculation of the dynamic pressure of flight
\[
\bar{q}=\frac{1}{2} \rho V^{2} =\frac{1}{2}\cdot\ \SI[round-precision=3]{\myDensity}{kg~m^{-3}} \cdot\ (\SI[round-precision=2]{\mySpeedMs}{m/s})^{2} =~\SI[round-precision=2]{\myDynamicpressureNm}{N m^{-2}}=~\SI[round-precision=3]{\myDynamicpressureBar}{bar}
\]
Lastly,the dynamic viscosity of the air at the given  altitude is :
\[
\mu =  \SI[round-precision=3]{\myCone}{\cdot\ 10^{-6}}{Kg} \biggl({m s \sqrt{K} }\biggr)^{-1} \cdot\ \frac {T^{\frac{3}{2}}}{T+\SI[round-precision=1]{\myCtwo}{K}} =
 \SI[round-precision=3]{\myCone}{\cdot\ 10^{-6}}{Kg} \biggl({m s \sqrt{K} }\biggr)^{-1} \cdot\ \frac {(\SI[round-precision=3]{\myTemperatureK}{K})^{\frac{3}{2}}}{\SI[round-precision=3]{\myTemperatureK}{K}+\SI[round-precision=1]{\myCtwo}{K}} =\SI[round-precision=6]{\myDynamicviscosity }{Kg}{\biggl({m s \sqrt{K} }\biggr)^{-1}}
\]
Which corresponds to a Reynolds number per unit of lenght
\[
 \frac{ Re }{l_\text{rif} } =\frac{ \rho V}{\mu }=\frac{ \SI[round-precision=3]{\myDensity}{kg~m^{-3}}~ \SI[round-precision=2]{\mySpeedMs}{m/s}}{\SI[round-precision=6]{\myDynamicviscosity }{Kg} }  = \SI[round-precision=0]{\myReynoldsnumeberperunitoflenght}{m^{-1}}


\]

\end{myExampleX}

\end{document}