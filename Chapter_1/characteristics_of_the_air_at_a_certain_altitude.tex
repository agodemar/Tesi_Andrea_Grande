\documentclass[[12pt,twoside]{book}
\usepackage{_my_document_style}
\begin{document}
%
\def\myAdiabaticindex{1.400000}
\def\myAltitudeM{8000.000000}
\def\myTemperaturesealevelK{288.160000}
\def\myTemperaturegradient{-0.006500}
\def\myAirspecificconstant{287.000000}
\def\myMachnumber{0.700000}
\def\myTemperatureK{236.150000}
\def\mySoundspeed{308.062697}
\def\mySpeedMs{215.643888}
\def\mySpeedkmh{776.317997}
\def\myGravitationalacceleration{9.810000}
\def\myDensitysealevel{1.225000}
\def\myRelativedensity{0.428708}
\def\myDynamicpressureNm{12210.746165}
\def\myDynamicpressureBar{0.122107}
\def\myDensity{0.525168}
\def\myCone{1.458000}
\def\myCtwo{110.400000}
\def\myReynoldsnumeberperunitoflenght{7.417567}

%
\begin{myExampleX}{Characteristics of the air at a given altitude}{\ding{46}}% \ \Keyboard\ %
\label{example:Characteristics:Of:The:Air:At:A:Certain:Altitude}
%
\noindent
Let us calculate air characteristics, based on the ISA model, at a flying altitude of $h=\SI[round-precision=0]{\myAltitudeM}{m}$.
% 
Moreover, it is required to determine the true airspeed at a Mach $M=\SI[round-precision=2]{\myMachnumber}{m}$ and the Reynolds number per unit of length $\frac{\text{\itshape Re} }{l_\text{ref}}$. The given altitude is inside the stratosphere, therefore, as the ISA model requires, we have a temperature gradient (known as Lapse Rate) $\text{\itshape LR}=\SI[round-precision=4]{\myTemperaturegradient}{K/m}$.
Consequently, the temperature at the flying altitude is:
\[
T = {T_\text{SL}- \text{\itshape LR} \, h   
    = \SI[round-precision=2]{\myTemperaturesealevelK}{K}
        + \left( \SI[round-precision=4]{\myTemperaturegradient}{\frac{K}{m}} \right) \cdot\SI[round-precision=0]{\myAltitudeM}{m} } 
    = \SI[round-precision=1]{\myTemperatureK}{K}
\]
and the related speed sound is:
\[
a =\sqrt{
    \gamma_\text{SL} \, R_\text{SL} \,T}
    = \sqrt{
        \SI[round-precision=1]{\myAdiabaticindex} \cdot
        \SI[round-precision=0]{\myAirspecificconstant}{N\,m\,kg^{-1}\,K^{-1}} \cdot
        \SI[round-precision=2]{\myTemperatureK}{K}} 
    = \SI[round-precision=2]{\mySoundspeed}{m/s}
\]
\noindent
%
For the given flying Mach number and altitude, there is the following flight velocity:
\[
 V = {a }~{M} =\SI[round-precision=2]{\mySoundspeed}{m/s} \cdot \SI[round-precision=2]{\myMachnumber}= ~\SI[round-precision=2]{\mySpeedMs}{m/s} = \SI[round-precision=2]{\mySpeedkmh}{km/h} 
 \]
the relative density value is:
\[
\begin{split}
\sigma 
    & ={} \frac{\rho}{\rho_\text{SL}} = 
    \left( \frac{T}{T_\text{SL} } \right)^{
        -\left(\displaystyle \dfrac{ g }{\textit{LR}\, R_\text{air}} + 1 \right)
        } 
    \\ 
    & ={} \left(\frac{
        \SI[round-precision=2]{\myTemperatureK}{K} 
        }{
        \SI[round-precision=2]{\myTemperaturesealevelK}{K}
        }\right)^{\displaystyle
            -\left( \dfrac{
                \SI[round-precision=2]{\myGravitationalacceleration}{m/s^2}
                }{
                \SI[round-precision=4]{\myTemperaturegradient}{K/m}
                \cdot
                \SI[round-precision=0]{\myAirspecificconstant}{N\,m\,kg^{-1}\,K^{-1}}
                }{}
            + 1 \right)
        }
        = \SI[round-precision=3]{\myRelativedensity}{}
\end{split}
\]
So the air density at the flight altitude is:
\[
\rho = \rho_\text{SL} \, \sigma 
     = \SI[round-precision=3]{\myDensitysealevel}{\si{\frac{kg}{m^{3}}}} \, \SI[round-precision=3]{\myRelativedensity} 
     = \SI[round-precision=3]{\myDensity}{\si{\frac{kg}{m^{3}}}}
\]
%
The above mentioned value and the flight speed value allow the calculation of the dynamic pressure of flight
\[
\bar{q}=\frac{1}{2} \rho V^{2} =\frac{1}{2}\cdot\ \SI[round-precision=3]{\myDensity}{kg~m^{-3}} \cdot\ (\SI[round-precision=2]{\mySpeedMs}{m/s})^{2} =~\SI[round-precision=2]{\myDynamicpressureNm}{N m^{-2}}=~\SI[round-precision=3]{\myDynamicpressureBar}{bar}
\]
Lastly, the dynamic viscosity of the air at the given  altitude is:
% https://www.cfd-online.com/Wiki/Sutherland%27s_law
\[
\begin{split}
\mu & {}= \mu_0 \, \left(\frac{T}{T_0}\right)^{\frac{3}{2}} \,                 \frac {T_0 + \SI{110.4}{K}}{T + \SI{110.4}{K}}
    \\
     & {}=\SI[round-precision=1,scientific-notation=true]{\myCone}  {\si {\frac{kg}{m\,s}}}\, \left(\frac{\SI[round-precision=2]{\myTemperatureK}{K}}  {\SI[round-precision=2]{\myTemperaturesealevelK}{K}}
     \right)^{\frac{3}{2}} \,  \frac{\SI[round-precision=2]{\myTemperaturesealevelK}{K} + \SI{110.4}{K}}{\SI[round-precision=2]{\myTemperatureK}{K} + \SI{110.4}{K}}
     \\
     & {}=\SI[round-precision=1,scientific-notation=true]{
     \myDynamicviscosity } {\si {\frac{kg}{m\,s}}}
\end{split}
\]
The above value of $\mu$ corresponds to a Reynolds number per unit of length:
\[
 \frac{\text{\itshape{Re}}}{l_\text{ref}} 
   = \frac{\rho V}{\mu}
   = \frac{
     \SI[round-precision=3]{\myDensity}{kg\,m^{-3}}\,
       \SI[round-precision=2]{\mySpeedMs}{m/s}
       }{
         \SI[round-precision=2,scientific-notation=true]{\myDynamicviscosity}{kg/(m\,s)}
       }
 = \SI[round-precision=0]{\myReynoldsnumeberperunitoflenght}{m^{-1}}
 \approx \SI[scientific-notation=engineering,round-precision=2]{\myReynoldsnumeberperunitoflenght}{m^{-1}}
\]
\end{myExampleX}
\end{document}