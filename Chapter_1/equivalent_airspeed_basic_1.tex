\documentclass[[12pt,twoside]{book}
\usepackage{_my_document_style}

\begin{document}

%===============================================================================================
%--BEGIN-EXERCISE:equivalent_airspeed_basic_1
%
\def\myInitialValueA{999}
\def\myISAAirGasConstNMTKGK{287}
\def\myISAAirAdiabaticIndex{1.4}
\def\myISAAirTemperatureSeaLevelK{288.16}
\def\myISAAirDensitySeaLevelKGMTcubed{1.225}
\def\myISAAirDensitySeaLevelSLUGFTcubed{0.0023769}
\def\myAltitudeMT{3000}
\def\myAltitudeFT{9842.52}
\def\myISALapseRateKMT{-0.0065}
\def\myAirDensityKGMTcubed{0.9091}
\def\myAirDensitySLUGFTcubed{0.0017639}
\def\myAirDensityRatio{0.7421}
\def\myAirDynamicViscosiyMTSecKG{1.694e-05}
\def\myAirTemperatureK{268.66}
\def\myAirSoundSpeedMTSec{328.55378}
\def\myAirSoundSpeedFTSec{1077.93236}
\def\myPrepareForStackNull{-1e+307}
\def\myInitialValueB{999}
\def\myMach{0.147}
\def\myFlightSpeedMTSec{48.368}
\def\myFlightSpeedFTSec{158.68767}
\def\myFlightSpeedKMH{174.1}
\def\myFlightSpeedKTS{94.02}
\def\myFlightSpeedEASMTSec{41.66667}
\def\myFlightSpeedEASFTSec{136.70166}
\def\myFlightSpeedEASKMH{150}
\def\myFlightSpeedEASKTS{80.99}
\def\myFlightQbarPA{1063.36806}
\def\myFlightQbarBAR{0.011}
\def\myInitialValueB{999}
\def\myUBodyMTSec{48.368}
\def\myUBodyFTSec{158.68767}
\def\myUBodyKMH{174.1}
\def\myUBodyKTS{94.02}
\def\myVBodyMTSec{0}
\def\myVBodyFTSec{0}
\def\myVBodyKMH{0}
\def\myVBodyKTS{0}
\def\myWBodyMTSec{0}
\def\myWBodyFTSec{0}
\def\myWBodyKMH{0}
\def\myWBodyKTS{0}

%
\begin{myExampleX}{True and equivalent airspeed}{\ding{46}}% \ \Keyboard\ %
\label{example:Equivalent:Airspeed:Basic:A}
%
\noindent
An equivalent airspeed
$V_\mathrm{e}
 =\SI[round-precision=0]{\myFlightSpeedEASKMH}{km/h}
 =\SI[round-precision=1]{\myFlightSpeedEASMTSec}{m/s}$
corresponds to a true airspeed at altitude $V\equiv V_\mathrm{t}$.
%
At the altitude $h=\SI[round-precision=0]{\myAltitudeMT}{m}$ we have an air density
$\rho=\SI[round-precision=3]{\myAirDensityKGMTcubed}{kg/m^3}$ and the
true flight speed is
\[
V = 
  \frac{ V_\text{e} }{ \sqrt{\dfrac{\rho}{\rho_\mathrm{SL}}} }
  = 
  \frac{
    \SI[round-precision=1]{\myFlightSpeedEASMTSec}{m/s}
  }{
    \sqrt{
      \dfrac{
        \SI[round-precision=3]{\myAirDensityKGMTcubed}{kg/m^3}
      }{
        \SI[round-precision=3]{\myISAAirDensitySeaLevelKGMTcubed}{kg/m^3}
      }
    }
  }
= 
  \mathunderline{mydarkblue}{ \SI[round-precision=1]{\myFlightSpeedMTSec}{\meter/\second} }
  = \mathunderline{mydarkblue}{ \SI[round-precision=1]{\myFlightSpeedKMH}{\kilo\meter/\hour} }
\]
%
The above value of $V$ is, as expected, higher than $V_\mathrm{e}$.

\end{myExampleX}
%
%--END-EXERCISE:equivalent_airspeed_basic_1
%===============================================================================================

\end{document}