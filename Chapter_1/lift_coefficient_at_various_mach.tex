\documentclass[[12pt,twoside]{book}
\usepackage{_my_document_style}
\begin{document}
%
\def\myAltitudeM{4000.000000}
\def\myAltitudeFt{13123.359580}
\def\myMassKg{79000.000000}
\def\myMassLb{174165.187126}
\def\myWingSurfaceMTSD{124.600000}
\def\myWingSurfaceFTSD{4400.207473}
\def\myMach{0.300000}
\def\myMach{0.400000}
\def\myMach{0.500000}
\def\myMach{0.600000}
\def\myMach{0.700000}
\def\myMach{0.800000}
\def\mySoundSpeed{324.578740}
\def\mySpeedMS{97.373622}
\def\mySpeedMS{129.831496}
\def\mySpeedMS{162.289370}
\def\mySpeedMS{194.747244}
\def\mySpeedMS{227.205118}
\def\mySpeedMS{259.662992}
\def\myGravitationalAcceleration{9.810000}
\def\myDensityKGMC{0.819129}
\def\myDynamicPressurePa{3883.338066}
\def\myDynamicPressurePa{6903.712118}
\def\myDynamicPressurePa{10787.050184}
\def\myDynamicPressurePa{15533.352264}
\def\myDynamicPressurePa{21142.618360}
\def\myDynamicPressurePa{27614.848470}
\def\myDynamicPressureBar{0.038833}
\def\myDynamicPressureBar{0.069037}
\def\myDynamicPressureBar{0.107871}
\def\myDynamicPressureBar{0.155334}
\def\myDynamicPressureBar{0.211426}
\def\myDynamicPressureBar{0.276148}
\def\myLiftCoefficient{1.601669}
\def\myLiftCoefficient{0.900939}
\def\myLiftCoefficient{0.576601}
\def\myLiftCoefficient{0.400417}
\def\myLiftCoefficient{0.294184}
\def\myLiftCoefficient{0.225235}

%
\begin{myExampleX}{Lift coefficient at various Mach numbers}{\ding{46}}% \ \Keyboard\ % 
\label{example:Lift:Coefficient:At:Various:Mach}
%
\noindent
Based on the data of the previous example, at an altitude of $h=\SI[round-precision=0]{\myAltitudeM}{m}$, for an aircraft of mass $m=\SI[round-precision=0]{\myMassKg}{kg}$, we want to calculate the lift coefficients in equilibrium level flight for different values of the flight Mach number, $M = 0.3\,,\, 0.4\,,\, 0.5\,,\, 0.6\,,\, 0.7\,,\, 0.8$. The trend of the lift coefficient as a function of Mach is observed in \figurename~\vref{fig:CL:at:Various:Mach}.

\medskip
\noindent
{\color{red} \bf Inserire le formule di calcolo ed un esempio di calcolo, per $M=0.5$.}
\pgfplotstableread[%
  header=false, col sep=space, row sep=newline, precision=6
  ]{Chapter_1/lift_coefficient_at_various_mach/dataset.txt}\myTableDataset

\pgfplotstablegetelem{2}{[index]0}\of{\myTableDataset}
\pgfmathsetmacro{\valueTwoZero}{\pgfplotsretval}
% \num[round-precision=2]{\valueTwoZero}

\pgfplotstablegetelem{2}{[index]1}\of{\myTableDataset}
\pgfmathsetmacro{\valueTwoOne}{\pgfplotsretval}

\pgfplotstablegetelem{2}{[index]2}\of{\myTableDataset}
\pgfmathsetmacro{\valueTwoTwo}{\pgfplotsretval}

\pgfplotstablegetelem{2}{[index]3}\of{\myTableDataset}
\pgfmathsetmacro{\valueTwoThree}{\pgfplotsretval}

\pgfplotstablegetelem{2}{[index]4}\of{\myTableDataset}
\pgfmathsetmacro{\valueTwoFour}{\pgfplotsretval}
%

Let's make as example, the calculation in case the Mach number is $M=\SI[round-precision=2]{\valueTwoZero}{}$.
We know that, being the speed of sound at altitude $h=\SI[round-precision=0]{\myAltitudeM}{m}$ equal to $a=\SI[round-precision=2]{\mySoundSpeed}{\meter/\second}$ ( calculated as in example~\ref{example:Characteristics:Of:The:Air:At:A:Certain:Altitude}
we can obtain the flight speed as:
\[
V = M\cdot a =        \SI[round-precision=1]{\valueTwoZero}{} \cdot \SI[round-precision=2]{\mySoundSpeed}{\meter/\second} = \SI[round-precision=2]{\valueTwoTwo}{\meter/\second} = \SI[round-precision=2]{\valueTwoThree}{\km/\hour}
\]
Moreover we have a dynamic flight pressure
\[
\begin{split}
\bar{q} = \frac{1}{2} \rho V^2 
  & {}=
  \num{0.5} \cdot \SI[round-precision=3]{\myDensityKGMC}{\kilogram\,\meter^{-3}}
    \, (\SI[round-precision=2]{\valueTwoTwo}{\meter/\second}\big)^2
\\
  & {}=
    { 
      \SI[round-precision=3,scientific-notation=true]{\valueTwoTwo}{\newton\,\meter^{-2}}
    }
    = 
    { 
      \SI[round-precision=3,scientific-notation=true]{\valueTwoFour}{ bar } 
    }
\end{split}
\]
If we take an equilibrium flight, in which lift $L$ equals weight
$W=mg$, we have a lift coefficient
$C_L$ that is
\[
C_L=\frac{L}{\bar{q}S}=\frac{m g}{\bar{q}S}
  = \frac{
      \SI[round-precision=0]{\myMassKg}{\kilogram} \cdot \SI{9.81}{\meter\,\second^{-2}}
    }{
      \SI[round-precision=3,scientific-notation=true]{\valueTwoTwo}{\newton\,\meter^{-2}}
      \cdot \SI[round-precision=1]{\myWingSurfaceMTSD}{\meter^2} 
    }
  = { \SI[round-precision=2]{\valueTwoOne}{} }
\]

\begin{table}[H]
    \centering
    \caption{Main values obtained at the altitude $h=\SI[round-precision=0]{\myAltitudeM}{m}$, for an aircraft of mass $m=\SI[round-precision=0]{\myMassKg}{kg}$.}
    \label{tab:Lift:Coefficient:At:Various:Mach}
    \pgfplotstabletypeset[ % local config, applies only for this table
      1000 sep={},
      columns/0/.style={
        fixed,
        fixed zerofill,
        precision=1,
        %showpos,
        column type=r,
        column name=$M$},
      columns/1/.style={
        fixed,
        fixed zerofill,
        precision=2,
        %showpos,
        column type=r,
        column name=$C_L$},
        columns/2/.style={
        fixed,
        fixed zerofill,
        precision=2,
        %showpos,
        column type=r,
        column name=$V$ (\si{m/s})},
        columns/3/.style={
        fixed,
        fixed zerofill,
        precision=2,
        %showpos,
        column type=r,
        column name=$V$ (\si{km/h})},
        columns/4/.style={
        fixed,
        fixed zerofill,
        precision=2,
        %showpos,
        column type=r,
        column name=$ \bar{q}$ (\si{Pa})},
        every head row/.style={before row=\toprule, after row=\midrule},
        every last row/.style={after row=\bottomrule},
      ]{\myTableDataset}
\end{table}
\medskip
\begin{figure}
\centering
%
%====================================== code-for-tikzpicture
%
\resizebox{0.7\textwidth}{!}{%
%
\pgfplotsset{width=7cm,compat=1.8}
% set style options for annotations with pins (see bottom of tikzpicture)
\tikzset{%
   every pin/.style={draw=none,
                     fill=gray!10,
                     %rectangle,rounded corners=0pt,
                     font=\relsize{0}}
                 }
%
% axis style, ticks, etc
\pgfplotsset{every axis/.append style={
                    label style={font=\relsize{1}},
                    tick label style={font=\relsize{0}}  
                    }}
%
% The data in tabular format (csv)
\pgfplotstableread[col sep=space,row sep=newline,precision=6]{%
    Chapter_1/lift_coefficient_at_various_mach/dataset.txt}\myTableDataset
\begin{tikzpicture}
% this is the lowermost x axis
\begin{axis}[
    xmin=80,xmax=300,
    xshift=2.5cm, yshift=-1.2cm,
    width=0.55\linewidth,
    axis x line*=bottom,
    hide y axis,
    ymin=0, ymax=10,
    %xtick = {80,100,...,300},
    xlabel={$V$ (\si{m/s})}
]
\end{axis}
\begin{axis}[%
    %
    view={0}{90},
    width=0.75\linewidth,height=0.45\linewidth,
    %
    xmin=0, xmax=1,
    %xtick={0, 0.1,...,1},
    x tick label style={
    /pgf/number format/.cd,
        fixed,
        %fixed zerofill,
        precision=2,
    /tikz/.cd
    },
    xlabel={$M$},
    ymin=0, ymax=1.7,
    %ytick={-1, -0.9,...,1.0},
    y tick label style={
    /pgf/number format/.cd,
        fixed,
        %fixed zerofill,
        precision=2,
    /tikz/.cd
    },
    ylabel={$C_L$},
    ylabel style={at={(ticklabel cs:0.5)}},
    ylabel shift=3pt,% horizontal shift outwards
    grid=major, grid style={dashed, gray!30},
    %legend entries = {$\lambda_\mathrm{W}=1.00$, $0.50$, $0.25$},
    legend style={%
        draw=none,%gray!30,
        fill=gray!10,
        at={(0.02,0.98)},
        anchor=north west,
        cells={anchor=east}
        }
    ]
    \addplot[%
        only marks, 
        %mark=none, 
        mark options={scale=1},
        ] table [x index=0, y index=1] from \myTableDataset
        ;
\end{axis}
\end{tikzpicture}
%
}% end-of-resizebox
%
%====================================== end-code-for-tikzpicture
    \caption{Equilibrium flight values of $C_L$ at different Mach numbers, at the altitude $h=\SI[round-precision=0]{\myAltitudeM}{m}$, for an aircraft of mass $m=\SI[round-precision=0]{\myMassKg}{kg}$.}
    \label{fig:CL:at:Various:Mach}
\end{figure}
\end{myExampleX}
\end{document}