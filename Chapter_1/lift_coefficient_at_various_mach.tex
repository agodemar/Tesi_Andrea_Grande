\documentclass[[12pt,twoside]{book}
\usepackage{_my_document_style}

\begin{document}

%
\def\myAltitudeM{4000.000000}
\def\myAltitudeFt{13123.359580}
\def\myMassKg{79000.000000}
\def\myMassLb{174165.187126}
\def\myWingSurfaceMTSD{124.600000}
\def\myWingSurfaceFTSD{4400.207473}
\def\myMach{0.300000}
\def\myMach{0.400000}
\def\myMach{0.500000}
\def\myMach{0.600000}
\def\myMach{0.700000}
\def\myMach{0.800000}
\def\mySoundSpeed{324.578740}
\def\mySpeedMS{97.373622}
\def\mySpeedMS{129.831496}
\def\mySpeedMS{162.289370}
\def\mySpeedMS{194.747244}
\def\mySpeedMS{227.205118}
\def\mySpeedMS{259.662992}
\def\myGravitationalAcceleration{9.810000}
\def\myDensityKGMC{0.819129}
\def\myDynamicPressurePa{3883.338066}
\def\myDynamicPressurePa{6903.712118}
\def\myDynamicPressurePa{10787.050184}
\def\myDynamicPressurePa{15533.352264}
\def\myDynamicPressurePa{21142.618360}
\def\myDynamicPressurePa{27614.848470}
\def\myDynamicPressureBar{0.038833}
\def\myDynamicPressureBar{0.069037}
\def\myDynamicPressureBar{0.107871}
\def\myDynamicPressureBar{0.155334}
\def\myDynamicPressureBar{0.211426}
\def\myDynamicPressureBar{0.276148}
\def\myLiftCoefficient{1.601669}
\def\myLiftCoefficient{0.900939}
\def\myLiftCoefficient{0.576601}
\def\myLiftCoefficient{0.400417}
\def\myLiftCoefficient{0.294184}
\def\myLiftCoefficient{0.225235}

%
\begin{myExampleX}{Lift coefficient at various Mach numbers}{\ding{46}}% \ \Keyboard\ % 
\label{example:Lift:Coefficient:At:Various:Mach}
%

\noindent
Based on the data of the previous example, at an altitude of h=\SI[round-precision=0]{\myAltitudeM}{m} we can calculate the lift coefficients for different flight Mach numbers M = {0.3 ; 0.4  ; 0.5 ; 0.6 ; 0.7 ; 0.8 } and graphically observe the trend of the lift coefficient as a function of Mach.

\pgfplotstableread[%
  header=false, col sep=space, row sep=newline, precision=6
  ]{Chapter_1/lift_coefficient_at_various_mach/dataset.txt}\myTableDataset

\pgfplotstablegetelem{1}{[index]1}\of{\myTableDataset}
\pgfmathsetmacro{\valueOneOne}{\pgfplotsretval}
\num[round-precision=2]{\valueOneOne}

\begin{table}[H]
    \centering
    \caption{Main values obtained at the given altitude.}
    \label{tab:Lift:Coefficient:At:Various:Mach}
    \pgfplotstabletypeset[ % local config, applies only for this table
      1000 sep={},
      columns/0/.style={
        fixed,
        fixed zerofill,
        precision=1,
        %showpos,
        column type=r,
        column name=$M$},
      columns/1/.style={
        fixed,
        fixed zerofill,
        precision=2,
        %showpos,
        column type=r,
        column name=$C_L$},
        columns/2/.style={
        fixed,
        fixed zerofill,
        precision=2,
        %showpos,
        column type=r,
        column name=$V$ (\si{m/s})},
        columns/3/.style={
        fixed,
        fixed zerofill,
        precision=2,
        %showpos,
        column type=r,
        column name=$V$ (\si{km/h})},
        columns/4/.style={
        fixed,
        fixed zerofill,
        precision=2,
        %showpos,
        column type=r,
        column name=$ \bar{q}$ (\si{Pa})},
        every head row/.style={before row=\toprule, after row=\midrule},
        every last row/.style={after row=\bottomrule},
      ]{\myTableDataset}
\end{table}

\medskip

\begin{figure}
    \centering
%
%
%====================================== code-for-tikzpicture
%
\resizebox{0.7\textwidth}{!}{%
%
\pgfplotsset{width=7cm,compat=1.8}
% set style options for annotations with pins (see bottom of tikzpicture)
\tikzset{%
   every pin/.style={draw=none,
                     fill=gray!10,
                     %rectangle,rounded corners=0pt,
                     font=\relsize{0}}
                 }
%
% axis style, ticks, etc
\pgfplotsset{every axis/.append style={
                    label style={font=\relsize{1}},
                    tick label style={font=\relsize{0}}  
                    }}
%
% The data in tabular format (csv)
\pgfplotstableread[col sep=space,row sep=newline, precision=6]{Chapter_1/lift_coefficient_at_various_mach/dataset.txt}\myTableDataset
\begin{tikzpicture}
  \begin{axis}[%
    %
    view={0}{90},
    width=0.75\linewidth,height=0.75\linewidth,
    %
    xmin=0, xmax=1,
    %xtick={0, 0.1,...,1},
    x tick label style={
        /pgf/number format/.cd,
            fixed,
            %fixed zerofill,
            precision=2,
        /tikz/.cd
    },
    xlabel={$M$},
    ymin=0, ymax=1.7,
    %ytick={-1, -0.9,...,1.0},
    y tick label style={
        /pgf/number format/.cd,
            fixed,
            %fixed zerofill,
            precision=2,
        /tikz/.cd
    },
    ylabel={$C_L$},
    ylabel style={at={(ticklabel cs:0.5)}},
    ylabel shift=3pt,% horizontal shift outwards
    grid=major, grid style={dashed, gray!30},
    %legend entries = {$\lambda_\mathrm{W}=1.00$, $0.50$, $0.25$},
    legend style={%
      draw=none,%gray!30,
      fill=gray!10,
      at={(0.02,0.98)},
      anchor=north west,
      cells={anchor=east}
    }
  ]
    \addplot[%
      only marks, 
      %mark=none, 
      mark options={scale=1},
      ] table [x index=0, y index=1] from \myTableDataset
;
  \end{axis}
\end{tikzpicture}
%
}% end-of-resizebox
%
%====================================== end-code-for-tikzpicture
    \caption{Insert caption here}
    \label{fig:my_label}
\end{figure}



\end{myExampleX}

\end{document}