\documentclass[[12pt,twoside]{book}
\usepackage{_my_document_style}

\begin{document}

%
\def\myAltitudeM{4000.000000}
\def\myAltitudeFt{13123.359580}
\def\myMassKg{79000.000000}
\def\myMassLb{174165.187126}
\def\myWingSurfaceMTSD{124.600000}
\def\myWingSurfaceFTSD{4400.207473}
\def\myMach{0.300000}
\def\myMach{0.400000}
\def\myMach{0.500000}
\def\myMach{0.600000}
\def\myMach{0.700000}
\def\myMach{0.800000}
\def\mySoundSpeed{324.578740}
\def\mySpeedMS{97.373622}
\def\mySpeedMS{129.831496}
\def\mySpeedMS{162.289370}
\def\mySpeedMS{194.747244}
\def\mySpeedMS{227.205118}
\def\mySpeedMS{259.662992}
\def\myGravitationalAcceleration{9.810000}
\def\myDensityKGMC{0.819129}
\def\myDynamicPressurePa{3883.338066}
\def\myDynamicPressurePa{6903.712118}
\def\myDynamicPressurePa{10787.050184}
\def\myDynamicPressurePa{15533.352264}
\def\myDynamicPressurePa{21142.618360}
\def\myDynamicPressurePa{27614.848470}
\def\myDynamicPressureBar{0.038833}
\def\myDynamicPressureBar{0.069037}
\def\myDynamicPressureBar{0.107871}
\def\myDynamicPressureBar{0.155334}
\def\myDynamicPressureBar{0.211426}
\def\myDynamicPressureBar{0.276148}
\def\myLiftCoefficient{1.601669}
\def\myLiftCoefficient{0.900939}
\def\myLiftCoefficient{0.576601}
\def\myLiftCoefficient{0.400417}
\def\myLiftCoefficient{0.294184}
\def\myLiftCoefficient{0.225235}

%
\begin{myExampleX}{Lift coefficient at various mach } {\ding{46}}% \ \Keyboard\ % 
\label{example:Equivalent:Airspeed:Basic:B}
%

\noindent
Based on the data of the previous example, at an altitude of h=\SI[round-precision=0]{\myAltitudeM}{m} calculate the lift coefficients for different flight Mach numbers M= {0.3 ; 0.4  ; 0.5 ; 0.6 ; 0.7 ; 0.8 }
we  have a dynamic flight pressure
\[
\begin{split}
\bar{q} = \frac{1}{2} \rho V^2 
  & {}=
  \num{0.5} \cdot \SI[round-precision=3]{\myDensityKGMC}{\kilogram\,\meter^{-3}}
    \, (\SI[round-precision=2]{\mySpeedMS}{\meter/\second}\big)^2
\\
  & {}=
    { 
      \SI[round-precision=3,scientific-notation=true]{\myDynamicPressurePa}{\newton\,\meter^{-2}} 
    }
    
\end{split}
\]


















\end{myExampleX}

\end{document}