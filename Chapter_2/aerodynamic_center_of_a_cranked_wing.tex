\documentclass[[12pt,twoside]{book}
\usepackage{_my_document_style}
\begin{document}
%
\def\myInitialValueA{999}
\def\myISAAirGasConstNMTKGK{287}
\def\myISAAirAdiabaticIndex{1.4}
\def\myISAAirTemperatureSeaLevelK{288.16}
\def\myISAAirDensitySeaLevelKGMTcubed{1.225}
\def\myISAAirDensitySeaLevelSLUGFTcubed{0.0023769}
\def\myAltitudeMT{3000}
\def\myAltitudeFT{9842.52}
\def\myISALapseRateKMT{-0.0065}
\def\myAirDensityKGMTcubed{0.9091}
\def\myAirDensitySLUGFTcubed{0.0017639}
\def\myAirDensityRatio{0.7421}
\def\myAirDynamicViscosiyMTSecKG{1.694e-05}
\def\myAirTemperatureK{268.66}
\def\myAirSoundSpeedMTSec{328.55378}
\def\myAirSoundSpeedFTSec{1077.93236}
\def\myPrepareForStackNull{-1e+307}
\def\myInitialValueB{999}
\def\myMach{0.147}
\def\myFlightSpeedMTSec{48.368}
\def\myFlightSpeedFTSec{158.68767}
\def\myFlightSpeedKMH{174.1}
\def\myFlightSpeedKTS{94.02}
\def\myFlightSpeedEASMTSec{41.66667}
\def\myFlightSpeedEASFTSec{136.70166}
\def\myFlightSpeedEASKMH{150}
\def\myFlightSpeedEASKTS{80.99}
\def\myFlightQbarPA{1063.36806}
\def\myFlightQbarBAR{0.011}
\def\myInitialValueB{999}
\def\myUBodyMTSec{48.368}
\def\myUBodyFTSec{158.68767}
\def\myUBodyKMH{174.1}
\def\myUBodyKTS{94.02}
\def\myVBodyMTSec{0}
\def\myVBodyFTSec{0}
\def\myVBodyKMH{0}
\def\myVBodyKTS{0}
\def\myWBodyMTSec{0}
\def\myWBodyFTSec{0}
\def\myWBodyKMH{0}
\def\myWBodyKTS{0}

%
% \begin{figure}[t]%[H]%[!htbp]
%   \centering
%   %\checkoddpage
%   %\centering
%     \includegraphics[width=0.72\textwidth]{exercises/wing_ac_cranked_1/wing_ac_cranked_1_drawing_panels.pdf}%
%   \caption{\finalhyphendemerits=1000
%           Ala \emph{cranked} proposta nell'esempio~\ref{example:Wing:Aerodynamic:Center:Cranked}.
%           Sono evidenziati il pannello interno $ABB'A'$, il pannello esterno $BCC'B'$ e il pannello esterno esteso $DCC'D'$ 
%           (\emph{constructed panel}) utilizzato per il calcolo del centro aerodinamico dell'ala.
%   }
%   \label{fig:Wing:Aerodynamic:Center:Cranked:Panels}%
% \end{figure}
% %
% \begin{figure}[t]%[H]%[!htbp]
%   %\centering
%   %\checkoddpage
%   %\centering
%     \includegraphics[width=0.78\textwidth]{exercises/wing_ac_cranked_1/wing_ac_cranked_1_drawing_mac.pdf}%
%   \caption{\finalhyphendemerits=1000
%           Corda media aerodinamica dell'ala proposta nell'esempio~\ref{example:Wing:Aerodynamic:Center:Cranked}.
%   }
%   \label{fig:Wing:Aerodynamic:Center:Cranked:MAC}%
% \end{figure}
%
\begin{myExampleX}{Centro aerodinamico di un'ala cranked}{\ding{46}\ \myIconGraph\ }% \ \Keyboard\ %
\label{example:Wing:Aerodynamic:Center:Cranked}
%
\noindent
In this example the position of the aerodynamic center will be calculated
of a wing similar to that of a Boeing~787.
The wing considered is of type \emph{cranked}, as can be seen from the planform shown in
figure~\ref{fig:Wing:Aerodynamic:Center:Cranked:Panels},and has the following characteristics:

\smallskip
\noindent
\adjustbox{center=\textwidth}{%
 \underline{\emph{Pannello 1} (inner panel)}
}

\smallskip
\noindent
\adjustbox{left=\textwidth}{%
  \adjustbox{right=0.39\textwidth}{%
    \emph{chords}:
  }\rule{0.5em}{0pt}% --> SPACER
  \adjustbox{left=0.59\textwidth}{%
    $c_{\mathrm{r},1}=\SI[round-precision=2]{\myChordRootWingIMT}{\metre}$,
    $c_{\mathrm{t},1}=\SI[round-precision=2]{\myChordTipWingIMT}{\metre}$,
    $\lambda_1 = \SI[round-precision=2]{\myTaperRatioWingI}{}$,
  }%
}

\smallskip
\noindent
\adjustbox{left=\textwidth}{%
  \adjustbox{right=0.39\textwidth}{%
    \emph{wingspan and wing surface}:
  }\rule{0.5em}{0pt}% --> SPACER
  \adjustbox{left=0.59\textwidth}{%
    $b_1=\SI[round-precision=2]{\mySpanWingIMT}{\metre}$,
    $S_1=\SI[round-precision=2]{\myAreaWingIMTsquared}{\meter^2}$,
    $\AR_1 = \SI[round-precision=2]{\myAspectRatioWingI}{}$,
  }%
}

\smallskip
\noindent
\adjustbox{left=\textwidth}{%
  \adjustbox{right=0.39\textwidth}{%
    \emph{sweep angles}:
  }\rule{0.5em}{0pt}% --> SPACER
  \adjustbox{left=0.59\textwidth,minipage=[t]{0.59\textwidth}}{%
    $\Lambda_\mathrm{le,1}=\SI[round-precision=4]{\mySweepLEWingIRAD}{\radian}=\SI[round-precision=1]{\mySweepLEWingIDEG}{\deg}$,
    \\
    $\Lambda_\mathrm{te,1}=\SI[round-precision=4]{\mySweepTEWingIRAD}{\radian}=\SI[round-precision=1]{\mySweepTEWingIDEG}{\deg}$,
  }%
}

\smallskip
\noindent
\adjustbox{left=\textwidth}{%
  \adjustbox{right=0.39\textwidth}{%
    \emph{gradient $C_{\ell_\mathlarger{\alpha}}$ of profile}:
  }\rule{0.5em}{0pt}% --> SPACER
  \adjustbox{left=0.59\textwidth,minipage=[t]{0.59\textwidth}}{%
    $C_{\ell_\mathlarger{\alpha},\mathrm{r},1}=\SI[round-precision=2]{\myCLAlphaRootWingIRAD}{\radian^{-1}}=\SI[round-precision=4]{\myCLAlphaRootWingIDEG}{\deg^{-1}}$,
    \\
    $C_{\ell_\mathlarger{\alpha},\mathrm{t},1}=\SI[round-precision=2]{\myCLAlphaTipWingIRAD}{\radian^{-1}}=\SI[round-precision=4]{\myCLAlphaTipWingIDEG}{\deg^{-1}}$,
  }%
}

\smallskip
\noindent
\adjustbox{left=\textwidth}{%
  \adjustbox{right=0.39\textwidth}{%
    \emph{twist}:
  }\rule{0.5em}{0pt}% --> SPACER
  \adjustbox{left=0.59\textwidth,minipage=[t]{0.59\textwidth}}{%
    $\alpha_{0\ell,\mathrm{r},1}=\SI[round-precision=3]{\myAlphaZeroLiftRootWingIRAD}{\radian}
      =\SI[round-precision=1]{\myAlphaZeroLiftRootWingIDEG}{\deg}$,\\
    $\alpha_{0\ell,\mathrm{t},1}=\SI[round-precision=3]{\myAlphaZeroLiftTipWingIRAD}{\radian}
      =\SI[round-precision=1]{\myAlphaZeroLiftTipWingIDEG}{\deg}$,\\
    $\epsilon_{\mathrm{g,t},1}=\SI[round-precision=3]{\myTwistWingIRAD}{\radian}
      =\SI[round-precision=1]{\myTwistWingIDEG}{\deg}$,
  }%
}

\smallskip
\noindent
\adjustbox{left=\textwidth}{%
  \adjustbox{right=0.39\textwidth}{%
    \emph{aerodynamic profile centers}:
  }\rule{0.5em}{0pt}% --> SPACER
  \adjustbox{left=0.59\textwidth}{%
    $\bar{x}_{\mathrm{ac,2D,1},\mathrm{r}}=\SI[round-precision=2]{\myXsiacRootWingI}{}$,
    $\bar{x}_{\mathrm{ac,2D,1},\mathrm{t}}=\SI[round-precision=2]{\myXsiacTipWingI}{}$,
  }%
}

\smallskip
\noindent
\adjustbox{left=\textwidth}{%
  \adjustbox{right=0.39\textwidth}{%
    \emph{coefficients $C_{\mathcal{m}_\mathlarger{\mathrm{ac}}}$ of profile}:
  }\rule{0.5em}{0pt}% --> SPACER
  \adjustbox{left=0.59\textwidth}{%
    $C_{\mathcal{m}_\mathlarger{\mathrm{ac}},\mathrm{r},1}=\SI[round-precision=3]{\myCmZeroRootWingI}{}$,
    $C_{\mathcal{m}_\mathlarger{\mathrm{ac}},\mathrm{t},1}=\SI[round-precision=3]{\myCmZeroTipWingI}{}$,
  }%
}

\medskip
\noindent
\adjustbox{center=\textwidth}{%
 \underline{\emph{Pannello 2} (outer pannel)}
}

\smallskip
\noindent
\adjustbox{left=\textwidth}{%
  \adjustbox{right=0.39\textwidth}{%
    \emph{chords}:
  }\rule{0.5em}{0pt}% --> SPACER
  \adjustbox{left=0.59\textwidth}{%
    $c_{\mathrm{r},2}=\SI[round-precision=2]{\myChordRootWingIIMT}{\metre}$,
    $c_{\mathrm{t},2}=\SI[round-precision=2]{\myChordTipWingIIMT}{\metre}$,
    $\lambda_2 = \SI[round-precision=2]{\myTaperRatioWingII}{}$,
  }%
}

\smallskip
\noindent
\adjustbox{left=\textwidth}{%
  \adjustbox{right=0.39\textwidth}{%
    \emph{wingspan and wing surface}:
  }\rule{0.5em}{0pt}% --> SPACER
  \adjustbox{left=0.59\textwidth}{%
    $b_2=\SI[round-precision=2]{\mySpanWingIIMT}{\metre}$,
    $S_2=\SI[round-precision=2]{\myAreaWingIIMTsquared}{\meter^2}$,
    $\AR_2 = \SI[round-precision=2]{\myAspectRatioWingII}{}$,
  }%
}

\smallskip
\noindent
\adjustbox{left=\textwidth}{%
  \adjustbox{right=0.39\textwidth}{%
    \emph{sweep angles}:
  }\rule{0.5em}{0pt}% --> SPACER
  \adjustbox{left=0.59\textwidth,minipage=[t]{0.59\textwidth}}{%
    $\Lambda_\mathrm{le,2}=\SI[round-precision=4]{\mySweepLEWingIIRAD}{\radian}=\SI[round-precision=1]{\mySweepLEWingIIDEG}{\deg}$,
    \\
    $\Lambda_\mathrm{te,2}=\SI[round-precision=4]{\mySweepTEWingIIRAD}{\radian}=\SI[round-precision=1]{\mySweepTEWingIIDEG}{\deg}$,
  }%
}

\smallskip
\noindent
\adjustbox{left=\textwidth}{%
  \adjustbox{right=0.39\textwidth}{%
    \emph{gradients $C_{\ell_\mathlarger{\alpha}}$ of profile}:
  }\rule{0.5em}{0pt}% --> SPACER
  \adjustbox{left=0.59\textwidth,minipage=[t]{0.59\textwidth}}{%
    $C_{\ell_\mathlarger{\alpha},\mathrm{r},2}=\SI[round-precision=2]{\myCLAlphaRootWingIIRAD}{\radian^{-1}}=\SI[round-precision=4]{\myCLAlphaRootWingIIDEG}{\deg^{-1}}$,
    \\
    $C_{\ell_\mathlarger{\alpha},\mathrm{t},2}=\SI[round-precision=2]{\myCLAlphaTipWingIIRAD}{\radian^{-1}}=\SI[round-precision=4]{\myCLAlphaTipWingIIDEG}{\deg^{-1}}$,
  }%
}

\smallskip
\noindent
\adjustbox{left=\textwidth}{%
  \adjustbox{right=0.39\textwidth}{%
    \emph{twist}:
  }\rule{0.5em}{0pt}% --> SPACER
  \adjustbox{left=0.59\textwidth,minipage=[t]{0.59\textwidth}}{%
    $\alpha_{0\ell,\mathrm{r},2}=\SI[round-precision=3]{\myAlphaZeroLiftRootWingIIRAD}{\radian}
      =\SI[round-precision=1]{\myAlphaZeroLiftRootWingIIDEG}{\deg}$,\\
    $\alpha_{0\ell,\mathrm{t},2}=\SI[round-precision=3]{\myAlphaZeroLiftTipWingIIRAD}{\radian}
      =\SI[round-precision=1]{\myAlphaZeroLiftTipWingIIDEG}{\deg}$,\\
    $\epsilon_{\mathrm{g,t},2}=\SI[round-precision=3]{\myTwistWingIIRAD}{\radian}
      =\SI[round-precision=1]{\myTwistWingIIDEG}{\deg}$,
  }%
}

\smallskip
\noindent
\adjustbox{left=\textwidth}{%
  \adjustbox{right=0.39\textwidth}{%
    \emph{aerodynamic profile centers}:
  }\rule{0.5em}{0pt}% --> SPACER
  \adjustbox{left=0.59\textwidth}{%
    $\bar{x}_{\mathrm{ac,2D,2},\mathrm{r}}=\SI[round-precision=2]{\myXsiacRootWingII}{}$,
    $\bar{x}_{\mathrm{ac,2D,2},\mathrm{t}}=\SI[round-precision=2]{\myXsiacTipWingII}{}$,
  }%
}

\smallskip
\noindent
\adjustbox{left=\textwidth}{%
  \adjustbox{right=0.39\textwidth}{%
    \emph{coefficients $C_{\mathcal{m}_\mathlarger{\mathrm{ac}}}$ of profile}:
  }\rule{0.5em}{0pt}% --> SPACER
  \adjustbox{left=0.59\textwidth}{%
    $C_{\mathcal{m}_\mathlarger{\mathrm{ac}},\mathrm{r},2}=\SI[round-precision=3]{\myCmZeroRootWingII}{}$,
    $C_{\mathcal{m}_\mathlarger{\mathrm{ac}},\mathrm{t},2}=\SI[round-precision=3]{\myCmZeroTipWingII}{}$,
  }%
}

\medskip
\noindent
\adjustbox{left=\textwidth}{%
  \adjustbox{right=0.39\textwidth}{%
    \emph{flight condition}:
  }\rule{0.5em}{0pt}% --> SPACER
  \adjustbox{left=0.59\textwidth}{%
    $\Mach = \SI[round-precision=2]{\myMach}{}$.
  }%
}

%-----------------------------------------------------------------------------------------------
% \begin{figure}%
%   [t]%[H]%[!htbp]
%   %\centering
%   %\checkoddpage
%   %\centering
%     \includegraphics[width=0.72\textwidth]{exercises/wing_ac_cranked_1/wing_ac_cranked_1_drawing_panels.pdf}%
%   \caption{\finalhyphendemerits=1000
%           Ala \emph{cranked} proposta nell'esempio~\ref{example:Wing:Aerodynamic:Center:Cranked}.
%           Sono evidenziati il pannello interno $ABB'A'$, il pannello esterno $BCC'B'$ e il pannello esterno esteso $DCC'D'$ 
%           (\emph{constructed panel}) utilizzato per il calcolo del centro aerodinamico dell'ala.
%   }
%   \label{fig:Wing:Aerodynamic:Center:Cranked:Panels}%
% \end{figure}%
%-----------------------------------------------------------------------------------------------

\bigskip
From the data it is easy to obtain the law of the chords:
\[
c(Y)=
\begin{cases}
%\left\{
%\begin{array}{cl}
c_1(Y) = A_{c,1} \, Y + B_{c,1} & \text{for }\makebox[3em][r]{$0$}     \le Y \le \frac{1}{2}b_1
\\[4pt]
c_2(Y) = A_{c,2} \, \bigg(Y-\dfrac{b_1}{2}\bigg) + B_{c,2} & \text{for }\makebox[3em][r]{$\frac{1}{2}b_1$}< Y \le \frac{1}{2}b
\end{cases}
%\right.
\]
with
\[
A_{c,1}
  = \frac{c_{\mathrm{t},1} - c_{\mathrm{r},1}}{b_1/2}
  = 
    2 \frac{
      \SI[round-precision=2]{\myChordTipWingIMT}{\metre} - \SI[round-precision=2]{\myChordRootWingIMT}{\metre}
    }{
      \SI[round-precision=2]{\mySpanWingIMT}{\metre}
    }
  = \mathunderline{mydarkblue}{ \SI[round-precision=3]{\myCoeffAChordWingI}{} }
\]
\[
B_{c,1}
  = c_{\mathrm{r},1}
  = \mathunderline{mydarkblue}{ \SI[round-precision=2]{\myCoeffBChordWingIMT}{\metre} }
\]
\[
A_{c,2}
  = \frac{c_{\mathrm{t},2} - c_{\mathrm{r},2}}{b_2/2}
  = 
    2 \frac{
      \SI[round-precision=2]{\myChordTipWingIIMT}{\metre} - \SI[round-precision=2]{\myChordRootWingIIMT}{\metre}
    }{
      \SI[round-precision=2]{\mySpanWingIIMT}{\metre}
    }
  = \mathunderline{mydarkblue}{ \SI[round-precision=3]{\myCoeffAChordWingII}{} }
\]
\[
B_{c,2}
  = c_{\mathrm{r},2}
  = \mathunderline{mydarkblue}{ \SI[round-precision=2]{\myCoeffBChordWingIIMT}{\metre} }
\]
Therefore, we get
\[
c(Y)=
\begin{cases}
%\left\{
%\begin{array}{cl}
c_1(Y) = 
  \SI[round-precision=3]{\myCoeffAChordWingI}{} \, Y 
    + \SI[round-precision=2]{\myCoeffBChordWingIIMT}{\metre} 
  & \text{per }
    \makebox[3.5em][r]{$\SI[round-precision=0]{0}{\metre}$} 
      \le Y \le 
      \calcSI[round-precision=2,fixed-exponent=0,scientific-notation=fixed]{0.5*\mySpanWingIMT}{\metre}
\\[4pt]
c_2(Y) 
  = \SI[round-precision=3]{\myCoeffAChordWingII}{} \, 
    \big(
      Y
      - \calcSI[round-precision=2,fixed-exponent=0,scientific-notation=fixed]{0.5*\mySpanWingIMT}{\metre}
    \big)
    + \SI[round-precision=2]{\myCoeffBChordWingIIMT}{\metre} 
  & \text{per }
    \makebox[3.5em][r]{%
      $\calcSI[round-precision=2,fixed-exponent=0,scientific-notation=fixed]{0.5*\mySpanWingIMT}{\metre}$
    }% end-of-makebox
      < Y 
      \le \calcSI[round-precision=2,fixed-exponent=0,scientific-notation=fixed]{0.5*\mySpanWingMT}{\metre}
\end{cases}
%\right.
\]

From the figure~\ref{fig:Wing:Aerodynamic:Center:Cranked:Panels} are observed
the points $A$, $B$, $B'$ and $A'$ which define the inner portion of the wing.
The inner panel has a taper ratio
\[
\lambda_1
  =\frac{c_{\mathrm{t},1}}{c_{\mathrm{r},1}}
  =\frac{\SI[round-precision=2]{\myChordTipWingIMT}{\metre}}{\SI[round-precision=2]{\myChordRootWingIMT}{\metre}}
  =\mathunderline{mydarkblue}{ \SI[round-precision=2]{\myTaperRatioWingI}{} }
\]
a wing surface
\[
S_1 = \frac{b_1}{2} \, c_{\mathrm{r},1} \, \big( 1 + \lambda_1 \big)
  =
    \num{0.5} \cdot \SI[round-precision=1]{\mySpanWingIMT}{\metre}
      \cdot \SI[round-precision=2]{\myChordRootWingIMT}{\metre}
      \cdot \big( 1 + \SI[round-precision=2]{\myTaperRatioWingI}{} \big) 
    = \mathunderline{mydarkblue}{ \SI[round-precision=1]{\myAreaWingIMTsquared}{\metre^2} }
\]
an aspect ratio
\[
\AR_1 
  = \frac{b_1^2}{S_1}
  = \frac{\big(\SI[round-precision=1]{\mySpanWingIMT}{\metre}\big)^2}{\SI[round-precision=1]{\myAreaWingIMTsquared}{\metre^2}}
  = \mathunderline{mydarkblue}{ \num[round-precision=2]{\myAspectRatioWingI} }
\]
From these values a mean aerodynamic chord is obtained
\[
\begin{split}
\bar{c}_1 & {}= \frac{2}{3} \, c_{\mathrm{r},1} \, \frac{1+\lambda_1 + \lambda_1^2}{1+\lambda_1} \\
  & {}=
    \num{0.667} \cdot \SI[round-precision=2]{\myChordRootWingIMT}{\metre}
      \cdot 
        \frac{
          1 + \SI[round-precision=2]{\myTaperRatioWingI}{} + \SI[round-precision=2]{\myTaperRatioWingI}{}^2
        }{
          1 + \SI[round-precision=2]{\myTaperRatioWingI}{}
        }
    = \mathunderline{mydarkblue}{ \SI[round-precision=2]{\myMACWingIMT}{\metre} }
\end{split}
\]
The wing section of the inner panel having a chord $\bar{c}_1$ has a leading edge
abscissa
\[
\begin{split}
X_{\mathrm{le},\bar{c}_1} 
  & {}=
    \frac{b_1}{6} \, \frac{1+2\lambda_1}{1+\lambda_1} \tan\Lambda_\mathrm{le,1} \\[3pt]
  & {}=
    \frac{\SI[round-precision=1]{\mySpanWingIMT}{\metre}}{6}
      \cdot 
      \frac{
        1 + 2\cdot\SI[round-precision=2]{\myTaperRatioWingI}{}
      }{
        1 + \SI[round-precision=2]{\myTaperRatioWingI}{}
      }
      \cdot \tan \big( \SI[round-precision=3]{\mySweepLEWingIRAD}{\radian} \big)
    = \mathunderline{mydarkblue}{ \SI[round-precision=2]{\myXMACLEToApexWingIMT}{\metre} }
\end{split}
\]
corresponding to the station
\[
\begin{split}
Y_{\bar{c}_1} 
  & {}=
    \frac{b_1}{6} \, \frac{1+2\lambda_1}{1+\lambda_1} \\[3pt]
  & {}=
    \frac{\SI[round-precision=1]{\mySpanWingIMT}{\metre}}{6}
      \cdot 
      \frac{
        1 + 2\cdot\SI[round-precision=2]{\myTaperRatioWingI}{}
      }{
        1 + \SI[round-precision=2]{\myTaperRatioWingI}{}
      }
    = \mathunderline{mydarkblue}{ \SI[round-precision=2]{\myYMACWingIMT}{\metre} }
\end{split}
\]
along the wingspan.

Analogamente, dalla figura~\ref{fig:Wing:Aerodynamic:Center:Cranked:Panels} si osservano
i punti $B$, $C$, $C'$ e $B'$ che definiscono la porzione esterna dell'ala.
Per il pannello esterno si ha un rapporto di rastremazione
\[
\lambda_2
  =\frac{c_{\mathrm{t},2}}{c_{\mathrm{r},2}}
  =\frac{\SI[round-precision=2]{\myChordTipWingIIMT}{\metre}}{\SI[round-precision=2]{\myChordRootWingIIMT}{\metre}}
  =\mathunderline{mydarkblue}{ \SI[round-precision=2]{\myTaperRatioWingII}{} }
\]
una superficie
\[
S_2 = \frac{b_2}{2} \, c_{\mathrm{r},2} \, \big( 1 + \lambda_2 \big)
  =
    \num{0.5} \cdot \SI[round-precision=1]{\mySpanWingIIMT}{\metre}
      \cdot \SI[round-precision=2]{\myChordRootWingIIMT}{\metre}
      \cdot \big( 1 + \SI[round-precision=2]{\myTaperRatioWingII}{} \big) 
    = \mathunderline{mydarkblue}{ \SI[round-precision=1]{\myAreaWingIIMTsquared}{\metre^2} }
\]
e un allungamento
\[
\AR_2 
  = \frac{b_2^2}{S_2}
  = \frac{\big(\SI[round-precision=1]{\mySpanWingIIMT}{\metre}\big)^2}{\SI[round-precision=1]{\myAreaWingIIMTsquared}{\metre^2}}
  = \mathunderline{mydarkblue}{ \num[round-precision=2]{\myAspectRatioWingII} }
\]

Conseguentemente, si ha una corda media aerodinamica
\[
\begin{split}
\bar{c}_2 & {}= \frac{2}{3} \, c_{\mathrm{r},2} \, \frac{1+\lambda_2 + \lambda_2^2}{1+\lambda_2} \\
  & {}=
    \num{0.667} \cdot \SI[round-precision=2]{\myChordRootWingIIMT}{\metre}
      \cdot 
        \frac{
          1 + \SI[round-precision=2]{\myTaperRatioWingII}{} + \SI[round-precision=2]{\myTaperRatioWingII}{}^2
        }{
          1 + \SI[round-precision=2]{\myTaperRatioWingII}{}
        }
    = \mathunderline{mydarkblue}{ \SI[round-precision=2]{\myMACWingIIMT}{\metre} }
\end{split}
\]
Il bordo d'attacco del profilo di corda $\bar{c}_2$ dista dunque
\[
\begin{split}
X_{\mathrm{le},\bar{c}_2} - X_B
  & {}=
    \frac{b_2}{6} \, \frac{1+2\lambda_2}{1+\lambda_2} \tan\Lambda_\mathrm{le,2} \\[3pt]
  & {}=
    \frac{\SI[round-precision=1]{\mySpanWingIIMT}{\metre}}{6}
      \cdot 
      \frac{
        1 + 2\cdot\SI[round-precision=2]{\myTaperRatioWingII}{}
      }{
        1 + \SI[round-precision=2]{\myTaperRatioWingII}{}
      }
      \cdot \tan \big( \SI[round-precision=3]{\mySweepLEWingIIRAD}{\radian} \big)
    = \mathunderline{mydarkblue}{ \SI[round-precision=2]{\myXMACLEToApexWingIIMT}{\metre} }
\end{split}
\]
in senso longitudinale dal punto $B$. Pertanto
\[
\begin{split}
X_{\mathrm{le},\bar{c}_2} & {}= X_B + \SI[round-precision=2]{\myXMACLEToApexWingIIMT}{\metre}
  = \frac{b_1}{2} \tan \Lambda_{\mathrm{le},1} + \SI[round-precision=2]{\myXMACLEToApexWingIIMT}{\metre}
\\
  & {}= \calcSI[round-precision=2,fixed-exponent=0,scientific-notation=fixed]{
          0.5 * \mySpanWingIMT
        }{\metre}
       \cdot \tan( \SI[round-precision=3]{\mySweepLEWingIRAD}{\radian} )
      + \SI[round-precision=2]{\myXMACLEToApexWingIIMT}{\metre}
    = \calcSI[round-precision=2,fixed-exponent=0,scientific-notation=fixed]{
          0.5 * \mySpanWingIMT * tan( \mySweepLEWingIRAD )
        }{\metre}
      + \SI[round-precision=2]{\myXMACLEToApexWingIIMT}{\metre}
    = \mathunderline{mydarkblue}{ 
      \calcSI[round-precision=2,fixed-exponent=0,scientific-notation=fixed]{
          0.5 * \mySpanWingIMT * tan( \mySweepLEWingIRAD )
          + \myXMACLEToApexWingIIMT
      }{\metre}
    }
\end{split}
\]
Conseguentemente, la stazione che individua lungo l'apertura il profilo del pannello
esterno di corda $\bar{c}_2$ è
\[
\begin{split}
Y_{\bar{c}_2} 
  & {}=
    \frac{b_1}{2} + 
    \frac{b_2}{6} \, \frac{1+2\lambda_2}{1+\lambda_2} \\[3pt]
  & {}=
    \calcSI[round-precision=2]{0.5*\mySpanWingIMT}{\metre} +
    \frac{\SI[round-precision=1]{\mySpanWingIIMT}{\metre}}{6}
      \cdot 
      \frac{
        1 + 2\cdot\SI[round-precision=2]{\myTaperRatioWingII}{}
      }{
        1 + \SI[round-precision=2]{\myTaperRatioWingII}{}
      }
    = \mathunderline{mydarkblue}{
      \calcSI[round-precision=2]{0.5*\mySpanWingIMT}{\metre} +
      \SI[round-precision=2]{\myYMACWingIIMT}{\metre} 
    }
    = \mathunderline{mydarkblue}{
      \calcSI[round-precision=2,fixed-exponent=0,scientific-notation=fixed]{0.5*\mySpanWingIMT + \myYMACWingIIMT}{\metre}
    }
\end{split}
\]

I calcoli precedenti permettono di determinare la corda media aerodinamica $\bar{c}$ dell'ala,
che ha superficie totale
\[
S = S_1 + S_2
  = \SI[round-precision=1]{\myAreaWingIMTsquared}{\metre^2}
    = \mathunderline{mydarkblue}{
      \SI[round-precision=1]{\myAreaWingCrankedMTsquared}{\metre^2}
    }
\]
e allungamento
\[
\AR = \frac{b^2}{S} 
    = \frac{
        \big( \SI[round-precision=2]{\mySpanWingMT}{\meter} \big)^2
      }{
        \SI[round-precision=1]{\myAreaWingCrankedMTsquared}{\metre^2}
      }
    = \mathunderline{mydarkblue}{
      \SI[round-precision=2]{\myAspectRatioWingCranked}{}
    }
\]
Il valore di $\bar{c}$ è il seguente:
\[
\bar{c} = \frac{S_1 \, \bar{c}_1 + S_2 \, \bar{c}_2} {S_1 + S_2}
  =
  \frac{\SI[round-precision=1]{\myAreaWingIMTsquared}{\metre^2} \cdot \SI[round-precision=2]{\myMACWingIMT}{\metre} + \SI[round-precision=1]{\myAreaWingIIMTsquared}{\metre^2} \cdot \SI[round-precision=2]{\myMACWingIIMT}{\metre}}{\SI[round-precision=1]{\myAreaWingIMTsquared}{\metre^2} + \SI[round-precision=1]{\myAreaWingIIMTsquared}{\metre^2}}
    = \mathunderline{mydarkblue}{ \SI[round-precision=2]{\myMACWingCrankedMT}{\metre} }
\]
Essendo in questo caso
$\bar{c} < c_{\mathrm{t},1}=\SI[round-precision=2]{\myMACWingIMT}{\metre}$,
si determina la stazione
\[
  Y_{\bar{c}} 
    = \frac{\bar{c} - B_{c,1}}{A_{c,1}}
    = \frac{
        \SI[round-precision=2]{\myMACWingCrankedMT}{\metre} 
        - \SI[round-precision=2]{\myCoeffBChordWingIMT}{\metre}
      }{
        \SI[round-precision=3]{\myCoeffAChordWingI}{}
      }
    = \mathunderline{mydarkblue}{
      \SI[round-precision=2]{\myYYMACWingCrankedMT}{\metre}
    }
\]
lungo l'apertura corrispondente al profilo di corda $\bar{c}$. Tale profilo
ha bordo d'attacco di ascissa
\[
X_{\mathrm{le},\bar{c}} 
  % = \frac{\bar{c} - B_{c,1}}{A_{c,1}} \tan \Lambda_{\mathrm{le},1}
  = Y_{\bar{c}} \tan \Lambda_{\mathrm{le},1}
  =
%	\frac{
%	\SI[round-precision=2]{\myMACWingCrankedMT}{\metre}
%	-
%	\SI[round-precision=2]{\myCoeffBChordWingIMT}{\metre}
%	}{
%	\SI[round-precision=2]{\myCoeffAChordWingI}{}
%	}
   \SI[round-precision=2]{\myYYMACWingCrankedMT}{\metre}
   \cdot
	\tan \big( \SI[round-precision=3]{\mySweepLEWingIRAD}{\radian} \big)
	= \mathunderline{mydarkblue}{ \SI[round-precision=2]{\myXXMACLEToApexWingCrankedMT}{\metre} }
\]

%-----------------------------------------------------------------------------------------------
% \begin{figure}
%   {SCfigure}[1.9]%
%   [t]%[H]%[!htbp]
%   %\centering
%   %\checkoddpage
%   %\centering
%     \includegraphics[width=0.78\textwidth]{exercises/wing_ac_cranked_1/wing_ac_cranked_1_drawing_mac.pdf}%
%   \caption{\finalhyphendemerits=1000
%           Corda media aerodinamica dell'ala proposta nell'esempio~\ref{example:Wing:Aerodynamic:Center:Cranked}.
%   }
%   \label{fig:Wing:Aerodynamic:Center:Cranked:MAC}%
% \end{figure}%

%-----------------------------------------------------------------------------------------------

Nella figura~\ref{fig:Wing:Aerodynamic:Center:Cranked:MAC}
è riportato il disegno della forma in pianta assegnata dove sono indicate le corde $\bar{c}_1$, $\bar{c}_2$ e $\bar{c}$
e le distanze $X_{\mathrm{le},\bar{c}}$ ed $Y_{\bar{c}}$.

\medskip
A questo punto si procede al calcolo del centro aerodinamico dell'ala assegnata.
Il metodo grafico utilizzato nell'esempio~\ref{example:Wing:Aerodynamic:Center:A}
per la determinazione del centro aerodinamico di un'ala a bordi dritti
può essere applicato qui a ciascuno dei due pannelli alari.
A differenza del caso di un'ala semplice, per un'ala \emph{cranked} bisogna ricavare 
un pannello 
esterno \emph{esteso} --- detto \emph{constructed outer panel} --- come mostrato nelle 
figure~\ref{fig:Wing:Aerodynamic:Center:Cranked:Panels}
e~\ref{fig:Wing:Aerodynamic:Center:Cranked:Panels:AC}. 
Esso si costruisce prolungando i bordi d'attacco
e d'uscita del pannello~2 verso l'interno,
partendo, rispettivamente, dai punti $B$ e $B'$, fino a incontrare nei punti $D$ e $D'$ la retta parallela 
all'asse $X$ di equazione $Y=\frac{1}{4}b_1$.
Pertanto, nei calcoli che seguono si considera il pannello interno
$\mathcal{P}_1=ABB'A'$ e,
al posto del pannello $\mathcal{P}_2=BCC'B$,
il pannello esterno esteso $\mathcal{P}_{2'}=DCC'D'$.

Per quanto riguarda il pannello interno $\mathcal{P}_{1}$,
dai dati e dalla figura~\ref{fig:Wing:Ac:K:One:Plots}
si legge
\[
\lambda_1=\SI[round-precision=2]{\myTaperRatioWingI}{}
%
%\quad \Longrightarrow \quad
%\quad \begin{array}{@{}c@{}} \myIconGraph \\[-4pt] \Longrightarrow \end{array} \quad
\adjustbox{center=4em}{%
  \adjustbox{lap=\width}{\raisebox{2.2ex}[0pt][0pt]{\myIconGraph}}$\Longrightarrow$%
}
%
K_{1,\mathcal{P}_{1}}
  = \mathunderline{mydarkblue}{ \SI[round-precision=3]{\myKOneACDatcomWingI}{} }
\]
dalla figura~\ref{fig:Wing:Ac:K:Two:Plots}
si legge
\[
\Lambda_\mathrm{le,1} = \SI[round-precision=1]{\mySweepLEWingIDEG}{\deg}\;,\;\,
\AR_1 = \SI[round-precision=1]{\myAspectRatioWingI}{}\;,\;\,
\lambda_1=\SI[round-precision=2]{\myTaperRatioWingI}{}
%\quad \Longrightarrow \quad
%\quad \begin{array}{@{}c@{}} \myIconGraph \\[-4pt] \Longrightarrow \end{array} \quad
\adjustbox{center=4em}{%
  \adjustbox{lap=\width}{\raisebox{2.2ex}[0pt][0pt]{\myIconGraph}}$\Longrightarrow$%
}
%
K_{2,\mathcal{P}_{1}} 
  = \mathunderline{mydarkblue}{ \SI[round-precision=3]{\myKTwoACDatcomWingI}{} }
\]
dalla figura~\ref{fig:Wing:Ac:X:AC:From:Apex:Plots}
si legge
\[
\Lambda_\mathrm{le,1} = \SI[round-precision=1]{\mySweepLEWingIDEG}{\deg}\;,\;\,
\Mach = \SI[round-precision=2]{\myMach}{}\;,\;\,
\AR_1 = \SI[round-precision=1]{\myAspectRatioWingI}{}\;,\;\,
\lambda_1=\SI[round-precision=2]{\myTaperRatioWingI}{}
%
%\quad \Longrightarrow \quad
%\quad \begin{array}{@{}c@{}} \myIconGraph \\[-4pt] \Longrightarrow \end{array} \quad
\adjustbox{center=4em}{%
  \adjustbox{lap=\width}{\raisebox{2.2ex}[0pt][0pt]{\myIconGraph}}$\Longrightarrow$%
}
%
\left.
\frac{X'_{\mathrm{ac}}}{c_\mathrm{r}}\right|_{\mathcal{P}_{1}}
  = \mathunderline{mydarkblue}{ \SI[round-precision=3]{\myXACOverChordRootDatcomWingI}{} }
\]
%
Si ottiene dunque
\[
\left.
\frac{x_{\mathrm{ac}}}{\bar{c}}\right|_{\mathcal{P}_{1}} 
  = K_{1,\mathcal{P}_{1}} 
    \left( 
      \left.\frac{X'_{\mathrm{ac}}}{c_\mathrm{r}}\right|_{\mathcal{P}_{1}} 
        - K_{2,\mathcal{P}_{1}}
    \right)
  = \SI[round-precision=3]{\myKOneACDatcomWingI}{} \,
    \Big(  
      \SI[round-precision=3]{\myXACOverChordRootDatcomWingI}{} 
        - \SI[round-precision=3]{\myKTwoACDatcomWingI}{}  
    \Big)
  = \mathunderline{mydarkblue}{ \SI[round-precision=3]{\myXsiACWingI}{} } 
\]

%-----------------------------------------------------------------------------------------------
% \begin{figure}
%   {SCfigure}[1.9]%
%   [t]%[H]%[!htbp]
%   %\centering
%   %\checkoddpage
%   %\centering
%     \includegraphics[width=0.78\textwidth]{exercises/wing_ac_cranked_1/wing_ac_cranked_1_drawing_panels_ac.pdf}%
%   \caption{\finalhyphendemerits=1000
%           Centri aerodinamici dei pannelli 
%           $\mathcal{P}_1=ABB'A'$ e $\mathcal{P}_{2'}=DCC'D'$.
%   }
%   \label{fig:Wing:Aerodynamic:Center:Cranked:Panels:AC}%
% \end{figure}

%-----------------------------------------------------------------------------------------------

Per quanto riguarda il pannello interno $\mathcal{P}_{2'}$,
si ricava
\[
b_{2'} = 2 \left( \frac{1}{2}b_2 + \frac{1}{4}b_1 \right)
  =
    2 \big(
    \calcSI[round-precision=2,fixed-exponent=0,scientific-notation=fixed]{0.5*\mySpanWingIIMT}{\metre}
    +
    \calcSI[round-precision=2,fixed-exponent=0,scientific-notation=fixed]{0.25*\mySpanWingIMT}{\metre}
    \big)
  = \SI[round-precision=2]{\mySpanWingIIPrimeMT}{m}
\]
\[
c_{\mathrm{r},2'}
  = A_{c,2} \, \bigg( - \frac{1}{4}b_1 \bigg) + B_{c,2}
  =
    \SI[round-precision=3]{\myCoeffAChordWingII}{} \, 
    \big(
      - \calcSI[round-precision=2,fixed-exponent=0,scientific-notation=fixed]{0.25*\mySpanWingIMT}{\metre}
    \big)
    + \SI[round-precision=2]{\myCoeffBChordWingIIMT}{\metre}
  = \SI[round-precision=2]{\myChordRootWingIIPrimeMT}{m}
\]
\[
\lambda_{2'}
  =\frac{c_{\mathrm{t},2}}{c_{\mathrm{r},2'}}
  =\frac{\SI[round-precision=2]{\myChordTipWingIIMT}{\metre}}{\SI[round-precision=2]{\myChordRootWingIIPrimeMT}{\metre}}
  =\mathunderline{mydarkblue}{ \SI[round-precision=2]{\myTaperRatioWingIIPrime}{} }
\]
\[
S_{2'} 
  = \frac{b_{2'}}{2} \, c_{\mathrm{r},{2'}} \, \big( 1 + \lambda_{2'} \big)
  =
    \num{0.5} \cdot \SI[round-precision=1]{\mySpanWingIIPrimeMT}{\metre}
      \cdot \SI[round-precision=2]{\myChordRootWingIIPrimeMT}{\metre}
      \cdot \big( 1 + \SI[round-precision=2]{\myTaperRatioWingIIPrime}{} \big) 
    = \mathunderline{mydarkblue}{ \SI[round-precision=1]{\myAreaWingIIPrimeMTsquared}{\metre^2} }
\]
\[
\AR_{2'} 
  = \frac{b_{2'}^2}{S_{2'}}
  = \frac{\big(\SI[round-precision=1]{\mySpanWingIIPrimeMT}{\metre}\big)^2}{\SI[round-precision=1]{\myAreaWingIIPrimeMTsquared}{\metre^2}}
  = \mathunderline{mydarkblue}{ \num[round-precision=2]{\myAspectRatioWingIIPrime} }
\]


Dai dati e dalla figura~\ref{fig:Wing:Ac:K:One:Plots}
si legge
\[
\lambda_{2'}=\SI[round-precision=2]{\myTaperRatioWingIIPrime}{}
%
%\quad \Longrightarrow \quad
%\quad \begin{array}{@{}c@{}} \myIconGraph \\[-4pt] \Longrightarrow \end{array} \quad
\adjustbox{center=4em}{%
  \adjustbox{lap=\width}{\raisebox{2.2ex}[0pt][0pt]{\myIconGraph}}$\Longrightarrow$%
}
%
K_{1,\mathcal{P}_{2'}}
  = \mathunderline{mydarkblue}{ \SI[round-precision=3]{\myKOneACDatcomWingII}{} }
\]
dalla figura~\ref{fig:Wing:Ac:K:Two:Plots}
si legge
\[
\Lambda_\mathrm{le,2} = \SI[round-precision=1]{\mySweepLEWingIIDEG}{\deg}\;,\;\,
\AR_{2'} = \SI[round-precision=1]{\myAspectRatioWingIIPrime}{}\;,\;\,
\lambda_{2'}=\SI[round-precision=2]{\myTaperRatioWingIIPrime}{}
%\quad \Longrightarrow \quad
%\quad \begin{array}{@{}c@{}} \myIconGraph \\[-4pt] \Longrightarrow \end{array} \quad
\adjustbox{center=4em}{%
  \adjustbox{lap=\width}{\raisebox{2.2ex}[0pt][0pt]{\myIconGraph}}$\Longrightarrow$%
}
%
K_{2,\mathcal{P}_{2'}} 
  = \mathunderline{mydarkblue}{ \SI[round-precision=3]{\myKTwoACDatcomWingII}{} }
\]
dalla figura~\ref{fig:Wing:Ac:X:AC:From:Apex:Plots}
si legge
\[
\Lambda_\mathrm{le,2} = \SI[round-precision=1]{\mySweepLEWingIIDEG}{\deg}\;,\;\,
\Mach = \SI[round-precision=2]{\myMach}{}\;,\;\,
\AR_{2'} = \SI[round-precision=1]{\myAspectRatioWingIIPrime}{}\;,\;\,
\lambda_{2'}=\SI[round-precision=2]{\myTaperRatioWingIIPrime}{}
%
%\quad \Longrightarrow \quad
%\quad \begin{array}{@{}c@{}} \myIconGraph \\[-4pt] \Longrightarrow \end{array} \quad
\adjustbox{center=4em}{%
  \adjustbox{lap=\width}{\raisebox{2.2ex}[0pt][0pt]{\myIconGraph}}$\Longrightarrow$%
}
%
\left.
\frac{X'_{\mathrm{ac}}}{c_\mathrm{r}}\right|_{\mathcal{P}_{2'}}
  = \mathunderline{mydarkblue}{ \SI[round-precision=3]{\myXACOverChordRootDatcomWingII}{} }
\]

A questo punto può calcolarsi la quantità
\[
\frac{X_{\mathrm{ac}}}{c_\mathrm{r}}
  = \frac{
    \left.\dfrac{X'_{\mathrm{ac}}}{c_\mathrm{r}}\right|_{\mathcal{P}_{1}}
      \, S_{1} \, C_{L_\mathlarger{\alpha}}\Big|_{\mathcal{P}_{1}}
    +
    \left.\dfrac{X'_{\mathrm{ac}}}{c_\mathrm{r}}\right|_{\mathcal{P}_{2'}}
      \, S_{2'} \, C_{L_\mathlarger{\alpha}}\Big|_{\mathcal{P}_{2'}}
  }{
    S_{1} \, C_{L_\mathlarger{\alpha}}\Big|_{\mathcal{P}_{2'}} 
    + S_{2'} \, C_{L_\mathlarger{\alpha}}\Big|_{\mathcal{P}_{2'}} 
  }
\]
dopo aver determinato per ciascun pannello il gradiente del coefficiente 
di portanza.
Per le caratteristiche geometriche dei pannelli $\mathcal{P}_{1}$ e 
$\mathcal{P}_{2'}$ si possono stimare i due gradienti come segue:
\[
C_{L_\mathlarger{\alpha}}\Big|_{\mathcal{P}_{1}}
  =
  \frac{
    a_0 \cos \Lambda_{\mathrm{le,1}}
  }{
    \sqrt{
      1 - \Mach^2 \cos^2 \Lambda_{\mathrm{le,1}}
        + \left( \dfrac{ a_0 \cos \Lambda_{\mathrm{le,1}} }{ \pi \AR_1} \right)^2
    }
    +
    \dfrac{ a_0 \cos \Lambda_{\mathrm{le,1}} }{ \pi \AR_1}
  }
\]
\[
C_{L_\mathlarger{\alpha}}\Big|_{\mathcal{P}_{2'}}
  =
  \frac{
    a_0 \cos \Lambda_{\mathrm{le,2}}
  }{
    \sqrt{
      1 - \Mach^2 \cos^2 \Lambda_{\mathrm{le,2}}
        + \left( \dfrac{ a_0 \cos \Lambda_{\mathrm{le,2}} }{ \pi \AR_{2'}} \right)^2
    }
    +
    \dfrac{ a_0 \cos \Lambda_{\mathrm{le,2}} }{ \pi \AR_{2'}}
  }
\]
con
\[
a_0 = 
  \frac{
    \SI[round-precision=3]{\myCLAlphaAtMACWingRAD}{\radian^{-1}}
  }{
    \sqrt{ 1 - \Mach^2 \cos^2 \Lambda_{\mathrm{le,1}} }
  }
  \qquad (\, \Lambda_{\mathrm{le,1}} = \Lambda_{\mathrm{le,2}} \,)
\]
Dai dati del problema è facile verificare che si ha
\[
C_{L_\mathlarger{\alpha}}\Big|_{\mathcal{P}_{1}}
  = \SI[round-precision=3]{\myCLAlphaPolhamusWingIRAD}{\radian^{-1}}
  = \SI[round-precision=4]{\myCLAlphaPolhamusWingIDEG}{\deg^{-1}}
\,,
\qquad
C_{L_\mathlarger{\alpha}}\Big|_{\mathcal{P}_{2'}}
  = \SI[round-precision=3]{\myCLAlphaPolhamusWingIIRAD}{\radian^{-1}}
  = \SI[round-precision=4]{\myCLAlphaPolhamusWingIIDEG}{\deg^{-1}}
\]
Pertanto, si ricava
\[
\frac{X_{\mathrm{ac}}}{c_\mathrm{r}}
  = \frac{
    \SI[round-precision=3]{\myXACOverChordRootDatcomWingI}{}
    \cdot \SI[round-precision=3]{\myCLAlphaPolhamusWingIRAD}{\radian^{-1}}
    \cdot \SI[round-precision=1]{\myAreaWingIMTsquared}{\metre^{2}}
    +
    \SI[round-precision=3]{\myXACOverChordRootDatcomWingII}{}
    \cdot \SI[round-precision=3]{\myCLAlphaPolhamusWingIIRAD}{\radian^{-1}}
    \cdot \SI[round-precision=1]{\myAreaWingIIPrimeMTsquared}{\metre^{2}}
  }{
    \SI[round-precision=3]{\myCLAlphaPolhamusWingIRAD}{\radian^{-1}}
    \cdot \SI[round-precision=1]{\myAreaWingIMTsquared}{\metre^{2}}
    +
    \cdot \SI[round-precision=3]{\myCLAlphaPolhamusWingIIRAD}{\radian^{-1}}
    \cdot \SI[round-precision=1]{\myAreaWingIIPrimeMTsquared}{\metre^{2}}  
  }
  = 
  \mathunderline{mydarkblue}{ \SI[round-precision=3]{\myXACOverChordRootDatcomWing}{} }
\]

Per un'ala \emph{cranked} questo è il risultato determinante, che permette di
ottenere
\[
X_{\mathrm{ac}} = \frac{X_{\mathrm{ac}}}{c_\mathrm{r}} \, c_{\mathrm{r},1}
  = \SI[round-precision=3]{\myXACOverChordRootDatcomWing}{}
    \cdot \SI[round-precision=2]{\myChordRootWingIMT}{\metre}
  = \SI[round-precision=2]{\myACWingToApexWingMT}{\metre} 
\]
cioè
\[
\frac{x_{\mathrm{ac}}}{\bar{c}}
  = 
  \frac{
    X_{\mathrm{ac}} - X_{\mathrm{le},\bar{c}}
  }{
    \bar{c}
  }
  =
  \frac{
    \SI[round-precision=2]{\myACWingToApexWingMT}{\metre}
      - \SI[round-precision=2]{\myXXMACLEToApexWingCrankedMT}{\metre}
  }{
    \SI[round-precision=2]{\myMACWingCrankedMT}{\metre}
  }
  =
  \mathunderline{mydarkblue}{ \SI[round-precision=3]{\myXsiACWing}{} }  
\]

Il centro aerodinamico di coordinate $(X_\mathrm{ac},0)$
è rappresentato nella figura~\ref{fig:Wing:Aerodynamic:Center:Cranked:Results}.

%-----------------------------------------------------------------------------------------------
% \begin{figure}[t]%[H]%[!htbp]
%   %\centering
%   %\checkoddpage
%   %\centering
%     \includegraphics[width=0.78\textwidth]{exercises/wing_ac_cranked_1/wing_ac_cranked_1_drawing_panels_ac.pdf}%
%   \caption{\finalhyphendemerits=1000
%           Centri aerodinamici dei pannelli 
%           $\mathcal{P}_1=ABB'A'$ e $\mathcal{P}_{2'}=DCC'D'$.
%   }
%   \label{fig:Wing:Aerodynamic:Center:Cranked:Panels:AC}%
% \end{figure}
%-----------------------------------------------------------------------------------------------
% \begin{figure}[t]%[H]%[!htbp]
%   %\centering
%   %\checkoddpage
%   %\centering
%     \includegraphics[width=0.78\textwidth]{exercises/wing_ac_cranked_1/wing_ac_cranked_1_drawing.pdf}%
%   \caption{\finalhyphendemerits=1000
%           Centro aerodinamico dell'ala proposta nell'esempio~\ref{example:Wing:Aerodynamic:Center:Cranked}. L'asse tratteggiato,
%           parallelo all'asse $Y$ e passante per il centro aerodinamico alare
%           è l'asse di beccheggio intorno al quale il coefficiente di momento
%           dell'ala è costante al variare dell'angolo d'attacco della corrente:
%           \smash{$\partial C_{\mathcal{M}_\mathlarger{\mathrm{ac}}}/\partial\alpha = 0$}.
%   }
%   \label{fig:Wing:Aerodynamic:Center:Cranked:Results}%
% \end{figure}

\end{myExampleX}




\end{document}