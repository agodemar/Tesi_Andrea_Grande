\documentclass[[12pt,twoside]{book}
\usepackage{_my_document_style}
\begin{document}
%
%\setcounter{chapter}{3}
%
\chapter%
   [Wing aerodynamics calculations]%
   {Wing aerodynamics calculations}
\label{chap:Wing}

\setcounter{minitocdepth}{2}% 1: chapter level; 2: section level
\minitoc %\mtcskip \minilof \minilot

\vspace{\baselineskip}

\noindent
In the following chapter, the goal is to deal with the main calculation related to the aircraft wing
specifically. Essential aspects are dealt with both from a geometric point of view and from a performance point of view such as the different configurations of wings (cranked, straight, equivalent etc.), and specifically it has been deeply treated 
how to obtain the zero lift angle, pitching moment around the wing aerodynamic center, the lift gradient, etc. through
analytical and numerical methods.
%
%\subimport{}{ PUT FILE NAME HERE }
\subimport{}{geometric_characteristics_of_a_straight_wing}
\subimport{}{geometric_characteristics_of_a_straight_and_tapered_wing}
\subimport{}{geometric_characteristics_of_a_wing_with_arrow_not_null}
\subimport{}{geometric_characteristics_of_a_cranked_wing}
\subimport{}{equivalent_plan_form_of_a_cranked_wing}
\subimport{}{zero_lift_angle_of_a_finite_wing}
\subimport{}{zero_lift_angle_of_a_cranked_wing}
\subimport{}{zero_lift_angle_graphic_method}
\subimport{}{lift_gradient_of_a_finite_wing}
\subimport{}{lift_gradient_aspect_ratio_effect}
\subimport{}{lift_gradient_polhamus_formula}
\subimport{}{aerodynamic_center_of_a_finite_wing}
\subimport{}{aerodynamic_center_of_a_cranked_wing}
\subimport{}{pitching_moment}
\subimport{}{pitching_moment_two}


\end{document}