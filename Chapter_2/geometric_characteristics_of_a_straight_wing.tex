\documentclass[[12pt,twoside]{book}
\usepackage{_my_document_style}

\begin{document}

%
\def\mySpanWingMT{26.000000}
\def\myChordRootWingMT{2.500000}
\def\myChordTipWingMT{2.500000}
\def\myTaperRatioWing{1.000000}
\def\myAreaWingMTsquared{65.000000}
\def\myMACWingMT{2.500000}
\def\myYMACWingMT{6.500000}
\def\myXLEMACWingMT{0.000000}
\def\myLambdaCQuarter{0.000000}
\def\myLambdaLEDeg{0.000000}
\def\myLambdaLERad{0.000000}
\def\myAspectRatioWing{10.400000}

%


\begin{myExampleX}{Geometric characteristics of a straight wing}{\ding{46}}% \ \Keyboard\ %
\label{example:Geometric:Characteristics:Of:A:Straight:Wing}
%
\noindent
Let's consider a wing with a finite wingspan equal to $b=\SI[round-precision=1]{\mySpanWingMT}{\metre}$,
with straight inlet and outlet edges and
null arrow angle, that is $\Lambda_{c/4}=\SI[round-precision=0]{\myLambdaCQuarter}{\degree}$.
In addition, the wing sections at the various stations $Y$ along the opening have constant chord
equal to $\SI[round-precision=2]{\myChordRootWingMT}{\metre}$
and they all have the same profile, ie the aerodynamic characteristics of the section are constant.
We want to calculate the following quantities:

\noindent
\adjustbox{center=\textwidth}{%
$S$\,, $\AR$\,, $\bar{c}$\,, $X_{\mathrm{le},\bar{c}}$\,, $Y_{\bar{c}}$
}% \,, $Y_{\bar{c}}$

\medskip
\noindent
Not being tapered, the wing has a root chord
$c_\mathrm{r}=\SI[round-precision=2]{\myChordRootWingMT}{\metre}$,
a tip chord $c_\mathrm{t}=\SI[round-precision=2]{\myChordTipWingMT}{\metre}$
and a taper ratio $\lambda=\SI[round-precision=0]{\myTaperRatioWing}{}$.
The leading edge arrow is
 $\Lambda_\mathrm{le}=\SI[round-precision=0]{\myLambdaCQuarter}{\degree}$.
\noindent
The wing surface is calculated as follows:
\[
\begin{split}
S & {}= \frac{b}{2} \, c_\mathrm{r} \, \big( 1 + \lambda \big) = b\,c_\mathrm{r}\\
  & {}=
    \num{0.5} \cdot \SI[round-precision=1]{\mySpanWingMT}{\metre}
      \cdot \SI[round-precision=2]{\myChordRootWingMT}{\metre}
      \cdot \big( 1 + \SI[round-precision=0]{\myTaperRatioWing}{} \big) 
    = { \SI[round-precision=1]{\myAreaWingMTsquared}{\metre^2} }
\end{split}
\]
In this case it is simply the area of a rectangle.

\noindent
The elongation is therefore
\[
\AR 
  = \frac{b^2}{S}
  = \frac{\big(\SI[round-precision=1]{\mySpanWingMT}{\metre}\big)^2}{\SI[round-precision=1]{\myAreaWingMTsquared}{\metre^2}}
  ={ \num[round-precision=2]{\myAspectRatioWing} }
\]
This value is greater than $\num[round-precision=0]{10}$ allows us to say that we are in
presence of a wing of high aspect ratio.


\noindent
The mean aerodynamic chord is given by
 
\[
\begin{split}
\bar{c} & {}= \frac{2}{3} \, c_\mathrm{r} \, \frac{1+\lambda + \lambda^2}{1+\lambda} \\
  & {}=
    \num{0.667} \cdot \SI[round-precision=2]{\myChordRootWingMT}{\metre}
      \cdot 
        \frac{
          1 + \SI[round-precision=0]{\myTaperRatioWing}{} + \SI[round-precision=0]{\myTaperRatioWing}{}^2
        }{
          1 + \SI[round-precision=0]{\myTaperRatioWing}{}
        }
    = \SI[round-precision=2]{\myMACWingMT}{\metre} 
\end{split}
\]
Note that for wings of this type the value of $\bar{c} $ is nothing other than that of the chord
of any wing section.

\noindent
The longitudinal distance of the leading edge of the mean aerodynamic chord from the
leading edge of the root chord is given by
\[
\begin{split}
X_{\mathrm{le},\bar{c}} 
  & {}=
    \frac{b}{6} \, \frac{1+2\lambda}{1+\lambda} \tan\Lambda_\mathrm{le} \\[3pt]
  & {}=
    \frac{\SI[round-precision=1]{\mySpanWingMT}{\metre}}{6}
      \cdot 
      \frac{
        1 + 2\cdot\SI[round-precision=0]{\myTaperRatioWing}{}
      }{
        1 + \SI[round-precision=0]{\myTaperRatioWing}{}
      }
      \cdot \tan \big( \SI[round-precision=0]{\myLambdaLERad}{\radian} \big)
    = { \SI[round-precision=0]{\myXLEMACWingMT}{\metre} }% \myXLEMACWingMT
\end{split}
\]
The null value confirms that the aerodynamic average chord, projected on the center plane
of the wing, overlaps the root chord.

\noindent
For the assigned wing, all the stations $ Y $ along the opening have a chord equal to $ \bar{c} $.
Formally, the station $ Y _ {\bar{c}} $ corresponding to the mean aerodynamic chord
is calculated as follows:
\[
\begin{split}
Y_{\bar{c}} 
  & {}=
    \frac{b}{6} \, \frac{1+2\lambda}{1+\lambda} = \\[3pt]
  & {}=
    \frac{\SI[round-precision=1]{\mySpanWingMT}{\metre}}{6}
      \cdot 
      \frac{
        1 + 2\cdot\SI[round-precision=2]{\myTaperRatioWing}{}
      }{
        1 + \SI[round-precision=2]{\myTaperRatioWing}{}
      }
    = { \SI[round-precision=2]{\myYMACWingMT}{\metre} }
    = \frac{b}{2} \, \frac{1}{2}
\end{split}
\]
The value obtained corresponds to the station along the wingspan halfway between the
the root chord and the tpi chord.
We will see later that for this type of straight wing with constant section, when twisting
geometric $ \epsilon_\mathrm {g} (Y) $ is identically null,
the basic aerodynamic characteristics of the section profile are transferred to the finished wing.
For example, the zero lift angle $\alpha_ {0L} $ coincides with the angle $ \alpha_ {0 \ell} $
of zero lift of the profile; just as $C_L \big|_{\alpha = 0} $ coincides with the profile $ C _ {\ell 0} $.
The gradient \smash{$C_{L_\mathlarger{\alpha}}$} relative to the wing must instead be corrected with respect to
\smash{$C_{\ell_\mathlarger{\alpha}}$} value relative to the profile for the finite elongation effect.

\end{myExampleX}
\end{document}
