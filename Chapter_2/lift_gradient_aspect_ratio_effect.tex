\documentclass[[12pt,twoside]{book}
\usepackage{_my_document_style}
\begin{document}
%
\begin{myExampleX}{Lift gradient of a finite wing. Effect of aspect ratio}{\ding{46}}% \ \Keyboard\ %
\label{example:Lift:Gradient:Aspect:Ratio:Effect}
%
\noindent
We want to calculate the lift gradient \smash{$C_{L_\mathlarger{\alpha}}$}
of a wing similar to that of the example~\ref{example:Lift:Gradient:Of:A:Finite:Wing}
verifying the effect of the aspect ratio variation.

%-------------------------------------------------
\def\myInitialValueA{999}
\def\myISAAirGasConstNMTKGK{287}
\def\myISAAirAdiabaticIndex{1.4}
\def\myISAAirTemperatureSeaLevelK{288.16}
\def\myISAAirDensitySeaLevelKGMTcubed{1.225}
\def\myISAAirDensitySeaLevelSLUGFTcubed{0.0023769}
\def\myAltitudeMT{3000}
\def\myAltitudeFT{9842.52}
\def\myISALapseRateKMT{-0.0065}
\def\myAirDensityKGMTcubed{0.9091}
\def\myAirDensitySLUGFTcubed{0.0017639}
\def\myAirDensityRatio{0.7421}
\def\myAirDynamicViscosiyMTSecKG{1.694e-05}
\def\myAirTemperatureK{268.66}
\def\myAirSoundSpeedMTSec{328.55378}
\def\myAirSoundSpeedFTSec{1077.93236}
\def\myPrepareForStackNull{-1e+307}
\def\myInitialValueB{999}
\def\myMach{0.147}
\def\myFlightSpeedMTSec{48.368}
\def\myFlightSpeedFTSec{158.68767}
\def\myFlightSpeedKMH{174.1}
\def\myFlightSpeedKTS{94.02}
\def\myFlightSpeedEASMTSec{41.66667}
\def\myFlightSpeedEASFTSec{136.70166}
\def\myFlightSpeedEASKMH{150}
\def\myFlightSpeedEASKTS{80.99}
\def\myFlightQbarPA{1063.36806}
\def\myFlightQbarBAR{0.011}
\def\myInitialValueB{999}
\def\myUBodyMTSec{48.368}
\def\myUBodyFTSec{158.68767}
\def\myUBodyKMH{174.1}
\def\myUBodyKTS{94.02}
\def\myVBodyMTSec{0}
\def\myVBodyFTSec{0}
\def\myVBodyKMH{0}
\def\myVBodyKTS{0}
\def\myWBodyMTSec{0}
\def\myWBodyFTSec{0}
\def\myWBodyKMH{0}
\def\myWBodyKTS{0}

\let\mySpanWingAMT\mySpanWingMT
\let\myAreaWingAMTsquared\myAreaWingMTsquared
\let\myAspectRatioWingA\myAspectRatioWing
\let\myCLAlphaMeanWingARAD\myCLAlphaMeanWingRAD
\let\myCLAlphaMeanWingADEG\myCLAlphaMeanWingDEG
\let\myCLAlphaWingARAD\myCLAlphaWingRAD
\let\myCLAlphaWingADEG\myCLAlphaWingDEG
%-------------------------------------------------
For this wing we have:

\smallskip
\noindent
\adjustbox{center=\textwidth}{%
$b=\SI[round-precision=2]{\mySpanWingAMT}{\meter}$,
$\AR=\SI[round-precision=2]{\myAspectRatioWingA}{}$,
$S=\SI[round-precision=2]{\myAreaWingAMTsquared}{\metre^2}$,
\smash{$\bar{C}_{\ell_\mathlarger{\alpha}}=\SI[round-precision=3]{\myCLAlphaMeanWingRAD}{\radian^{-1}}$}
}

\smallskip
\noindent
\adjustbox{center=\textwidth}{%
$C_{L_\mathlarger{\alpha}}
  =\mathunderline{mydarkblue}{ \SI[round-precision=2]{\myCLAlphaWingARAD}{\radian^{-1}} }
  =\mathunderline{mydarkblue}{ \SI[round-precision=3]{\myCLAlphaWingADEG}{\deg^{-1}} }
$
}

\smallskip
%-------------------------------------------------
\input{Chapter_2/lift_gradient_aspect_ratio_effect/data_b.tex}
\let\mySpanWingBMT\mySpanWingMT
\let\myAreaWingBMTsquared\myAreaWingMTsquared
\let\myAspectRatioWingB\myAspectRatioWing
\let\myCLAlphaMeanWingBRAD\myCLAlphaMeanWingRAD
\let\myCLAlphaMeanWingBDEG\myCLAlphaMeanWingDEG
\let\myCLAlphaWingBRAD\myCLAlphaWingRAD
\let\myCLAlphaWingBDEG\myCLAlphaWingDEG
%-------------------------------------------------
For a wing of equal root chord and taper ratio, but of wingspan $b'$ decreased by $20\%$ compared to $b$ we have:

\smallskip
\noindent
\adjustbox{center=\textwidth}{%
$b'=\SI[round-precision=2]{\mySpanWingBMT}{\meter}$,
$\AR'=\SI[round-precision=2]{\myAspectRatioWingB}{}$,
$S'=\SI[round-precision=2]{\myAreaWingBMTsquared}{\metre^2}$,
\smash{$\bar{C}_{\ell_\mathlarger{\alpha}}'=\SI[round-precision=4]{\myCLAlphaMeanWingRAD}{\radian^{-1}}$}
}

\smallskip
\noindent
\adjustbox{center=\textwidth}{%
$C_{L_\mathlarger{\alpha}}'
  =\mathunderline{mydarkblue}{ \SI[round-precision=2]{\myCLAlphaWingBRAD}{\radian^{-1}} }
  =\mathunderline{mydarkblue}{ \SI[round-precision=3]{\myCLAlphaWingBDEG}{\deg^{-1}} }
$
}

\smallskip
%-------------------------------------------------
\input{Chapter_2/lift_gradient_aspect_ratio_effect/data_c.tex}
\let\mySpanWingCMT\mySpanWingMT
\let\myAreaWingCMTsquared\myAreaWingMTsquared
\let\myAspectRatioWingC\myAspectRatioWing
\let\myCLAlphaMeanWingCRAD\myCLAlphaMeanWingRAD
\let\myCLAlphaMeanWingCDEG\myCLAlphaMeanWingDEG
\let\myCLAlphaWingCRAD\myCLAlphaWingRAD
\let\myCLAlphaWingCDEG\myCLAlphaWingDEG
%-------------------------------------------------
Similarly, for a wingspan $b''$ increased by $20\%$ compared to $b$ we have:

\smallskip
\noindent
\adjustbox{center=\textwidth}{%
$b''=\SI[round-precision=2]{\mySpanWingCMT}{\meter}$,
$\AR''=\SI[round-precision=2]{\myAspectRatioWingC}{}$,
$S''=\SI[round-precision=2]{\myAreaWingCMTsquared}{\metre^2}$,
\smash{$\bar{C}_{\ell_\mathlarger{\alpha}}''=\SI[round-precision=4]{\myCLAlphaMeanWingRAD}{\radian^{-1}}$}
}

\smallskip
\noindent
\adjustbox{center=\textwidth}{%
$C_{L_\mathlarger{\alpha}}''
  =\mathunderline{mydarkblue}{ \SI[round-precision=2]{\myCLAlphaWingCRAD}{\radian^{-1}} }
  =\mathunderline{mydarkblue}{ \SI[round-precision=3]{\myCLAlphaWingCDEG}{\deg^{-1}} }
$
}

\medskip
Lift gradient values $C_{L_\mathlarger{\alpha}}'$ and $C_{L_\mathlarger{\alpha}}''$
are calculated by applying the procedure of the example~\ref{example:Lift:Gradient:Of:A:Finite:Wing}.
It is observed that for a gradually increasing aspect ratio
($\AR' < \AR < \AR''$) the gradient of the lifting line of the wing increases
($C_{L_\mathlarger{\alpha}}' < C_{L_\mathlarger{\alpha}} < C_{L_\mathlarger{\alpha}}''$).

See the figure~\ref{fig:CLAlpha:Wing:Planforms:B} for a graphic confirmation of previous results.
\end{myExampleX}
\begin{figure}
  [t]%[H]%[!htbp]
  %\centering
  %\checkoddpage
  %\centering
    \includegraphics[width=0.70\textwidth]{Chapter_2/lift_gradient_aspect_ratio_effect/wing_planforms_drawing.pdf}
  \caption{\finalhyphendemerits=1000
           Planform with different wingspan but the same root and tip chords.}
  \label{fig:CLAlpha:Wing:Planforms:B}%
\end{figure}
\end{document}