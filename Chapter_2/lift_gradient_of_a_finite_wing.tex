\documentclass[[12pt,twoside]{book}
\usepackage{_my_document_style}
\begin{document}
%
\begin{myExampleX}{Lift gradient of a finite wing}{\ding{46}}% \ \Keyboard\ %
\label{example:Lift:Gradient:Of:A:Finite:Wing}
%
\def\mySpanWingMT{26.800000}
\def\myMach{0.700000}
\def\myAspectRatioWing{7.108753}
\def\myChordRootWingMT{5.200000}
\def\myChordTipWingMT{2.340000}
\def\myTaperRatioWing{0.450000}
\def\myAreaWingMTsquared{101.036000}
\def\myCoeffAChordWing{-0.213433}
\def\myCoeffBChordWingMT{5.200000}
\def\myCLAlphaRootWingRAD{6.150000}
\def\myCLAlphaTipWingRAD{6.050000}
\def\myCoeffAClalphaWingRADMT{-0.007463}
\def\myCoeffAClalphaWingDEGMT{-0.000130}
\def\myCoeffBClalphaWingRAD{6.150000}
\def\myCLAlphaMeanWingRAD{6.106322}
\def\myCLAlphaMeanWingDEG{0.106575}
\def\myInducedDragFactorWing{0.900000}
\def\myCLAlphaWingRAD{4.683465}
\def\myCLAlphaWingDEG{0.081742}

%
\noindent
A finite wing with straight edges and zero sweep angle is assigned, i.e. with the sweep  angle of the focus line $\Lambda_{c/4}=\SI[round-precision=0]{\mySweepQuarterChordWingDEG}{\degree}$.
The remaining characteristics of the lifting surface are as follows:

\smallskip
\noindent
\adjustbox{left=\textwidth}{%
  \adjustbox{right=0.39\textwidth}{%
    \emph{chords}:
  }\rule{0.5em}{0pt}% --> SPACER
  \adjustbox{left=0.59\textwidth}{%
    $c_\mathrm{r}=\SI[round-precision=2]{\myChordRootWingMT}{\metre}$,
    $c_\mathrm{t}=\SI[round-precision=2]{\myChordTipWingMT}{\metre}$,
    $\lambda = \SI[round-precision=2]{\myTaperRatioWing}{}$,
  }%
}

\smallskip
\noindent
\adjustbox{left=\textwidth}{%
  \adjustbox{right=0.39\textwidth}{%
    \emph{wingspan and surface}:
  }\rule{0.5em}{0pt}% --> SPACER
  \adjustbox{left=0.59\textwidth}{%
    $b=\SI[round-precision=2]{\mySpanWingMT}{\metre}$,
    $S=\SI[round-precision=2]{\myAreaWingMTsquared}{\meter^2}$,
    $\AR = \SI[round-precision=2]{\myAspectRatioWing}{}$,
  }%
}

\smallskip
\noindent
\adjustbox{left=\textwidth}{%
  \adjustbox{right=0.39\textwidth}{%
    \emph{gradients $C_{\ell_\mathlarger{\alpha}}$ of profile}:
  }\rule{0.5em}{0pt}% --> SPACER
  \adjustbox{left=0.59\textwidth}{%
    $C_{\ell_\mathlarger{\alpha},\mathrm{r}}=\SI[round-precision=2]{\myCLAlphaRootWingRAD}{\radian^{-1}}$,
    $C_{\ell_\mathlarger{\alpha},\mathrm{t}}=\SI[round-precision=2]{\myCLAlphaTipWingRAD}{\radian^{-1}}$,
  }%
}

\smallskip
For the assigned wing we want to calculate the gradient \smash{$C_{L_\mathlarger{\alpha}}$}.

\medskip
If we assume the following linear laws
\[
c \big( Y \big) = A_c \, Y + B_c 
\qquad
C_{\ell_\mathlarger{\alpha}} \big( Y \big) = A_{C_{\ell_\alpha}} \, Y + B_{C_{\ell_\alpha}}
\]
we have, regarding the law of the section chords
\[
A_c
  = \frac{c_\mathrm{t} - c_\mathrm{r}}{b/2}
  = 
    2 \frac{
      \SI[round-precision=2]{\myChordTipWingMT}{\metre} - \SI[round-precision=2]{\myChordRootWingMT}{\metre}
    }{
      \SI[round-precision=2]{\mySpanWingMT}{\metre}
    }
  = \mathunderline{mydarkblue}{ \SI[round-precision=3]{\myCoeffAChordWing}{} }
\]
\[
B_c
  = c_\mathrm{r}
  = \mathunderline{mydarkblue}{ \SI[round-precision=2]{\myCoeffBChordWingMT}{\metre} }
\]
therefore
\[
c \big( Y \big) = A_c \, Y + B_c
  = \SI[round-precision=3]{\myCoeffAChordWing}{} \, Y
    + \SI[round-precision=2]{\myCoeffBChordWingMT}{\metre}
\]
Similarly, for the law of gradients of the section lift coefficient
\[
\begin{split}
A_{C_{\ell_\alpha}}
  & {}= \frac{C_{\ell_\mathlarger{\alpha},\mathrm{t}} - C_{\ell_\mathlarger{\alpha},\mathrm{r}}}{b/2}
  = 
    2 \frac{
      \SI[round-precision=2]{\myCLAlphaTipWingRAD}{\radian^{-1}} 
      - \SI[round-precision=2]{\myCLAlphaRootWingRAD}{\radian^{-1}}
    }{
      \SI[round-precision=2]{\mySpanWingMT}{\metre}
    }
\\
  & {}= 
    \mathunderline{mydarkblue}{ 
      \SI[round-precision=5]{\myCoeffAClalphaWingRADMT}{\big(\radian\,\metre\big)^{-1}} 
    }
  = 
    \mathunderline{mydarkblue}{ 
      \SI[round-precision=7]{\myCoeffAClalphaWingDEGMT}{\big(\deg\,\metre\big)^{-1}} 
    }
\end{split}
\]
\[
B_{C_{\ell_\alpha}}
  = C_{\ell_\mathlarger{\alpha},\mathrm{r}}
  = \mathunderline{mydarkblue}{ \SI[round-precision=3]{\myCoeffBClalphaWingRAD}{\radian^{-1}} }
\]
i.e.
\[
C_{\ell_\mathlarger{\alpha}} \big( Y \big) = A_{C_{\ell_\alpha}} \, Y + B_{C_{\ell_\alpha}}
  = \SI[round-precision=5]{\myCoeffAClalphaWingRADMT}{\big(\radian\,\metre\big)^{-1}} \, Y
    + \SI[round-precision=3]{\myCoeffBClalphaWingRAD}{\radian^{-1}}
\]

The laws obtained above allow to calculate the average gradient
\begin{multline}
\nonumber
\bar{C}_{\ell_\mathlarger{\alpha}}
  = \frac{2}{S} \int_0^{b/2} 
      c\big(Y\big) \, C_{\ell_\mathlarger{\alpha}} \big(Y\big) \diff{Y}
\\
  = \frac{2}{ \SI[round-precision=2]{\myAreaWingMTsquared}{\metre^{2}} } 
    \int_0^{ \calcSI[round-precision=1,fixed-exponent=0,scientific-notation=fixed]{0.5*\mySpanWingMT}{\metre} } 
    \big[
      \SI[round-precision=5]{\myCoeffAClalphaWingRADMT}{\big(\radian\,\metre\big)^{-1}} \, Y
      + \SI[round-precision=3]{\myCoeffBClalphaWingRAD}{\radian^{-1}}
    \big]
    \rule{5em}{0pt}% <-- SPACER
\\
    \rule{5em}{0pt}% <-- SPACER
    \cdot \big[
      \SI[round-precision=3]{\myCoeffAChordWing}{} \, Y
      + \SI[round-precision=2]{\myCoeffBChordWingMT}{\metre}
    \big]
    \diff{Y}
\\
  = \mathunderline{mydarkblue}{ \SI[round-precision=3]{\myCLAlphaMeanWingRAD}{\radian^{-1}} }
  = \mathunderline{mydarkblue}{ \SI[round-precision=4]{\myCLAlphaMeanWingDEG}{\deg^{-1}} }
\end{multline}
It can be used for the calculation of the lift gradient of the finite wing
\[
C_{L_\mathlarger{\alpha}}
  = 
    \frac{
      \bar{C}_{\ell_\mathlarger{\alpha}}
    }{
      1 + \dfrac{\bar{C}_{\ell_\mathlarger{\alpha}}}{\pi \AR \, e_\Wing}
    }
  =
    \frac{
      \SI[round-precision=3]{\myCLAlphaMeanWingRAD}{\radian^{-1}}
    }{
      1 + 
        \dfrac{ \SI[round-precision=3]{\myCLAlphaMeanWingRAD}{\radian^{-1}} }{
          \num[round-precision=2]{3.14} 
          \cdot \SI[round-precision=2]{\myAspectRatioWing}{}
          \cdot \SI[round-precision=2]{\myInducedDragFactorWing}{}
        }
    }
  = \mathunderline{mydarkblue}{ \SI[round-precision=3]{\myCLAlphaWingRAD}{\radian^{-1}} }
  = \mathunderline{mydarkblue}{ \SI[round-precision=4]{\myCLAlphaWingDEG}{\deg^{-1}} }
\]
\end{myExampleX}
\end{document}