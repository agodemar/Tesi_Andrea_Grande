\documentclass[[12pt,twoside]{book}
\usepackage{_my_document_style}
\begin{document}
%
\def\mySpanWingMT{26.800000}
\def\myMach{0.700000}
\def\myAspectRatioWing{7.262873}
\def\myChordRootWingMT{5.200000}
\def\myChordTipWingMT{2.180000}
\def\myTaperRatioWing{0.419231}
\def\myAreaWingMTsquared{98.892000}
\def\myCriticalMachNumberMACWing{0.790000}
\def\mySweepLEWingDEG{27.500000}
\def\mySweepLEWingRAD{0.479966}
\def\mySweepHalfChordWingRAD{0.387281}
\def\mySweepHalfChordWingDEG{22.189589}
\def\myKPolhamus{1.060316}
\def\myCLAlphaWingRAD{5.720110}
\def\myCLAlphaWingDEG{327.738174}

%
\begin{myExampleX}{Lift gradient of a finite wing, Polhamus formula}{\ding{46}}% \ \Keyboard\ %
\label{example:Lift:Gradient:Polhamus:Formula}
%
\noindent
We want to calculate the gradient \smash{$C_{L_\mathlarger{\alpha}}$} of a finite wing
with sweep angle using a formula known as \emph{Polhamus formula }:
\begin{equation}\label{eq:Aero:Polhamus:Formula}
C_{L_{\mathlarger{\alpha}}} = 
   \frac{ 2\pi\AR
   }{
      2 + \sqrt{ 4+ \frac{\AR^2 \big(1-M^2\big)}{k_\mathrm{P}^2}\left({1+\frac{\tan^2\Lambda_{c/2}}{1-M^2}} \right) }
   }
\end{equation}
with
\begin{equation}\label{eq:Aero:K:Polhamus:Formula}
k_\mathrm{P}={
   \begin{cases}      1+\AR\dfrac{\num[round-precision=2]{1.87}-\num[round-precision=6]{0.000233}\Lambda_\mathrm{le}}{\num[round-precision=0]{100}}
      & \text{if}\, \AR<4
      \\[1em]
      1+\dfrac{
         \big(\num[round-precision=1]{8.2}-\num[round-precision=1]{2.3}\Lambda_\mathrm{le}\big)
         - \AR\big(\num[round-precision=2]{0.22}-\num[round-precision=3]{0.153}\Lambda_\mathrm{le}\big)
         }{\num[round-precision=0]{100}}
      & \text{if}\, \AR \ge 4
   \end{cases}
}
\end{equation}
and $\Lambda_\text{le}$ in radians.
the (\ref{eq:Aero:Polhamus:Formula}) it is to be considered applicable for finite wings
having
\begin{equation}\label{eq:Aero:Limitations:Polhamus:Formula}
\Lambda_\mathrm{le} < \SI[round-precision=0]{32}{\degree}
\,,\quad
\SI[round-precision=1]{0.4}{} < \lambda \le 1
\,,\quad
3 \le \AR \le 8
\end{equation}
and for a flight Mach number $\Mach < \Mach_\mathrm{cr}$.

Considering a wing with straight leading and trailing edges,
%simile a quella dell'esempio~\ref{example:Wing:Basic:A},
with wingspan $b=\SI[round-precision=1]{\mySpanWingMT}{\metre}$,
root chord $c_\mathrm{r}=\SI[round-precision=2]{\myChordRootWingMT}{\metre}$,
tip chord $c_\mathrm{t}=\SI[round-precision=2]{\myChordTipWingMT}{\metre}$
and leading edge sweep angle
$\Lambda_\mathrm{le}=\SI[round-precision=1]{\mySweepLEWingDEG}{\deg}$.
Assume a flight Mach number $\Mach=\SI[round-precision=1]{\myMach}{}$
and a $\Mach_\mathrm{cr}\simeq\SI[round-precision=2]{\myCriticalMachNumberMACWing}{}$.

The assigned planform has
a taper ratio
$\lambda=\SI[round-precision=2]{\myTaperRatioWing}{}$,
a wing surface
$S=\SI[round-precision=1]{\myAreaWingMTsquared}{\metre^2}$,
an aspect ratio
$\AR=\num[round-precision=2]{\myAspectRatioWing}$
and a sweep angle
$\Lambda_{c/2}=\SI[round-precision=3]{\mySweepHalfChordWingRAD}{\radian}=\SI[round-precision=1]{\mySweepHalfChordWingDEG}{\deg}$.

The data of the wing are such as to satisfy the conditions (\ref{eq:Aero:Limitations:Polhamus:Formula})
and, replaced in (\ref{eq:Aero:Polhamus:Formula})-(\ref{eq:Aero:K:Polhamus:Formula}), 
allow to derive
\[
k_\mathrm{P} 
  =
    1+\dfrac{
      \big(
        \num[round-precision=1]{8.2}
        -\num[round-precision=1]{2.3}
        \cdot \SI[round-precision=3]{\mySweepLEWingRAD}{\radian}
      \big)
      - \num[round-precision=2]{\myAspectRatioWing}\,
      \big(
        \num[round-precision=2]{0.22}-\num[round-precision=3]{0.153}
        \cdot \SI[round-precision=3]{\mySweepLEWingRAD}{\radian}
      \big)
    }{\num[round-precision=0]{100}}
  = \mathunderline{mydarkblue}{ \SI[round-precision=2]{\myKPolhamus}{} }
\]
and
\[
\begin{split}
C_{L_{\mathlarger{\alpha}}} 
  & {}= 
   \frac{ 2\pi\AR
   }{
      2 + \sqrt{ 4+ \frac{\AR^2 \big(1-M^2\big)}{k_\mathrm{P}^2}\left({1+\frac{\tan^2\Lambda_{c/2}}{1-M^2}}\right)}
   }
\\[10pt]
  & {}= 
  \frac{ \num[round-precision=3]{6.28} \cdot \num[round-precision=2]{\myAspectRatioWing}
   }{
      2 + \sqrt{ 4+ 
      \frac{
        \num[round-precision=2]{\myAspectRatioWing}^2 \big( 1-\num[round-precision=2]{\myMach}^2 \big)
      }{
        \SI[round-precision=2]{\myKPolhamus}{}^2}
        \left({ 1 +
          \frac{
            \tan^2 \SI[round-precision=3]{\mySweepHalfChordWingRAD}{\radian}
          }{
            1-\num[round-precision=2]{\myMach}^2
          }
        }\right)
      }
   }
    =\mathunderline{mydarkblue}{ \SI[round-precision=2]{\myCLAlphaWingRAD}{\radian^{-1}} }
    =\mathunderline{mydarkblue}{ \SI[round-precision=3]{\myCLAlphaWingDEG}{\deg^{-1}} }
\end{split}
\]
\end{myExampleX}

\EnlargedFigureX% needs two latex passes
  {p}% #1: t, b, p
  {%
    \centering
    \begin{tabular}{@{}c@{}}
      \subfloat%\subfigure
          [%
            \smash{$C_{L_\mathlarger{\alpha}}$} as $\Mach$ changes, for different values of parameter $\AR$.
          ]%
          {\label{fig:Wing:CLAlpha:Polhamus:Formula:Plots:A}%
		      \includegraphics%
		        %[width=0.52\textwidth]%
		        % [height=5.5cm]%
		        [width=0.45\textwidth]%
		        {Chapter_2/lift_gradient_polhamus_formula/plot_wing_CLalpha_LamLE15_lam05.pdf}%
          }%
      \\
      \subfloat%\subfigure
          [%
            \smash{$C_{L_\mathlarger{\alpha}}$} as $\Mach$ changes, for different values of parameter $\Lambda_\text{le}$.
          ]%
          {\label{fig:Wing:CLAlpha:Polhamus:Formula:Plots:B}%
		      \includegraphics%
		        %[width=0.52\textwidth]%
		        % [height=5.5cm]%
		        [width=0.45\textwidth]%
		        {Chapter_2/lift_gradient_polhamus_formula/plot_wing_CLalpha_lam05_AR5.pdf}%
          }%
      \\
      \subfloat%\subfigure
          [%
            \smash{$C_{L_\mathlarger{\alpha}}$} as $\Mach$ changes, for different values of parameter $\lambda$.
          ]%
          {\label{fig:Wing:CLAlpha:Polhamus:Formula:Plots:C}%
		      \includegraphics%
		        %[width=0.52\textwidth]%
		        % [height=5.5cm]%
		        [width=0.45\textwidth]%
		        {Chapter_2/lift_gradient_polhamus_formula/plot_wing_CLalpha_LamLE15_AR5.pdf}%
          }%
    \end{tabular}
  }% #2: the image file included by \includegraphics
  {\finalhyphendemerits=1000
    Graphical representations of the Polhamus formula for the calculation of the  \smash{$C_{L_\mathlarger{\alpha}}$} gradient.%
  }% #3: the caption text
  {fig:Wing:CLAlpha:Polhamus:Formula:Plots}%% #4: the label
%
\end{document}