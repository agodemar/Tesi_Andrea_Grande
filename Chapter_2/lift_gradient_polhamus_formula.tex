\documentclass[[12pt,twoside]{book}
\usepackage{_my_document_style}
\begin{document}
%
\input{chapter_2/lif}
%
\begin{myExampleX}{Gradiente di portanza di un'ala finita, formula di Polhamus}{\ding{46}}% \ \Keyboard\ %
\label{example:Wing:CLAlpha:C}
%
\noindent
Si vuole calcolare il gradiente \smash{$C_{L_\mathlarger{\alpha}}$} di un'ala finita
a freccia con una formula nota con il nome di \emph{formula di Polhamus}:
\begin{equation}\label{eq:Aero:Polhamus:Formula}
C_{L_{\mathlarger{\alpha}}} = 
   \frac{ 2\pi\AR
   }{
      2 + \SQRT{ 4+ \frac{\AR^2 \big(1-M^2\big)}{k_\mathrm{P}^2}\LEFTRIGHT(){1+\frac{\tan^2\Lambda_{c/2}}{1-M^2}} }
   }
\end{equation}
con
\begin{equation}\label{eq:Aero:K:Polhamus:Formula}
k_\mathrm{P}=\LEFTRIGHT\lcbrace.{
   \begin{array}{ll}
      1+\AR\dfrac{\num[round-precision=2]{1.87}-\num[round-precision=6]{0.000233}\Lambda_\mathrm{le}}{\num[round-precision=0]{100}}
      & \text{se}\, \AR<4
      \\[1em]
      1+\dfrac{
         \big(\num[round-precision=1]{8.2}-\num[round-precision=1]{2.3}\Lambda_\mathrm{le}\big)
         - \AR\big(\num[round-precision=2]{0.22}-\num[round-precision=3]{0.153}\Lambda_\mathrm{le}\big)
         }{\num[round-precision=0]{100}}
      & \text{se}\, \AR \ge 4
   \end{array}
}
\end{equation}
e $\Lambda_\text{le}$ in radianti.
La (\ref{eq:Aero:Polhamus:Formula}) è da ritenersi applicabile per ali finite
aventi
\begin{equation}\label{eq:Aero:Limitations:Polhamus:Formula}
\Lambda_\mathrm{le} < \SI[round-precision=0]{32}{\degree}
\,,\quad
\SI[round-precision=1]{0.4}{} < \lambda \le 1
\,,\quad
3 \le \AR \le 8
\end{equation}
e per un numero di Mach di volo $\Mach < \Mach_\mathrm{cr}$.

Si consideri un'ala con bordi d'attacco e d'uscita rettilinei,
%simile a quella dell'esempio~\ref{example:Wing:Basic:A},
di apertura $b=\SI[round-precision=1]{\mySpanWingMT}{\metre}$,
corda di radice $c_\mathrm{r}=\SI[round-precision=2]{\myChordRootWingMT}{\metre}$,
corda d'estremità $c_\mathrm{t}=\SI[round-precision=2]{\myChordTipWingMT}{\metre}$
e freccia del bordo d'attacco
$\Lambda_\mathrm{le}=\SI[round-precision=1]{\mySweepLEWingDEG}{\deg}$.
Si assuma un numero di Mach di volo $\Mach=\SI[round-precision=1]{\myMach}{}$
e un $\Mach_\mathrm{cr}\simeq\SI[round-precision=2]{\myCriticalMachNumberMACWing}{}$.

Si lascia per esercizio al lettore il compito di verificare che
la forma in pianta assegnata ha
un rapporto di rastremazione
$\lambda=\SI[round-precision=2]{\myTaperRatioWing}{}$,
una superficie alare
$S=\SI[round-precision=1]{\myAreaWingMTsquared}{\metre^2}$,
un allungamento alare
$\AR=\num[round-precision=2]{\myAspectRatioWing}$
e un angolo
$\Lambda_{c/2}=\SI[round-precision=3]{\mySweepHalfChordWingRAD}{\radian}=\SI[round-precision=1]{\mySweepHalfChordWingDEG}{\deg}$.

I dati dell'ala sono tali da soddisfare le condizioni (\ref{eq:Aero:Limitations:Polhamus:Formula})
e, sostituiti nelle (\ref{eq:Aero:Polhamus:Formula})-(\ref{eq:Aero:K:Polhamus:Formula}), 
permettono di ricavare
\[
k_\mathrm{P} 
  =
    1+\dfrac{
      \big(
        \num[round-precision=1]{8.2}
        -\num[round-precision=1]{2.3}
        \cdot \SI[round-precision=3]{\mySweepLEWingRAD}{\radian}
      \big)
      - \num[round-precision=2]{\myAspectRatioWing}\,
      \big(
        \num[round-precision=2]{0.22}-\num[round-precision=3]{0.153}
        \cdot \SI[round-precision=3]{\mySweepLEWingRAD}{\radian}
      \big)
    }{\num[round-precision=0]{100}}
  = \mathunderline{mydarkblue}{ \SI[round-precision=2]{\myKPolhamus}{} }
\]
e
\[
\begin{split}
C_{L_{\mathlarger{\alpha}}} 
  & {}= 
   \frac{ 2\pi\AR
   }{
      2 + \SQRT{ 4+ \frac{\AR^2 \big(1-M^2\big)}{k_\mathrm{P}^2}\LEFTRIGHT(){1+\frac{\tan^2\Lambda_{c/2}}{1-M^2}} }
   }
\\[10pt]
  & {}= 
  \frac{ \num[round-precision=3]{6.28} \cdot \num[round-precision=2]{\myAspectRatioWing}
   }{
      2 + \SQRT{ 4+ 
      \frac{
        \num[round-precision=2]{\myAspectRatioWing}^2 \big( 1-\num[round-precision=2]{\myMach}^2 \big)
      }{
        \SI[round-precision=2]{\myKPolhamus}{}^2}
        \LEFTRIGHT(){ 1 +
          \frac{
            \tan^2 \SI[round-precision=3]{\mySweepHalfChordWingRAD}{\radian}
          }{
            1-\num[round-precision=2]{\myMach}^2
          }
        } 
      }
   }
    =\mathunderline{mydarkblue}{ \SI[round-precision=2]{\myCLAlphaWingRAD}{\radian^{-1}} }
    =\mathunderline{mydarkblue}{ \SI[round-precision=3]{\myCLAlphaWingDEG}{\deg^{-1}} }
\end{split}
\]

\end{myExampleX}
\end{document}