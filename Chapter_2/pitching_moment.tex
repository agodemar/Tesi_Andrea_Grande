\documentclass[[12pt,twoside]{book}
\usepackage{_my_document_style}
\begin{document}
%
\def\mySweepLEWingDEG{0.000000}
\def\mySweepLEWingRAD{0.000000}
\def\myChordRootWingMT{2.500000}
\def\myChordTipWingMT{1.000000}
\def\mySpanWingMT{16.000000}
\def\myTaperRatioWing{0.400000}
\def\myAreaWingMTsquared{28.000000}
\def\myAspectRatioWing{9.142857}
\def\myMach{0.400000}
\def\myCLAlphaRootWingRAD{6.150000}
\def\myCLAlphaRootWingDEG{0.107338}
\def\myCLAlphaTipWingRAD{6.050000}
\def\myCLAlphaTipWingDEG{0.105592}
\def\myAlphaZeroLiftRootWingDEG{-3.000000}
\def\myAlphaZeroLiftTipWingDEG{-1.500000}
\def\myAlphaZeroLiftRootWingRAD{-0.052360}
\def\myAlphaZeroLiftTipWingRAD{-0.026180}
\def\myXsiacRootWing{0.250000}
\def\myXsiacTipWing{0.250000}
\def\myCmZeroRootWing{-0.080000}
\def\myCmZeroTipWing{-0.100000}
\def\mySweepQuarterChordWingDEG{-3.297150}
\def\mySweepQuarterChordWingRAD{-0.057546}
\def\myMACWingMT{1.857143}
\def\myYMACWingMT{3.428571}
\def\myXMACLEToApexWingMT{0.000000}
\def\myKOneACDatcomWing{1.342000}
\def\myKTwoACDatcomWing{0.000000}
\def\myXACOverChordRootDatcomWing{0.176000}
\def\myXsiACWing{0.236192}
\def\myACWingToApexWingMT{0.438642}
\def\myACWingToMACLEWingMT{0.438642}
\def\myCoeffAChordWing{-0.187500}
\def\myCoeffBChordWingMT{2.500000}
\def\myCoeffAAeroTwistWingRADMT{0.003272}
\def\myCoeffACmZeroWingMT{-0.002500}
\def\myCoeffBCmZeroWing{-0.080000}
\def\myCmZeroAWing{-0.087308}
\def\myCmZeroBWing{-0.003465}
\def\myCmZeroWing{-0.090772}
\def\myCoeffAWing{-0.500000}
\def\myCoeffBWing{0.059375}
\def\myCoeffCWing{-0.000469}
\def\myCoeffDWing{-0.000088}
\def\myCoeffEWing{-0.043792}
\def\myCoeffFWing{0.027072}
\def\myCoeffGWing{-0.004997}
\def\myCoeffHWing{0.000241}
\def\myCoeffIWing{0.230009}
\def\myCoeffJWing{-0.084804}
\def\myCoeffKWing{0.005203}
\def\myCoeffLWing{-0.000010}
\def\myCLAlphaMeanWingRAD{6.107143}
\def\myCoeffAClalphaWingRADMT{-0.012500}
\def\myCoeffBClalphaWingRAD{6.150000}
\def\myTwistWingDEG{-2.500000}
\def\myTwistWingRAD{-0.043633}
\def\myCoeffATwistWingRADMT{-0.005454}
\def\myCoeffBAeroTwistWingRAD{-0.052360}
\def\myAlphaZeroLiftWingRAD{-0.022440}
\def\myAlphaZeroLiftWingDEG{-1.285714}

%

%
\begin{myExampleX}{Momento di beccheggio intorno al centro aerodinamico dell'ala}{\ding{46}}% \ \Keyboard\ %
\label{example:Wing:Cmac:A}
%
\noindent
Si consideri l'ala a bordi dritti rappresentata nella figura~\ref{fig:Wing:Cmac:Results:A}
per la quale è nullo l'angolo di freccia del bordo d'attacco, 
cioè $\Lambda_\mathrm{le}=\SI[round-precision=0]{\mySweepLEWingDEG}{\deg}$.
Le rimanenti caratteristiche della superficie portante sono le seguenti:

\smallskip
\noindent
\adjustbox{left=\textwidth}{%
  \adjustbox{right=0.39\textwidth}{%
    \emph{corde}:
  }\rule{0.5em}{0pt}% --> SPACER
  \adjustbox{left=0.59\textwidth}{%
    $c_\mathrm{r}=\SI[round-precision=2]{\myChordRootWingMT}{\metre}$,
    $c_\mathrm{t}=\SI[round-precision=2]{\myChordTipWingMT}{\metre}$,
    $\lambda = \SI[round-precision=2]{\myTaperRatioWing}{}$,
  }%
}

\smallskip
\noindent
\adjustbox{left=\textwidth}{%
  \adjustbox{right=0.39\textwidth}{%
    \emph{apertura e superficie}:
  }\rule{0.5em}{0pt}% --> SPACER
  \adjustbox{left=0.59\textwidth}{%
    $b=\SI[round-precision=2]{\mySpanWingMT}{\metre}$,
    $S=\SI[round-precision=2]{\myAreaWingMTsquared}{\meter^2}$,
    $\AR = \SI[round-precision=2]{\myAspectRatioWing}{}$,
  }%
}

\smallskip
\noindent
\adjustbox{left=\textwidth}{%
  \adjustbox{right=0.39\textwidth}{%
    \emph{gradienti $C_{\ell_\mathlarger{\alpha}}$ di profilo}:
  }\rule{0.5em}{0pt}% --> SPACER
  \adjustbox{left=0.59\textwidth,minipage=[t]{0.59\textwidth}}{%
    $C_{\ell_\mathlarger{\alpha},\mathrm{r}}=\SI[round-precision=2]{\myCLAlphaRootWingRAD}{\radian^{-1}}=\SI[round-precision=4]{\myCLAlphaRootWingDEG}{\deg^{-1}}$,
    \\
    $C_{\ell_\mathlarger{\alpha},\mathrm{t}}=\SI[round-precision=2]{\myCLAlphaTipWingRAD}{\radian^{-1}}=\SI[round-precision=4]{\myCLAlphaTipWingDEG}{\deg^{-1}}$,
  }%
}

\smallskip
\noindent
\adjustbox{left=\textwidth}{%
  \adjustbox{right=0.39\textwidth}{%
    \emph{angoli di portanza nulla}:
  }\rule{0.5em}{0pt}% --> SPACER
  \adjustbox{left=0.59\textwidth,minipage=[t]{0.59\textwidth}}{%
    $\alpha_{0\ell,\mathrm{r}}=\SI[round-precision=3]{\myAlphaZeroLiftRootWingRAD}{\radian}
      =\SI[round-precision=1]{\myAlphaZeroLiftRootWingDEG}{\deg}$,\\
    $\alpha_{0\ell,\mathrm{t}}=\SI[round-precision=3]{\myAlphaZeroLiftTipWingRAD}{\radian}
      =\SI[round-precision=1]{\myAlphaZeroLiftTipWingDEG}{\deg}$,
  }%
}

\smallskip
\noindent
\adjustbox{left=\textwidth}{%
  \adjustbox{right=0.39\textwidth}{%
    \emph{centri aerodinamici di profilo}:
  }\rule{0.5em}{0pt}% --> SPACER
  \adjustbox{left=0.59\textwidth}{%
    $\bar{x}_{\mathrm{ac,2D},\mathrm{r}}=\SI[round-precision=2]{\myXsiacRootWing}{}$,
    $\bar{x}_{\mathrm{ac,2D},\mathrm{t}}=\SI[round-precision=2]{\myXsiacTipWing}{}$,
  }%
}

\smallskip
\noindent
\adjustbox{left=\textwidth}{%
  \adjustbox{right=0.39\textwidth}{%
    \emph{coefficienti $C_{\mathcal{m}_\mathlarger{\mathrm{ac}}}$ di profilo}:
  }\rule{0.5em}{0pt}% --> SPACER
  \adjustbox{left=0.59\textwidth}{%
    $C_{\mathcal{m}_\mathlarger{\mathrm{ac}},\mathrm{r}}=\SI[round-precision=3]{\myCmZeroRootWing}{}$,
    $C_{\mathcal{m}_\mathlarger{\mathrm{ac}},\mathrm{t}}=\SI[round-precision=3]{\myCmZeroTipWing}{}$,
  }%
}

\smallskip
\noindent
\adjustbox{left=\textwidth}{%
  \adjustbox{right=0.39\textwidth}{%
    \emph{condizione di volo}:
  }\rule{0.5em}{0pt}% --> SPACER
  \adjustbox{left=0.59\textwidth}{%
    $\Mach = \SI[round-precision=2]{\myMach}{}$.
  }%
}

\smallskip
Per l'ala assegnata si vuole calcolare il coefficiente \smash{$C_{\mathcal{M}_\mathlarger{\mathrm{ac}}}$}.

\medskip
L'angolo di freccia della linea dei quarti di corda --- linea che coincide in questo caso con la linea dei
fuochi delle sezioni alari --- si calcola come segue:
\[
\begin{split}
\tan
\Lambda_{c/4}
   = \tan\Lambda_\mathrm{le}-\dfrac{(4/4)(1-\lambda)}{\AR(1+\lambda)}
   & {}=
    \tan (\SI[round-precision=3]{\mySweepLEWingRAD}{\radian})
      - \dfrac{
         \num[round-precision=2]{1.0}
         \cdot (1-\SI[round-precision=2]{\myTaperRatioWing}{})
      }{
         \num[round-precision=2]{\myAspectRatioWing}
         \cdot (1+\SI[round-precision=2]{\myTaperRatioWing}{})} 
\\[3pt]
   & \Rightarrow
   \quad
   \Lambda_{c/4}
      = \mathunderline{mydarkblue}{ \SI[round-precision=4]{\mySweepQuarterChordWingRAD}{\radian} }
      = \mathunderline{mydarkblue}{ \SI[round-precision=1]{\mySweepQuarterChordWingDEG}{\deg} }
\end{split}
\]
L'ala assegnata ha dunque un angolo di freccia $\Lambda_{c/4}$ piccolo e negativo.

La corda media aerodinamica è in questo caso
\[
\bar{c} = \frac{2}{3} \, c_\mathrm{r} \, \frac{1+\lambda + \lambda^2}{1+\lambda}
  =
    \num{0.667} \cdot \SI[round-precision=2]{\myChordRootWingMT}{\metre}
      \cdot 
        \frac{
          1 + \SI[round-precision=2]{\myTaperRatioWing}{} + \SI[round-precision=2]{\myTaperRatioWing}{}^2
        }{
          1 + \SI[round-precision=2]{\myTaperRatioWing}{}
        }
    = \mathunderline{mydarkblue}{ \SI[round-precision=2]{\myMACWingMT}{\metre} }
\]

La distanza longitudinale del bordo d'attacco della corda media aerodinamica dal
bordo d'attacco della corda di radice è nulla, 
$X_{\mathrm{le},\bar{c}} = \SI[round-precision=2]{\myXMACLEToApexWingMT}{\metre}$, 
essendo nullo l'angolo $\Lambda_\mathrm{le}$.

La stazione $Y_{\bar{c}}$ alla quale la legge delle corde $c(Y)$ assume il valore $\bar{c}$ è
\[
Y_{\bar{c}} 
  =
    \frac{b}{6} \, \frac{1+2\lambda}{1+\lambda}
  =
    \frac{\SI[round-precision=1]{\mySpanWingMT}{\metre}}{6}
      \cdot 
      \frac{
        1 + 2\cdot\SI[round-precision=2]{\myTaperRatioWing}{}
      }{
        1 + \SI[round-precision=2]{\myTaperRatioWing}{}
      }
    = \mathunderline{mydarkblue}{ \SI[round-precision=2]{\myYMACWingMT}{\metre} }
\]


Per il numero di Mach di volo assegnato
il centro aerodinamico dell'ala si calcola come nell'esempio~\ref{example:Wing:Aerodynamic:Center:A}.
Dai dati e dalla figura~\ref{fig:Wing:Ac:K:One:Plots}
si legge
\[
\lambda=\SI[round-precision=2]{\myTaperRatioWing}{}
%
%\quad \Longrightarrow \quad
%\quad \begin{array}{@{}c@{}} \myIconGraph \\[-4pt] \Longrightarrow \end{array} \quad
\adjustbox{center=4em}{%
  \adjustbox{lap=\width}{\raisebox{2.2ex}[0pt][0pt]{\myIconGraph}}$\Longrightarrow$%
}
%
K_1
  = \mathunderline{mydarkblue}{ \SI[round-precision=3]{\myKOneACDatcomWing}{} }
\]
dalla figura~\ref{fig:Wing:Ac:K:Two:Plots}
si legge
\[
\Lambda_\mathrm{le} = \SI[round-precision=1]{\mySweepLEWingDEG}{\deg}\;,\;\,
\AR = \SI[round-precision=1]{\myAspectRatioWing}{}\;,\;\,
\lambda=\SI[round-precision=2]{\myTaperRatioWing}{}
%\quad \Longrightarrow \quad
%\quad \begin{array}{@{}c@{}} \myIconGraph \\[-4pt] \Longrightarrow \end{array} \quad
\adjustbox{center=4em}{%
  \adjustbox{lap=\width}{\raisebox{2.2ex}[0pt][0pt]{\myIconGraph}}$\Longrightarrow$%
}
%
K_2 
  = \mathunderline{mydarkblue}{ \SI[round-precision=3]{\myKTwoACDatcomWing}{} }
\]
dalla figura~\ref{fig:Wing:Ac:X:AC:From:Apex:Plots}
si legge
\[
\Lambda_\mathrm{le} = \SI[round-precision=1]{\mySweepLEWingDEG}{\deg}\;,\;\,
\Mach = \SI[round-precision=2]{\myMach}{}\;,\;\,
\AR = \SI[round-precision=1]{\myAspectRatioWing}{}\;,\;\,
\lambda=\SI[round-precision=2]{\myTaperRatioWing}{}
%
%\quad \Longrightarrow \quad
%\quad \begin{array}{@{}c@{}} \myIconGraph \\[-4pt] \Longrightarrow \end{array} \quad
\adjustbox{center=4em}{%
  \adjustbox{lap=\width}{\raisebox{2.2ex}[0pt][0pt]{\myIconGraph}}$\Longrightarrow$%
}
%
\frac{X'_{\mathrm{ac}}}{c_\mathrm{r}}
  = \mathunderline{mydarkblue}{ \SI[round-precision=3]{\myXACOverChordRootDatcomWing}{} }
\]
%
Si ottiene dunque
\[
\frac{x_{\mathrm{ac}}}{\bar{c}} 
  = K_1 \left( \frac{X'_{\mathrm{ac}}}{c_\mathrm{r}} - K_2  \right)
  = \SI[round-precision=3]{\myKOneACDatcomWing}{} \,
    \Big(  
      \SI[round-precision=3]{\myXACOverChordRootDatcomWing}{} 
        - \SI[round-precision=3]{\myKTwoACDatcomWing}{}  
    \Big)
  = \mathunderline{mydarkblue}{ \SI[round-precision=3]{\myXsiACWing}{} } 
\]
%
Pertanto, la posizione $X_\mathrm{ac}$ è calcolabile come

\[
X_\mathrm{ac} 
  =
    X_{\mathrm{le},\bar{c}} + \left( \frac{x_{\mathrm{ac}}}{\bar{c}} \right) \bar{c}
=
    \SI[round-precision=2]{\myXMACLEToApexWingMT}{\metre}
      + \SI[round-precision=3]{\myXsiACWing}{}
        \cdot \SI[round-precision=2]{\myMACWingMT}{\metre}
      = \SI[round-precision=2]{\myXMACLEToApexWingMT}{\metre}
        % + \calcSI[round-precision=2,fixed-exponent=0,scientific-notation=fixed]{\myXsiACWing*\myMACWingMT}{\metre}
        + \SI[round-precision=2]{\myACWingToMACLEWingMT}{\metre} 
    = \mathunderline{mydarkblue}{ 
      %\calcSI[round-precision=2,fixed-exponent=0,scientific-notation=fixed]{
      %  \myXMACLEToApexWingMT + \myXsiACWing*\myMACWingMT
      %}{\metre} 
      \SI[round-precision=2]{\myACWingToApexWingMT}{\metre} 
    }%
\]

La posizione del centro aerodinamico dell'ala individua una retta normale al piano di mezzeria
di equazione $X=X_\mathrm{ac}$ rispetto alla quale può essere calcolata la legge dei bracci
del carico basico
\[
x_\mathrm{b}(Y) 
  = X_\mathrm{ac} 
    - \Big[ Y \tan \Lambda_\mathrm{le} + \bar{x}_{\mathrm{ac,2D}} (Y) \, c(Y) \Big]
\]
La funzione $\bar{x}_{\mathrm{ac,2D}}(Y)$ esprime la distanza adimensionale del centro 
aerodinamico del profilo alla stazione $Y$ dal bordo d'attacco della corda locale. 
Per l'ala assegnata questa funzione coincide con la legge costante
$\bar{x}_{\mathrm{ac,2D}}(Y) = \frac{1}{4}$ e l'espressione precedente,
essendo $\Lambda_\mathrm{le}$ nullo, si particolarizza in
\[
x_\mathrm{b}(Y) = X_\mathrm{ac} - \frac{1}{4} \, c(Y)
\]

Il coefficiente di momento di beccheggio $C_{\mathcal{M}_\mathlarger{\mathrm{ac}}}$ dell'ala
intorno al suo centro aerodinamico è dato dalla formula
\[
C_{\mathcal{M}_\mathlarger{\mathrm{ac}}}
  = 
    \underbrace{% needs mt2pro, \underbrace otherwise
      %
      \frac{2}{S\,\bar{c}} \int_0^{b/2} C_{\mathcal{m}_\mathlarger{\mathrm{ac}},\mathrm{2D}}(Y) \; c^2(Y)\, \diff{Y}
      %
      %\rule[-1.2em]{0pt}{2em}% --> SPACER
    }_{ \mathlarger{ C_{\mathcal{M}_\mathlarger{\mathrm{ac,a}}} } } 
    +
    \underbrace{% needs mt2pro, \underbrace otherwise
      %
      \frac{2}{S\,\bar{c}} \int_0^{b/2} \big(cC_{\ell}\big)_\mathrm{b}(Y) \; x_\mathrm{b}(Y)\, \diff{Y}
      %
      %\rule[-1.2em]{0pt}{2em}% --> SPACER
    }_{ \mathlarger{ C_{\mathcal{M}_\mathlarger{\mathrm{ac,b}}} } } 
\]
%
dove 
\begin{compactenum}[{\color{gray}$\bullet$}]%[(\itshape i\upshape)]
\item
$C_{\mathcal{m}_\mathlarger{\mathrm{ac}},\mathrm{2D}}(Y)$ è la legge dei coefficienti di
momento dei profili intorno al centro aerodinamico di sezione,
\item
$\big(cC_{\ell}\big)_\mathrm{b}(Y) = c(Y) \, C_{\ell,\mathrm{b}} (Y)$ è il prodotto della corda locale
$c(Y)$ e del coefficiente di portanza \emph{basico} di sezione
$C_{\ell,\mathrm{b}}(Y)$ --- presente quando 
l'ala è posta all'angolo di portanza nulla $\alpha_{0L}$ (angolo misurato rispetto alla corda di radice),
\item
$x_\mathrm{b}(Y)$ è il braccio tra il punto di applicazione del carico basico locale e il centro
aerodinamico dell'ala.
\end{compactenum}

La formula precedente mostra che il coefficiente \smash{$C_{\mathcal{M}_\mathlarger{\mathrm{ac}}}$}
è la somma dei due contributi:
\begin{compactenum}[\itshape a\normalfont)]
\item
$C_{\mathcal{M}_\mathlarger{\mathrm{ac,a}}}$, ottenibile cumulando le coppie pure
  \[
    \diff{}\mathcal{M}_{\mathrm{ac},\mathrm{a}} 
      = C_{\mathcal{m}_\mathlarger{\mathrm{ac}},\mathrm{2D}}(Y) \; \bar{q}_\infty \,
        \underbrace{% needs mt2pro, \underbrace otherwise
          c(Y)\diff{Y}
        }_{ \mathlarger{ \diff{S} } }
          \, c(Y)
    = \bar{q}_\infty \, C_{\mathcal{m}_\mathlarger{\mathrm{ac}},\mathrm{2D}}(Y) \, c^2(Y) \, \diff{Y} 
  \]
\item
$C_{\mathcal{M}_\mathlarger{\mathrm{ac,b}}}$, ottenibile cumulando i momenti di trasporto
  \[
    \diff{}\mathcal{M}_{\mathrm{ac},\mathrm{b}} 
      = C_{\ell,\mathrm{b}}(Y)\,x_\mathrm{b}(Y)\; \bar{q}_\infty \, 
        \underbrace{% needs mt2pro, \underbrace otherwise
          c(Y)\diff{Y}
        }_{ \mathlarger{ \diff{S} } }
    = \bar{q}_\infty \, \big( cC_{\ell} \big)_\mathrm{b} (Y) \,x_\mathrm{b}(Y)\, \diff{Y}
  \]
\hfill(si noti che il braccio $x_b$ ha le dimensioni di una lunghezza)
\end{compactenum}

\noindent
Questi contributi vengono integrati lungo l'apertura alare e adimensionalizzati dividendoli
per $\bar{q}_\infty S \bar{c}$.

Per quanto riguarda il termine $C_{\mathcal{M}_\mathlarger{\mathrm{ac,a}}}$,
procedendo come mostrato nell'esempio~\ref{example:Wing:Alpha:Zero:Lift:A}
si ricava
\[
c \big( Y \big) = A_c \, Y + B_c
  = \SI[round-precision=3]{\myCoeffAChordWing}{} \, Y
    + \SI[round-precision=2]{\myCoeffBChordWingMT}{\metre}
\]
e, analogamente, si pone
\[
C_{\mathcal{m}_\mathlarger{\mathrm{ac}}} \big( Y \big) 
  = A_{C_{\mathcal{m}_\mathlarger{\mathrm{ac}}}} \, Y + B_{C_{\mathcal{m}_\mathlarger{\mathrm{ac}}}}
\]
con
\[
A_{C_{\mathcal{m}_\mathlarger{\mathrm{ac}}}}
  = \frac{C_{\mathcal{m}_\mathlarger{\mathrm{ac}},\mathrm{t}} 
    - C_{\mathcal{m}_\mathlarger{\mathrm{ac}},\mathrm{r}}}{b/2}
  = 
    2 \frac{
      \SI[round-precision=3]{\myCmZeroTipWing}{} 
      - ( \SI[round-precision=3]{\myCmZeroRootWing}{} )
    }{
      \SI[round-precision=2]{\mySpanWingMT}{\metre}
    }
  = 
    \mathunderline{mydarkblue}{ 
      \SI[round-precision=5]{\myCoeffACmZeroWingMT}{\metre^{-1}} 
    }
\]
\[
B_{C_{\mathcal{m}_\mathlarger{\mathrm{ac}}}}
  = C_{\mathcal{m}_\mathlarger{\mathrm{ac}},\mathrm{r}}
  = \mathunderline{mydarkblue}{ \SI[round-precision=3]{\myCoeffBCmZeroWing}{} }
\]
cioè
\[
C_{\mathcal{m}_\mathlarger{\mathrm{ac,2D}}} \big( Y \big) 
  = A_{C_{\mathcal{m}_\mathlarger{\mathrm{ac}}}} \, Y + B_{C_{\mathcal{m}_\mathlarger{\mathrm{ac}}}}
  = \SI[round-precision=5]{\myCoeffACmZeroWingMT}{\metre^{-1}} \, Y
    % +
    \SI[round-precision=3]{\myCoeffBCmZeroWing}{}
\]
%
Pertanto, la formula di calcolo del contributo dovuto alle coppie pure fornisce
\[
\begin{split}
C_{\mathcal{M}_\mathlarger{\mathrm{ac,a}}}
  & {}= \frac{2}{S\,\bar{c}} \int_0^{b/2} 
    C_{\mathcal{m}_\mathlarger{\mathrm{ac}},\mathrm{2D}}(Y) \; c^2(Y)\, \diff{Y}
\\
  & {}= 
  \frac{2}{S\,\bar{c}} \int_0^{b/2} 
    \Big( 
      A_{C_{\mathcal{m}_\mathlarger{\mathrm{ac}}}} \, Y + B_{C_{\mathcal{m}_\mathlarger{\mathrm{ac}}}} 
    \Big)
    \Big( A_c \, Y + B_c \Big)^{\! 2}
    \, \diff{Y}
\\
  & {}= 
  \frac{2}{S\,\bar{c}} \int_0^{b/2} 
    \Big( 
      \SI[round-mode=figures,round-precision=3,
        fixed-exponent=0,scientific-notation=fixed]{\myCoeffAWing}{\metre^2}
      + 
      \SI[round-mode=figures,round-precision=3,
        %fixed-exponent=2,
        %scientific-notation=fixed
        ]{\myCoeffBWing}{\metre}\, Y
      \SI[round-mode=figures,round-precision=3,
        fixed-exponent=-4,
        scientific-notation=fixed % engineering % 
        ]{\myCoeffCWing}{}\, Y^2
      \SI[round-mode=figures,round-precision=3,
        %fixed-exponent=-4,
        %scientific-notation=fixed
        ]{\myCoeffDWing}{\metre^{-1}}\, Y^3
    \Big)
    \, \diff{Y}
\\
  & {}= \mathunderline{mydarkblue}{ \SI[round-precision=4]{\myCmZeroAWing}{} }
\end{split}
\]
Si osservi che il valore dell'integrale definito è ottenibile sviluppando simbolicamente
la funzione integranda che è un polinomio di grado $3$ nella variabile $Y$.

\smallskip
Dall'Aerodinamica è noto che, per un generico angolo d'attacco $\alpha$ al quale l'ala produce un
coefficiente di portanza $C_L$,
il \emph{carico alare} $\gamma$ (\emph{wing span loading}) è esprimibile come
\[
\gamma(Y) \triangleq \frac{c(Y) \, C_\ell (Y)}{2 b} = C_L \, \gamma_{\mathrm{a}1}(Y) + \gamma_\mathrm{b}(Y) 
\]
Il carico è la somma del carico \emph{addizionale} $C_L \, \gamma_{\mathrm{a}1}(Y)$
--- un contributo che si annulla all'angolo di portanza nulla, in cui $C_L=0$ ---
e del carico \emph{basico} $\gamma_\mathrm{b}(Y)$.
Quest'ultimo, in particolare, è banalmente nullo per ali non svergolate (né geometricamente né aerodinamicamente)
e, in generale, ha una legge di variazione con la $Y$ il cui diagramma sottende un'area nulla:
\[
\int_0^{1} \gamma_\mathrm{b}(\eta) \diff{\eta} = 0 \qquad \text{con $\eta \triangleq \dfrac{Y}{b/2}$}
\]

Nella formula di calcolo di \smash{$C_{\mathcal{M}_\mathlarger{\mathrm{ac}},b}$}
compare $\big(cC_{\ell}\big)_\mathrm{b}(Y) = 2b\gamma_\mathrm{b}(Y)$.
Questa legge di carico può essere determinata per via numerica oppure in maniera approssimata 
dalla conoscenza delle leggi di svergolamento geometrico $\epsilon_\mathrm{g}(Y)$ e aerodinamico
$\alpha_{0\ell}(Y)$ e dell'angolo di portanza nulla dell'ala $\alpha_{0L}$.

L'espressione approssimata del carico basico, nota dalla Teoria vorticosa delle ali finite, è la seguente:
\[
\big(cC_{\ell}\big)_\mathrm{b}(Y)
  = \frac{1}{2} \, c(Y) \, C_{\ell_\mathlarger{\alpha}}(Y) \,
    \bigg\{ \alpha_{0L} - \Big[ \alpha_{0\ell}\big(Y\big) - \epsilon_\mathrm{g}\big(Y\big) \Big] \bigg\}
\]
nella quale il fattore $\frac{1}{2}$ tiene conto di effetti tridimensionali nella regione intorno
alla mezzeria dell'ala.
%
Nella formula precedente va trovata l'espressione della legge dei gradienti di portanza.
In prima approssimazione si può porre 
$C_{\ell_\mathlarger{\alpha}} (Y)=2\pi$; in alternativa si ricava la seguente legge lineare:
\smash{$C_{\ell_\mathlarger{\alpha}} \big( Y \big) = A_{C_{\ell_\alpha}} \, Y + B_{C_{\ell_\alpha}}$}.

Procedendo come mostrato nell'esempio~\ref{example:Wing:Alpha:Zero:Lift:A}
si ricava
%\[
%c \big( Y \big) = A_c \, Y + B_c
%  = \SI[round-precision=3]{\myCoeffAChordWing}{} \, Y
%    + \SI[round-precision=2]{\myCoeffBChordWingMT}{\metre}
%\]

\[
\alpha_{0\ell} \big( Y \big) = A_{\alpha} \, Y + B_{\alpha}
  = (\SI[round-precision=5]{\myCoeffAAeroTwistWingRADMT}{\radian/\metre}) \, Y
    \SI[round-precision=4]{\myAlphaZeroLiftRootWingRAD}{\radian}
\]

\[
\epsilon_\mathrm{g} \big( Y \big) = A_{\epsilon} \, Y + B_{\epsilon}
  = (\SI[round-precision=5]{\myCoeffATwistWingRADMT}{\radian/\metre}) \, Y
\]

\noindent
e
\[
C_{\ell_\mathlarger{\alpha}} \big( Y \big) = A_{C_{\ell_\alpha}} \, Y + B_{C_{\ell_\alpha}}
  = \SI[round-precision=5]{\myCoeffAClalphaWingRADMT}{\big(\radian\,\metre\big)^{-1}} \, Y
    + \SI[round-precision=3]{\myCoeffBClalphaWingRAD}{\radian^{-1}}
\]

Si ricava inoltre
\[
\begin{split}
\alpha_{0L} 
  & {}= \frac{2}{S} \int_0^{b/2} 
    \Big[ 
      \alpha_{0\ell}\big(Y\big) - \epsilon_\mathrm{g}\big(Y\big) 
    \Big] \, c(Y) \diff{Y}
\\[3pt]
  & {}= \frac{2}{S} \int_0^{b/2} 
    \bigg[ \Big( A_{\alpha} \, Y + B_{\alpha} \Big) - A_{\epsilon} \, Y \bigg] \Big( A_c Y + B_c \Big)
      \diff{Y}
  = \mathunderline{mydarkblue}{ \SI[round-precision=4]{\myAlphaZeroLiftWingRAD}{\radian} }
  = \mathunderline{mydarkblue}{ \SI[round-precision=1]{\myAlphaZeroLiftWingDEG}{\deg} }
\end{split}
\]

La formula di calcolo di \smash{$C_{\mathcal{M}_\mathlarger{\mathrm{ac}},b}$} diventa dunque

\[
\begin{split}
C_{\mathcal{M}_\mathlarger{\mathrm{ac,b}}} 
  & {}=
  \frac{2}{S\,\bar{c}} \int_0^{b/2} \big(cC_{\ell}\big)_\mathrm{b}(Y) \; x_\mathrm{b}(Y)\, \diff{Y}
\\[1pt]
  & {}=
  \frac{1}{S\,\bar{c}} \int_0^{b/2}
% \big(cC_{\ell}\big)_\mathrm{b}(Y)
%\cancel{2}\;
\overbrace{% needs mtp2pro
%\frac{1}{\cancel{2}}
c(Y) \, C_{\ell_\mathlarger{\alpha}}(Y) \,
    \bigg\{ \alpha_{0L} - \Big[ \alpha_{0\ell}\big(Y\big) - \epsilon_\mathrm{g}\big(Y\big) \Big] \bigg\}
}^{ \mathlarger{ \big(cC_{\ell}\big)_\mathrm{b}(Y) } }
\\[3pt]
  & \rule{4cm}{0pt}
%
% x_\mathrm{b}(Y) 
    \cdot
    \underbrace{% needs mtp2pro
    \bigg\{ 
      X_\mathrm{ac} 
        - \Big[ Y \tan \Lambda_\mathrm{le} + \bar{x}_{\mathrm{ac,2D}} (Y) \, c(Y) \Big]
    \bigg\} 
    }_{ \mathlarger{ x_\mathrm{b}(Y) } }
    \, \diff{Y}
\end{split}
\]
\noindent
Sostituendo i valori precedentemente trovati si ottiene
\[
 \begin{split}
C_{\mathcal{M}_\mathlarger{\mathrm{ac,b}}} 
  & {}=
  \frac{1}{S\,\bar{c}} \int_0^{b/2}
    %\overcbrace{% needs mtp2pro
    % c(Y) 
    \Big( A_c Y + B_c \Big)
    \, 
    \overcbrace^{ \mathlarger{ \big(cC_{\ell}\big)_\mathrm{b}(Y) } }
\\[2pt]
  & \hfill \rule{6cm}{0pt}
%
% x_\mathrm{b}(Y)
    \cdot
    %\undercbrace_{ \mathlarger{ x_\mathrm{b}(Y) } }
    \, \diff{Y}
\\[2pt]
  & {}= 
  \frac{2}{S\,\bar{c}} \int_0^{b/2} 
    \Big( 
      \SI[round-mode=figures,round-precision=3,
        %fixed-exponent=0,
        %scientific-notation=fixed
        ]{\myCoeffEWing}{\metre^2}
      + 
      \SI[round-mode=figures,round-precision=3,
        %fixed-exponent=2,
        %scientific-notation=fixed
        ]{\myCoeffFWing}{\metre}\, Y
      \SI[round-mode=figures,round-precision=3,
        %fixed-exponent=-4,
        %scientific-notation=fixed % engineering % 
        ]{\myCoeffGWing}{}\, Y^2
      +
      \SI[round-mode=figures,round-precision=3,
        %fixed-exponent=-4,
        %scientific-notation=fixed
        ]{\myCoeffHWing}{\metre^{-1}}\, Y^3
    \Big)
    \, \diff{Y}
\\[4pt]
  & {}= \mathunderline{mydarkblue}{ \SI[round-precision=5]{\myCmZeroBWing}{} }
\end{split}
\]
Si osservi che il valore dell'integrale definito è ottenibile sviluppando simbolicamente
la funzione integranda che è un polinomio di grado $3$ (in questo caso particolare) nella variabile $Y$.

Infine, il valore del coefficiente di momento di beccheggio dell'ala intorno al suo centro aerodinamico
è
\[
C_{\mathcal{M}_\mathlarger{\mathrm{ac}}} 
  = C_{\mathcal{M}_\mathlarger{\mathrm{ac,a}}} + C_{\mathcal{M}_\mathlarger{\mathrm{ac,b}}}
  = \SI[round-precision=4]{\myCmZeroAWing}{} 
    +
    ( \SI[round-precision=5]{\myCmZeroBWing}{} )
  = \mathunderline{mydarkblue}{ \SI[round-precision=4]{\myCmZeroWing}{} }
\]

Nella figura~\ref{fig:Wing:Cmac:Results:A} è rappresentata la forma in pianta dell'ala assegnata,
la legge dello svergolamento geometrico $\epsilon_\mathrm{g}(y)$ e dello svergolamento
aerodinamico $\alpha_{0\ell}(Y)$.
È rappresentata inoltre la legge del carico addizionale ottenuta con il \emph{metodo ingegneristico di Schrenk}
\[
\big( cC_\ell \big)_\mathrm{a1} = 2b \, \gamma_\mathrm{a1}
  = \frac{1}{2} \Big[ c_\mathrm{ell}(Y) + c_\mathrm{eff}(Y) \Big]
\]
dove
\[
c_\mathrm{ell}(Y) = \frac{4S}{\pi b} \sqrt{1 - \frac{2Y}{b}}
\qquad
c_\mathrm{eff}(Y) = \frac{ c(Y) \, C_{\ell_\mathlarger{\alpha}}(Y) }{ \bar{C}_{\ell_\mathlarger{\alpha}} }
\]
%
e 
\smash{$\bar{C}_{\ell_\mathlarger{\alpha}}=\SI[round-precision=3]{\myCLAlphaMeanWingRAD}{\radian^{-1}}$}.
%
Infine sono rappresentate la funzione $x_\mathrm{b}(Y)$ e la legge approssimata del carico basico
\[
\begin{split}
\big( cC_\ell \big)_\mathrm{b} (Y)
  & {}=
    0.5*\Big( A_c Y + B_c \Big)
      \Big( A_{C_{\ell_\alpha}} \, Y + B_{C_{\ell_\alpha}} \Big)
    \bigg\{ \alpha_{0L} 
      - \Big[ \Big( A_{\alpha} \, Y + B_{\alpha} \Big) - A_{\epsilon} \, Y \Big]
    \bigg\}
\\[3pt]
  & {}=
      \SI[round-mode=figures,round-precision=3,
        %fixed-exponent=0,
        %scientific-notation=fixed
        ]{\myCoeffIWing}{\metre}
      % + 
      \SI[round-mode=figures,round-precision=3,
        %fixed-exponent=2,
        %scientific-notation=fixed
        ]{\myCoeffJWing}{}\, Y
      +
      \SI[round-mode=figures,round-precision=3,
        %fixed-exponent=-4,
        %scientific-notation=fixed % engineering % 
        ]{\myCoeffKWing}{\metre^{-1}}\, Y^2
      % +
      \SI[round-mode=figures,round-precision=3,
        %fixed-exponent=-4,
        %scientific-notation=fixed
        ]{\myCoeffLWing}{\metre^{-2}}\, Y^3
\end{split}
\]

\end{myExampleX}














\end{document}