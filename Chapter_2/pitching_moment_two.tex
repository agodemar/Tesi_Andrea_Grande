\documentclass[[12pt,twoside]{book}
\usepackage{_my_document_style}
\begin{document}
%
\def\myInitialValueA{999}
\def\myISAAirGasConstNMTKGK{287}
\def\myISAAirAdiabaticIndex{1.4}
\def\myISAAirTemperatureSeaLevelK{288.16}
\def\myISAAirDensitySeaLevelKGMTcubed{1.225}
\def\myISAAirDensitySeaLevelSLUGFTcubed{0.0023769}
\def\myAltitudeMT{3000}
\def\myAltitudeFT{9842.52}
\def\myISALapseRateKMT{-0.0065}
\def\myAirDensityKGMTcubed{0.9091}
\def\myAirDensitySLUGFTcubed{0.0017639}
\def\myAirDensityRatio{0.7421}
\def\myAirDynamicViscosiyMTSecKG{1.694e-05}
\def\myAirTemperatureK{268.66}
\def\myAirSoundSpeedMTSec{328.55378}
\def\myAirSoundSpeedFTSec{1077.93236}
\def\myPrepareForStackNull{-1e+307}
\def\myInitialValueB{999}
\def\myMach{0.147}
\def\myFlightSpeedMTSec{48.368}
\def\myFlightSpeedFTSec{158.68767}
\def\myFlightSpeedKMH{174.1}
\def\myFlightSpeedKTS{94.02}
\def\myFlightSpeedEASMTSec{41.66667}
\def\myFlightSpeedEASFTSec{136.70166}
\def\myFlightSpeedEASKMH{150}
\def\myFlightSpeedEASKTS{80.99}
\def\myFlightQbarPA{1063.36806}
\def\myFlightQbarBAR{0.011}
\def\myInitialValueB{999}
\def\myUBodyMTSec{48.368}
\def\myUBodyFTSec{158.68767}
\def\myUBodyKMH{174.1}
\def\myUBodyKTS{94.02}
\def\myVBodyMTSec{0}
\def\myVBodyFTSec{0}
\def\myVBodyKMH{0}
\def\myVBodyKTS{0}
\def\myWBodyMTSec{0}
\def\myWBodyFTSec{0}
\def\myWBodyKMH{0}
\def\myWBodyKTS{0}

%
\begin{myExampleX}{Pitching moment around the wing aerodynamic center}{\ding{46}}% \ \Keyboard\ %
\label{example:Wing:Cmac:B}
%
\noindent
The example~\ref{example:Wing:Cmac:A} (see figure~\ref{fig:Wing:Cmac:Results:A}) is taken up here considering
a more accurate law of basic loading.
It was obtained numerically and is plotted in the figure~\ref{fig:Wing:Cmac:Results:BA}.
The numerical values are shown in the table~\ref{tab:Wing:Cmac:Results:B}.
%
In the figure~\ref{fig:Wing:Cmac:Results:BB} the function is compared
$(cC_\ell)_\mathrm{b,Schrenk}(Y)$, obtained in conjunction with the application of the Schrenk engineering method for the additional loading (see example~\ref{example:Wing:Cmac:A}),
with function $(cC_\ell)_\mathrm{b}(Y)$, which interpolates a sequence of discrete values
obtained by applying Prandtl's Lifting Line Theory.

\begin{table}[tb]
\caption{%
 Wing assigned in the examples~\ref{example:Wing:Cmac:A} and ~\ref{example:Wing:Cmac:B}.
Numerical values with which the basic loading was calculated $(cC_\ell)_\mathrm{b}$ 
  applying Prandtl's Lifting Line Theory..
}
\label{tab:Wing:Cmac:Results:B}
\centering
\includegraphics[width=0.95\textwidth]{Chapter_2/pitching_moment_two/wing_Cmac_2_loading_table.pdf}
\end{table}

In the case discussed here the function $x_\mathrm{b}(Y)$ and the value of
 $C_{\mathcal{M}_\mathlarger{\mathrm{ac,b}}}$ coincide with what was obtained
in the example~\ref{example:Wing:Cmac:A}.
As for the value of $C_{\mathcal{M}_\mathlarger{\mathrm{ac,b}}}$, having a more accurate trend of the basic loading, a different value is obtained

\[
C_{\mathcal{M}_\mathlarger{\mathrm{ac,b}}} 
  =
  \frac{2}{S\,\bar{c}} \int_0^{b/2} \big(cC_{\ell}\big)_\mathrm{b}(Y) \; x_\mathrm{b}(Y)\, \diff{Y}
  = 
  \mathunderline{mydarkblue}{ \SI[round-precision=5]{\myCmZeroBWing}{} }
\]
The integral in this formula was evaluated numerically after defining a function
interpolating $(cC_\ell)_\mathrm{b}(Y)$ piecewise linear in the interval
$[0,\frac{1}{2}b]$.
The figure~\ref{fig:Wing:Cmac:Results:BC} plot the discrete values of the
integrating function
%$\big(cC_{\ell}\big)_\mathrm{b}(Y) \, x_\mathrm{b}(Y)$,
$\big(cC_{\ell}\big)_\mathrm{b} \, x_\mathrm{b}$,
already listed in the seventh column of the table~\ref{tab:Wing:Cmac:Results:B}.
The area subtended by the graph of the function is negative as well as, obviously, the sign
of \smash{$C_{\mathcal{M}_\mathlarger{\mathrm{ac,b}}}$}.

\end{myExampleX}
\begin{figure}
  [t]%[H]%[!htbp]
  %\centering
  %\checkoddpage
  %\centering
    \includegraphics[width=0.80\textwidth]{Chapter_2/pitching_moment_two/wing_Cmac_2_loading_drawing.pdf}%
  \caption{\finalhyphendemerits=1000
           Wing assigned in the examples~\ref{example:Wing:Cmac:A} and~\ref{example:Wing:Cmac:B}.
           Additional loading diagrams $(cC_\ell)_\mathrm{a1}$ and basic loading $(cC_\ell)_\mathrm{b}$ obtained numerically
            (from the Prandtl's Lifting Line Theory)
            and law $x_\mathrm{b}(Y)$ of the basic loading arms.}
  \label{fig:Wing:Cmac:Results:BA}%
\end{figure}


%-----------------------------------------------------------------------------------------------
\begin{figure}[t]%[H]%[!htbp]
  %\centering
  %\checkoddpage
  %\centering
    \includegraphics[width=0.80\textwidth]{Chapter_2/pitching_moment_two/wing_Cmac_2_loading_drawing_2.pdf}%
  \caption{\finalhyphendemerits=1000
           Wing assigned in the examples~\ref{example:Wing:Cmac:A} and ~\ref{example:Wing:Cmac:B}.
           Detail of the figure~\ref{fig:Wing:Cmac:Results:BA}.
           Basic loading diagram$(cC_\ell)_\mathrm{b}$ obtained numerically
            with the Lifting Line Theory
         and approximate basic loading $(cC_\ell)_\mathrm{b,Schrenk}$ obtained concurrently
            the application of the Schrenk engineering method for the additional loading
           (see also the figure~\ref{fig:Wing:Cmac:Results:A}).}
  \label{fig:Wing:Cmac:Results:BB}%
\end{figure}
%-----------------------------------------------------------------------------------------------
\begin{figure}
  [t]%[H]%[!htbp]
  %\centering
  %\checkoddpage
  %\centering
    \includegraphics[width=0.80\textwidth]{Chapter_2/pitching_moment_two/wing_Cmac_2_loading_drawing_3.pdf}%
  \caption{\finalhyphendemerits=1000
         Wing assigned in the examples~\ref{example:Wing:Cmac:A} and ~\ref{example:Wing:Cmac:B}.
          Diagram of the two product functions $(cC_\ell)_\mathrm{b}(Y)\,x_\mathrm{b}(Y)$ 
          and $(cC_\ell)_\mathrm{b,Schrenk}(Y)\,x_\mathrm{b}(Y)$
           (see the    figure~\ref{fig:Wing:Cmac:Results:BC}).
           The integral of these functions allows to estimate the contribution
           \smash{$C_{\mathcal{M}_\mathlarger{\mathrm{ac,b}}}$} to the coefficient
            of pitching moment \smash{$C_{\mathcal{M}_\mathlarger{\mathrm{ac}}}$}
      of the wing around its aerodynamic center.
  }
  \label{fig:Wing:Cmac:Results:BC}%
\end{figure}%
\end{document}
