\PassOptionsToPackage{usenames,dvipsnames}{xcolor}

\documentclass[crop=true]{standalone}

\standaloneconfig{border=0mm 0cm 0cm 0cm} % left bottom right top

\usepackage{_my_basic_document_style}

\usepackage{tikz}
  \usetikzlibrary{calc}
  %\usetikzlibrary{fpu}
  \usetikzlibrary{intersections}
  \usetikzlibrary{positioning}
  \usetikzlibrary{trees}
  \usetikzlibrary{arrows}

\usepackage{pgfplots}
\usepackage{pgfplotstable}

% \pgfkeys{
%   /pgf/number format/.cd,
%       use comma
% }
\pgfplotsset{
  compat=newest,
  every axis/.append style={
      semithick,% thick
      tick style={semithick}
  },
  every axis title/.append style={
      font=\relsize{0}
  },
  every axis legend/.append style={
      cells={anchor=west},%
      fill=gray!10,
      font=\relsize{0},
      at={(0.03,0.97)},
      anchor=north west,thin,draw=none
  }
}

\usepackage{booktabs,colortbl}

\begin{document}
%
% RESULTS PRODUCED BY MATHCAD/MATLAB
%
\def\myInitialValueA{999}
\def\myISAAirGasConstNMTKGK{287}
\def\myISAAirAdiabaticIndex{1.4}
\def\myISAAirTemperatureSeaLevelK{288.16}
\def\myISAAirDensitySeaLevelKGMTcubed{1.225}
\def\myISAAirDensitySeaLevelSLUGFTcubed{0.0023769}
\def\myAltitudeMT{3000}
\def\myAltitudeFT{9842.52}
\def\myISALapseRateKMT{-0.0065}
\def\myAirDensityKGMTcubed{0.9091}
\def\myAirDensitySLUGFTcubed{0.0017639}
\def\myAirDensityRatio{0.7421}
\def\myAirDynamicViscosiyMTSecKG{1.694e-05}
\def\myAirTemperatureK{268.66}
\def\myAirSoundSpeedMTSec{328.55378}
\def\myAirSoundSpeedFTSec{1077.93236}
\def\myPrepareForStackNull{-1e+307}
\def\myInitialValueB{999}
\def\myMach{0.147}
\def\myFlightSpeedMTSec{48.368}
\def\myFlightSpeedFTSec{158.68767}
\def\myFlightSpeedKMH{174.1}
\def\myFlightSpeedKTS{94.02}
\def\myFlightSpeedEASMTSec{41.66667}
\def\myFlightSpeedEASFTSec{136.70166}
\def\myFlightSpeedEASKMH{150}
\def\myFlightSpeedEASKTS{80.99}
\def\myFlightQbarPA{1063.36806}
\def\myFlightQbarBAR{0.011}
\def\myInitialValueB{999}
\def\myUBodyMTSec{48.368}
\def\myUBodyFTSec{158.68767}
\def\myUBodyKMH{174.1}
\def\myUBodyKTS{94.02}
\def\myVBodyMTSec{0}
\def\myVBodyFTSec{0}
\def\myVBodyKMH{0}
\def\myVBodyKTS{0}
\def\myWBodyMTSec{0}
\def\myWBodyFTSec{0}
\def\myWBodyKMH{0}
\def\myWBodyKTS{0}

%
\pgfplotstableset{
  col sep=comma, skip first n={20},
}
%
%% http://tex.stackexchange.com/questions/10284/multiple-files-input-to-one-pgfplotstable
\pgfplotstableread{data_loading_1.csv}\myDataLoading
%
% read another table file
\pgfplotstableread{data_loading_2.csv}\dataB
%
% add columns to the first table
\pgfplotstablecreatecol[copy column from table={\dataB}{[index] 1}] {7} {\myDataLoading}
\pgfplotstablecreatecol[copy column from table={\dataB}{[index] 2}] {8} {\myDataLoading}
\pgfplotstablecreatecol[copy column from table={\dataB}{[index] 3}] {9} {\myDataLoading}
%
\pgfplotstabletypeset[
  every even row/.style={
    before row={\rowcolor[gray]{0.9}}
  },
  every head row/.style={
    before row=\toprule,
    %after row=\midrule
    after row={
      \rule{0pt}{1.1em}
      (\si{m}) & (\si{m}) & (\si{m}) & (\si{m}) & (\si{m}) & (\si{m}) & (\si{m^2}) & (\si{m^2})\\
      \midrule
    }
  },
  every last row/.style={
    after row=\bottomrule
  },
  col sep=comma, skip first n={20},
  header=false,
  columns/0/.style={column type=|c,column name=$Y$},
  columns={0,3,5,4,6,7,8,9},
  columns/0/.style={
    column name=$Y$,
    fixed, precision=2,zerofill
  },
  columns/3/.style={
    column name=$\big(cC_\ell\big)_\mathrm{a1}$,
    fixed, precision=3,zerofill
  },
  columns/4/.style={
    column name=$\big(cC_\ell\big)_\mathrm{b}$,
    fixed, precision=4,zerofill
  },
  columns/5/.style={
    column name=$\big(cC_\ell\big)_\mathrm{a1,Schrenk}$,
    fixed, precision=4,zerofill
  },
  columns/6/.style={
    column name=$\big(cC_\ell\big)_\mathrm{b,Schrenk}$,
    fixed, precision=4,zerofill
  },
  columns/7/.style={
    column name=$x_\mathrm{b}$,
    fixed, precision=4,zerofill
  },
  columns/8/.style={
    column name=$\big(cC_\ell\big)_\mathrm{b} x_\mathrm{b}$,
    fixed, precision=4,zerofill
  },
  columns/9/.style={
    column name=$\big(cC_\ell\big)_\mathrm{b,Schrenk} x_\mathrm{b}$,
    fixed, precision=4,zerofill
  },
]
\myDataLoading
%
\end{document}