\PassOptionsToPackage{usenames,dvipsnames}{xcolor}

\documentclass[crop=true]{standalone}

\standaloneconfig{border=0mm 0cm 0cm 0cm} % left bottom right top

\usepackage{_my_basic_document_style}

\usepackage{tikz}
  \usetikzlibrary{calc}
  %\usetikzlibrary{fpu}
  \usetikzlibrary{intersections}
  \usetikzlibrary{positioning}
  \usetikzlibrary{trees}
  \usetikzlibrary{arrows}

\usepackage{pgfplots}
\usepackage{pgfplotstable}

% \pgfkeys{
%   /pgf/number format/.cd,
%       use comma
% }
\pgfplotsset{
  compat=newest,
  every axis/.append style={
      semithick,% thick
      tick style={semithick}
  },
  every axis title/.append style={
      font=\relsize{0}
  },
  every axis legend/.append style={
      cells={anchor=west},%
      fill=gray!10,
      font=\relsize{0},
      at={(0.03,0.97)},
      anchor=north west,thin,draw=none
  }
}

\usepackage{booktabs,colortbl}

\begin{document}
%
% RESULTS PRODUCED BY MATHCAD/MATLAB
%
\def\mySpanWingMT{10.600000}
\def\mySpanWingIMT{6.360000}
\def\mySpanWingIIMT{4.240000}
\def\myChordRootWingMT{1.440000}
\def\myChordRootWingIMT{1.440000}
\def\myChordRootWingIIMT{1.440000}
\def\myChordTipWingMT{0.860000}
\def\myChordTipWingIMT{1.440000}
\def\myChordTipWingIIMT{0.860000}
\def\mySweepLEWingIIDEG{0.000000}
\def\myCoeffAChordWingI{0.000000}
\def\myCoeffBChordWingIMT{1.440000}
\def\myCoeffAChordWingII{-0.273585}
\def\myCoeffBChordWingIIMT{1.440000}
\def\myAlphaZeroLiftRootWingIDEG{-2.500000}
\def\myAlphaZeroLiftTipWingIDEG{-2.500000}
\def\myAlphaZeroLiftRootWingIRAD{-0.043633}
\def\myAlphaZeroLiftTipWingIRAD{-0.043633}
\def\myAlphaZeroLiftRootWingIIDEG{-2.500000}
\def\myAlphaZeroLiftTipWingIIDEG{-1.000000}
\def\myAlphaZeroLiftRootWingIIRAD{-0.043633}
\def\myAlphaZeroLiftTipWingIIRAD{-0.017453}
\def\myTaperRatioWingI{1.000000}
\def\myTaperRatioWingII{0.597222}
\def\myTwistWingIDEG{0.000000}
\def\myTwistWingIRAD{0.000000}
\def\myTwistWingIIDEG{-3.000000}
\def\myTwistWingIIRAD{-0.052360}
\def\myAreaWingIMTsquared{9.158400}
\def\myAreaWingIIMTsquared{4.876000}
\def\myAreaWingMTsquared{14.034400}
\def\myCoeffAAeroTwistWingIRADMT{0.000000}
\def\myCoeffBAeroTwistWingIRAD{-0.043633}
\def\myCoeffAAeroTwistWingIIRADMT{0.012349}
\def\myCoeffBAeroTwistWingIIRAD{-0.043633}
\def\myCoeffATwistWingIRADMT{0.000000}
\def\myCoeffATwistWingIIRADMT{-0.024698}
\def\myCoeffBTwistWingIRAD{0.000000}
\def\myCoeffBTwistWingIIRAD{0.000000}
\def\myAlphaZeroLiftWingIRAD{-0.028653}
\def\myAlphaZeroLiftWingIDEG{-1.641680}
\def\myAlphaZeroLiftWingIIRAD{0.008329}
\def\myAlphaZeroLiftWingIIDEG{0.477233}
\def\myAlphaZeroLiftWingRAD{-0.020323}
\def\myAlphaZeroLiftWingDEG{-1.164447}

%
\pgfplotstableset{
  col sep=comma, skip first n={20},
}
%
%% http://tex.stackexchange.com/questions/10284/multiple-files-input-to-one-pgfplotstable
\pgfplotstableread{data_loading_1.csv}\myDataLoading
%
% read another table file
\pgfplotstableread{data_loading_2.csv}\dataB
%
% add columns to the first table
\pgfplotstablecreatecol[copy column from table={\dataB}{[index] 1}] {7} {\myDataLoading}
\pgfplotstablecreatecol[copy column from table={\dataB}{[index] 2}] {8} {\myDataLoading}
\pgfplotstablecreatecol[copy column from table={\dataB}{[index] 3}] {9} {\myDataLoading}
%
\pgfplotstabletypeset[
  every even row/.style={
    before row={\rowcolor[gray]{0.9}}
  },
  every head row/.style={
    before row=\toprule,
    %after row=\midrule
    after row={
      \rule{0pt}{1.1em}
      (\si{m}) & (\si{m}) & (\si{m}) & (\si{m}) & (\si{m}) & (\si{m}) & (\si{m^2}) & (\si{m^2})\\
      \midrule
    }
  },
  every last row/.style={
    after row=\bottomrule
  },
  col sep=comma, skip first n={20},
  header=false,
  columns/0/.style={column type=|c,column name=$Y$},
  columns={0,3,5,4,6,7,8,9},
  columns/0/.style={
    column name=$Y$,
    fixed, precision=2,zerofill
  },
  columns/3/.style={
    column name=$\big(cC_\ell\big)_\mathrm{a1}$,
    fixed, precision=3,zerofill
  },
  columns/4/.style={
    column name=$\big(cC_\ell\big)_\mathrm{b}$,
    fixed, precision=4,zerofill
  },
  columns/5/.style={
    column name=$\big(cC_\ell\big)_\mathrm{a1,Schrenk}$,
    fixed, precision=4,zerofill
  },
  columns/6/.style={
    column name=$\big(cC_\ell\big)_\mathrm{b,Schrenk}$,
    fixed, precision=4,zerofill
  },
  columns/7/.style={
    column name=$x_\mathrm{b}$,
    fixed, precision=4,zerofill
  },
  columns/8/.style={
    column name=$\big(cC_\ell\big)_\mathrm{b} x_\mathrm{b}$,
    fixed, precision=4,zerofill
  },
  columns/9/.style={
    column name=$\big(cC_\ell\big)_\mathrm{b,Schrenk} x_\mathrm{b}$,
    fixed, precision=4,zerofill
  },
]
\myDataLoading
%
\end{document}