\documentclass[[12pt,twoside]{book}
\usepackage{_my_document_style}
\begin{document}
%
\def\mySpanWingMT{16.000000}
\def\myMach{0.400000}
\def\myAspectRatioWing{9.142857}
\def\myChordRootWingMT{2.500000}
\def\myChordTipWingMT{1.000000}
\def\myTaperRatioWing{0.400000}
\def\myAreaWingMTsquared{28.000000}
\def\myCoeffAChordWing{-0.187500}
\def\myCoeffBChordWingMT{2.500000}
\def\myCLAlphaRootWingRAD{6.150000}
\def\myCLAlphaTipWingRAD{6.050000}
\def\myCoeffAClalphaWingRADMT{-0.012500}
\def\myCoeffAClalphaWingDEGMT{-0.000218}
\def\myCoeffBClalphaWingRAD{6.150000}
\def\myCLAlphaMeanWingRAD{6.107143}
\def\myCLAlphaMeanWingDEG{0.106590}
\def\myInducedDragFactorWing{0.902301}
\def\myCLAlphaWingRAD{4.942481}
\def\myCLAlphaWingDEG{0.086263}
\def\mySweepTmaxWingDEG{-0.540000}
\def\mySweepTmaxWingRAD{-0.009425}
\def\myDownwashGradientLLTAtMachZeroWing{0.381410}
\def\myDownwashGradientLLTWing{0.349568}

%
%
\begin{myExampleX}{Gradient of the downwash angle}{\ding{46}}% \ \Keyboard\ %
\label{example:Wing:Downwash:A}
%
\noindent
We want to calculate the gradient \smash{$\diff{\epsilon}/\diff{\alpha}$} of a finite wing
which has an aspect ratio
\[
\AR 
  = \frac{b^2}{S}
  = \frac{\big(\SI[round-precision=0]{\mySpanWingMT}{\metre}\big)^2}{\SI[round-precision=0]{\myAreaWingMTsquared}{\metre^2}}
  = \mathunderline{mydarkblue}{ \num[round-precision=2]{\myAspectRatioWing} }
\]

The characteristics of the lifting surface and section profiles allow
to obtain an average value of the gradient
\begin{multline}
\nonumber
\bar{C}_{\ell_\mathlarger{\alpha}}
  = \frac{2}{S} \int_0^{b/2} 
      c\big(Y\big) \, C_{\ell_\mathlarger{\alpha}} \big(Y\big) \diff{Y}
\\
  = \frac{2}{ \SI[round-precision=0]{\myAreaWingMTsquared}{\metre^{2}} } 
    \int_0^{ \calcSI[round-precision=1,fixed-exponent=0,scientific-notation=fixed]{0.5*\mySpanWingMT}{\metre} } 
    \big[
      \SI[round-precision=5]{\myCoeffAClalphaWingRADMT}{\big(\radian\,\metre\big)^{-1}} \, Y
      + \SI[round-precision=3]{\myCoeffBClalphaWingRAD}{\radian^{-1}}
    \big]
    \rule{5em}{0pt}% <-- SPACER
\\
    \rule{5em}{0pt}% <-- SPACER
    \cdot \big[
      \SI[round-precision=3]{\myCoeffAChordWing}{} \, Y
      + \SI[round-precision=2]{\myCoeffBChordWingMT}{\metre}
    \big]
    \diff{Y}
\\
  = \mathunderline{mydarkblue}{ \SI[round-precision=3]{\myCLAlphaMeanWingRAD}{\radian^{-1}} }
  = \mathunderline{mydarkblue}{ \SI[round-precision=4]{\myCLAlphaMeanWingDEG}{\deg^{-1}} }
\end{multline}
%
It can be used for the calculation of the lift gradient of the finished wing
\[
C_{L_\mathlarger{\alpha}}
  = 
    \frac{
      \bar{C}_{\ell_\mathlarger{\alpha}}
    }{
      1 + \dfrac{\bar{C}_{\ell_\mathlarger{\alpha}}}{\pi \AR \, e_\Wing}
    }
  =
    \frac{
      \SI[round-precision=3]{\myCLAlphaMeanWingRAD}{\radian^{-1}}
    }{
      1 + 
        \dfrac{ \SI[round-precision=3]{\myCLAlphaMeanWingRAD}{\radian^{-1}} }{
          \num[round-precision=2]{3.14} 
          \cdot \SI[round-precision=2]{\myAspectRatioWing}{}
          \cdot \SI[round-precision=2]{\myInducedDragFactorWing}{}
        }
    }
  = \mathunderline{mydarkblue}{ \SI[round-precision=3]{\myCLAlphaWingRAD}{\radian^{-1}} }
  = \mathunderline{mydarkblue}{ \SI[round-precision=4]{\myCLAlphaWingDEG}{\deg^{-1}} }
\]
having assumed a \emph{span efficiency factor} 
%$e_\Wing=\SI[round-precision=2]{\myInducedDragFactorWing}{}$.
given by the empirical formula%
\[
e_\Wing = \frac{2}{2-\AR+\sqrt{4+\AR^2\big(1+\tan^2\Lambda_{t_\mathrm{max}}\big)}}
  = \SI[round-precision=2]{\myInducedDragFactorWing}{}
\]
with $\Lambda_{t_\mathrm{max}}$ the sweep angle of the point line of percentage thickness
maximum of profiles. In this case, all the percentage thicknesses are found
at $30\%$ of chords and 
$\Lambda_{t_\mathrm{max}} = \SI[round-precision=2]{\mySweepTmaxWingDEG}{\deg}$
and $\tan \Lambda_{t_\mathrm{max}} = \calcSI[round-precision=3,fixed-exponent=0,scientific-notation=fixed]{tan(\mySweepTmaxWingRAD)}{}$.

It is therefore obtained
\[
\left.\frac{ \diff{\epsilon} }{ \diff{\alpha} }\right|_{\Mach=0}
  = 2\, \dfrac{\bar{C}_{\ell_\mathlarger{\alpha}}}{\pi \AR \, e_\Wing}
  = 2\,
  \dfrac{ \SI[round-precision=3]{\myCLAlphaMeanWingRAD}{\radian^{-1}} }{
    \num[round-precision=2]{3.14} 
      \cdot \SI[round-precision=2]{\myAspectRatioWing}{}
      \cdot \SI[round-precision=2]{\myInducedDragFactorWing}{}
  }
  = \mathunderline{mydarkblue}{ \SI[round-precision=2]{\myDownwashGradientLLTAtMachZeroWing}{} }
\]
and finally
\[
\frac{ \diff{\epsilon} }{ \diff{\alpha} }
  =
  \sqrt{1-\Mach^2} \,
  \left.\frac{ \diff{\epsilon} }{ \diff{\alpha} }\right|_{\Mach=0}
  = \sqrt{1-\SI[round-precision=2]{\myMach}{}^2} \,
    \cdot \SI[round-precision=2]{\myDownwashGradientLLTAtMachZeroWing}{}      
  = \mathunderline{mydarkblue}{ \SI[round-precision=2]{\myDownwashGradientLLTWing}{} }
\]
\end{myExampleX}
\end{document}