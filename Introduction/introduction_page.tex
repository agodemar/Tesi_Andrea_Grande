\thispagestyle{empty}

This thesis deals with conceptual topics of aircraft stability and control and in general of the aerodynamics applied to the aircraft.

The encountered topics are the main parameters which have to be estimated for an aircraft, such as its fundamental geometric and aerodynamics characteristics.

Introductory calculations are made in the chapter 1. Subsequently the calculations concerning the wing alone are dealt with in the chapter 2. The isolated fuselage and then the wing/fuselage configuration are dealt with in the chapter 3 and finally the tailplane is dealt with in the chapter 4. These preliminary calculations are necessary for the estimation of the aircraft performance.

The formal structure given to this project is that of a technical report where the numerical results obtained are documented as in a real scientific document.  

The production workflow is essentially based on the use of Matlab, a programming language, which is used as a calculation software. For example in some situations it was necessary to create anonymous functions and consequent use them in numerical formulas for the calculation of definite integrals. Sometimes it was necessary also the interpolation technique and the use of pgfplots to create graphs in \LaTeX{} (see more insights about the software used in the appendix ~\ref{chap:Appendix:Software:Used}).  Some scripts are shown as examples in the appendix~\ref{chap:Appendix:Matlab:Scripts}. 

The entire drafting of this manuscript is made in \LaTeX{}, using Overleaf platform. 
Many data files were created by Matlab, which contain all the obtained results. They were uploaded to the Github online repository, specially created for this activity, and then imported in the \LaTeX{} project doing a synchronization.

The numerical values present in these data files, were taken for the preparation of the exercises, a step that was fundamental given the high number of exercises in order to avoid errors.
%~\ref{chap:Appendix:Matlab:Scripts}