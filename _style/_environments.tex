\usepackage[amsmath,hyperref]%
  {ntheorem}

% definisce uno stile di teorema ad header vuoto
% richiede ntheorem.sty
\makeatletter
  \newtheoremstyle{mytheoremstyle}
    {\item[]}%
    {\item[]}
\makeatother
\theoremstyle{mytheoremstyle}% <------  seleziona lo stile appena definito
%\theoremstyle{margin}       %          ... altri stili predefiniti in ntheorem
%\theoremstyle{nonumberplain}

% nuovo theorem-like environment
\newtheorem{myExampleT}{}[chapter]

% myExample label & format
\newcommand\myExampleLabel{Example}
\newcommand\myExampleName{\myExampleLabel}
\newcommand\myExampleLabelFormat{%
    %\textbf{\bf\ding{46}\ \myExampleLabel\ \themyExampleT}
    \textbf{\upshape\hspace{3pt}\myExampleLabel\ \themyExampleT}\hfill\textbf{\bf\ding{46}\ \raisebox{-2pt}[0pt][0pt]{\relsize{4}\Keyboard\hspace{2pt}\ComputerMouse}\ }%
}
\newcommand\myExampleEndMark{%
    {\ding{118}}% \ding{111}
}

\newcommand\myExampleMarkPencilKeyboardMouse{%
    \bf\ding{46}\ \raisebox{-2pt}[0pt][0pt]{\relsize{4}\Keyboard\hspace{2pt}\ComputerMouse}%
}
\newcommand\myExampleMarkKeyboardMouse{%
    \raisebox{-2pt}[0pt][0pt]{\relsize{4}\Keyboard\hspace{2pt}\ComputerMouse}%
}

% environment per gli Esempi
%\makeatletter
\newenvironment{myExample}{%
    %###########################################################
    % ciò che viene eseguito all'inizio dell'environment
    %###########################################################
    \begin{myExampleT}
    %\stepcounter{myexamplecounter}% incrementa il contatore
    \par\vskip 8pt% a capo e skip verticale
    \noindent
    % striscia colorata
    \makebox[0em][l]{%
        {\color{mylightblue!50}\rule[-8pt]{\textwidth}{1.8em}}%
    }%
    % scritta
    \makebox[1.0\textwidth][c]{%
        {\myExampleLabelFormat}%
    }%
    \vskip 8pt% recupera gli 8pt di \rule[-8pt]
    \smallskip%
    %\noindent%
    \upshape% testo normale
}{%
    %###########################################################
    % ciò che viene eseguito alla fine dell'environment
    %###########################################################
    %\addvspace{3.2ex plus 0.8ex minus 0.2ex}%
    \end{myExampleT}
    %\par\vskip 2pt
    %\noindent
    \smallskip
    % striscia colorata
    \makebox[0em][l]{%
        {\color{mylightblue!50}\rule%
                            [0pt]%[-9pt]%
                            {\textwidth}{1.8em}}%
    }%
    % simbolo di chiusura
    \makebox[1.0\textwidth][c]{%
        {\raisebox{7pt}{\myExampleEndMark}}
    }%
    %\par\vskip-2pt%
    %\noindent
    \medskip
}%
%\makeatother

%.................................................. Matlab Tip
% nuovo theorem-like environment
\newtheorem{myMatlabTipT}{}[chapter]

% myExample label & format
\newcommand\myMatlabTipLabel{Matlab {\itshape tip}}
\newcommand\myMatlabTipLabelFormat{%
    %\textbf{\bf\ding{46}\ {\relsize{2}\Keyboard}\ \myMatlabTipLabel\ \themyMatlabTipT}
    \textbf{\upshape\hspace{3pt}\myMatlabTipLabel\ \themyMatlabTipT}\hfill\textbf{\bf\ding{46}\ \raisebox{-2pt}[0pt][0pt]{\relsize{4}\Keyboard\hspace{2pt}\ComputerMouse}\ }%
}
\newcommand\myMatlabTipEndMark{%
    {\ding{118}}% \ding{111}%
}

% environment per gli Esempi
%\makeatletter
\newenvironment{myMatlabTip}{%
    %###########################################################
    % ci? che viene eseguito all'inizio dell'environment
    %###########################################################
    \begin{myMatlabTipT}%
    %\stepcounter{myexamplecounter}% incrementa il contatore
    \par\vskip 8pt% a capo e skip verticale
    \noindent
    % striscia colorata
    \makebox[0em][l]{%
        {\color{mylightblue!50}\rule[-8pt]{\textwidth}{1.8em}}%
    }%
    % scritta
    \makebox[1.0\textwidth][l]{%
        {\myMatlabTipLabelFormat}%
    }%
    \vskip 8pt% recupera gli 8pt di \rule[-8pt]
    \smallskip%
    %\noindent
    \upshape% testo normale
}{%
    %###########################################################
    % ci? che viene eseguito alla fine dell'environment
    %###########################################################
    %\addvspace{3.2ex plus 0.8ex minus 0.2ex}%
    \end{myMatlabTipT}
    %\par\vskip 2pt
    %\noindent
    \smallskip
    % striscia colorata
    \makebox[0em][l]{%
        {\color{mylightblue!50}\rule%
                            [0pt]%[-9pt]%
                            {\textwidth}{1.8em}}%
    }%
    % simbolo di chiusura
    \makebox[1.0\textwidth][c]{%
        {\raisebox{7pt}{\myExampleEndMark}}
    }%
    %\par\vskip-2pt%
    %\noindent
    \medskip
}%
%\makeatother

% nuovo theorem-like environment
%\newtheorem{myExampleTX}{}[chapter]

% environment per gli Esempi esteso (X)
%\makeatletter
%
% http://stackoverflow.com/questions/1903254/how-to-define-in-latex-a-new-counter-that-includes-the-chapter-number-too
%\newcounter{myExampleX}
%\def\themyExampleX{\thechapter.\arabic{myExampleX}}
\newenvironment{myExampleX}[3][2.0em]{% 
    % accetta tre argomenti: 
    % [#1 (dimensione opzionale, default = 2.0em) altezza della striscia azzurra, 
    % #2 titolo, 
    % #3 simbolo nell'header
    %
    % TODO: implementare l'ambiente myExerciseX analogamente
    %
    %###########################################################
    % qui codice che viene eseguito all'inizio dell'environment
    %###########################################################
    \begin{myExampleT}
	%\refstepcounter{myExampleX}% incrementa il contatore
    \par\vskip 8pt% a capo e skip verticale
    \noindent
    % striscia colorata
    \makebox[0em][l]{%
        {\color{mylightblue!50}\rule[-#1 + 1.2em]{\textwidth}{#1}}%
    }%
    \checkoddpage% needs changepage
    \ifoddpage% needs changepage, needs two laTeX passes
      % RHS page
      % scritta
      \makebox[1.0\textwidth][l]{%
        %{\myExampleLabelFormat}%
        \textbf{\upshape\hspace{3pt}\myExampleLabel\ \themyExampleT:}\rule{2ex}{0pt}\parbox[t]{0.80\textwidth}{#2}%
        %\hfill\textbf{#2\ }%
      }%
      \makebox[0em][l]{\rule{1.2ex}{0pt}\textbf{#3}}    
    \else% 
      % LHS page
      \makebox[0em][r]{\textbf{#3}\rule{1.2ex}{0pt}}
      % scritta
      \makebox[1.0\textwidth][l]{%
        %{\myExampleLabelFormat}%
        \textbf{\upshape\hspace{3pt}\myExampleLabel\ \themyExampleT:}\rule{2ex}{0pt}\parbox[t]{0.80\textwidth}{#2}%
        %\hfill\textbf{#2\ }%
      }%
    \fi
    %
    \vskip 8pt% recupera gli 8pt di \rule[-8pt]
    \smallskip%
    %\noindent%
    \upshape% testo normale
}{%
    %###########################################################
    % ci? che viene eseguito alla fine dell'environment
    %###########################################################
    %\addvspace{3.2ex plus 0.8ex minus 0.2ex}%
    \end{myExampleT}
    %\par\vskip 2pt
    \noindent
    \smallskip
    % striscia colorata
    \makebox[0em][l]{%
        {\color{mylightblue!50}\rule%
                            [0pt]%[-9pt]%
                            {\textwidth}{0.6em}}% 1.8em
    }%
    % simbolo di chiusura
    \makebox[1.0\textwidth][c]{%
        %{\raisebox{7pt}{\myExampleEndMark}}
        {\raisebox{0.6pt}{\relsize{-2}\myExampleEndMark}}
    }%
    %\par\vskip-2pt%
    %\noindent
    \medskip
}%
%\makeatother

% environment per gli Esercizi esteso (X)
%\makeatletter

% nuovo theorem-like environment
\newtheorem{myExerciseT}{}[chapter]

\newcommand\myExerciseEndMark{%
    {\ding{118}}% \ding{111}
}

% myExercise label & format
\newcommand\myExerciseLabel{Exercise}
\newcommand\myExerciseLabelFormat{%
    %\textbf{\bf\ding{46}\ \myExampleLabel\ \themyExampleT}
    \textbf{\upshape\hspace{3pt}\myExampleLabel\ \themyExampleT}%
    \hfill\textbf{\bf\ding{46}\ \raisebox{-2pt}[0pt][0pt]{\relsize{4}\Keyboard\hspace{2pt}\ComputerMouse}\ }%
}

\newenvironment{myExerciseX}[2]{% accetta due argomenti: #1 titolo, #2 simbolo nell'header
    %###########################################################
    % ci? che viene eseguito all'inizio dell'environment
    %###########################################################
    \begin{myExerciseT}
    %\stepcounter{myexamplecounter}% incrementa il contatore
    \par\vskip 8pt% a capo e skip verticale
    \noindent
    % striscia colorata
    \makebox[0em][l]{%
        {\color{mylightblue!50}\rule[-8pt]{\textwidth}{1.8em}}%
    }%
    % scritta
    \makebox[1.0\textwidth][c]{%
        %{\myExampleLabelFormat}%
        \textbf{\upshape\hspace{3pt}\myExerciseLabel\ \themyExerciseT:}\quad\textit{#1}\hfill\textbf{#2\ }%
    }%
    \vskip 8pt% recupera gli 8pt di \rule[-8pt]
    \smallskip%
    %\noindent%
    \upshape% testo normale
}{%
    %###########################################################
    % ci? che viene eseguito alla fine dell'environment
    %###########################################################
    %\addvspace{3.2ex plus 0.8ex minus 0.2ex}%
    \end{myExerciseT}
    %\par\vskip 2pt
    \noindent
    \smallskip
    % striscia colorata
    \makebox[0em][l]{%
        {\color{mylightblue!50}\rule%
                            [0pt]%[-9pt]%
                            {\textwidth}{0.6em}}% 1.8em
    }%
    % simbolo di chiusura
    \makebox[1.0\textwidth][c]{%
        %{\raisebox{7pt}{\myExampleEndMark}}
        {\raisebox{0.6pt}{\relsize{-2}\myExerciseEndMark}}
    }%
    %\par\vskip-2pt%
    %\noindent
    \medskip
}%
%\makeatother
