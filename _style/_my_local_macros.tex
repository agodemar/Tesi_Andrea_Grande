% AUTHOR: Agostino De Marco, agostino.demarco@unina.it

\newcommand{\mystrutt}[2]{\makebox[0pt]{\rule[#1]{0pt}{#2}}}

%
%---------------------------------------
% emphasize equation, needs empheq and tcolorbox
\newtcbox{\mymath}[1][]{%
    nobeforeafter, 
    % math upper, 
    tcbox raise base,
    enhanced, colframe=gray, %blue!30!black,
    colback=blue!30, boxrule=1pt,
    #1}
%---------------------------------------
%

\newcommand{\myMathSF}[1]{\mbox{\sffamily #1}}

%................................ Vectors
%\renewcommand{\vec}[1]{%
%                \ifMathTimeProII\boldsymbol{#1}\else\vec{#1}\fi%
%             }

\renewcommand{\vec}[1]{\boldsymbol{#1}}%
\newcommand{\VEC}[1]{\vec{#1}}%

\newcommand{\vecFT}{\ensuremath\vec{F}_{\mspace{-6mu}\text{T}}}
\newcommand{\vecMT}{\ensuremath\vec{\mathcal{M}}_{\mspace{-3mu}\text{T}}}
\newcommand{\vecFA}{\ensuremath\vec{F}_{\mspace{-6mu}\text{A}}}
\newcommand{\vecMA}{\ensuremath\vec{\mathcal{M}}_{\mspace{-3mu}\text{A}}}

\newcommand{\MATLAB}%
   {\mbox{Matlab}\xspace}
   %{\mbox{\tt MATLAB}\xspace}
\newcommand{\Matlab}%
   {\mbox{Matlab}\xspace}
   %{\mbox{\tt MATLAB}\xspace}
\newcommand{\SIMULINK}%
   {\mbox{Simulink}\xspace}
   %{\mbox{\tt Simulink}\xspace}
\newcommand{\Simulink}%
   {\mbox{Simulink}\xspace}
   %{\mbox{\tt Simulink}\xspace}

% operatori matematici
\newcommand*{\diff}{\mathop{}\!\mathrm{d}}% vedi guida GuIT, Beccari, pp. 113-114
\newcommand*{\ddt}[1]{\frac{\textstyle\diff{#1}}{\textstyle\diff{t}}}

% http://tex.stackexchange.com/questions/9466/color-underline-a-formula
\def\mathunderline#1#2{\color{#1}\underline{{\color{black}#2}}\color{black}}

% macro per la definizione della simbologia aeronautica

\newcommand{\Vinf}{\ensuremath{V_{\!\infty}}\xspace} % velocità asintotica
\newcommand{\vVinf}{\ensuremath{\vec{V}_{\!\!\!\infty}}\xspace} % vettore velocità asintotica

\newcommand{\Vzero}{\ensuremath{V_{\!0}}\xspace} % velocità asintotica
\newcommand{\vVzero}{\ensuremath{\vec{V}_{\mspace{-8mu}0}}\xspace} % vettore velocità asintotica

\newcommand{\Vdive}{\ensuremath{V_\text{dive}}\xspace}
\newcommand{\Vcruise}{\ensuremath{V_\text{cruise}}\xspace}

% Often used sub/superscripts
% packages amsmath & mathtools are needed

\newcommand{\Aero}{\ensuremath{\text{%
  %\mdseries\scshape a%
  A%
  }}}
\newcommand{\Thrust}{\ensuremath{\text{%
  %\mdseries\scshape t%
  T%
  }}}
\newcommand{\Grav}{\ensuremath{\text{%
  %\mdseries\scshape g%
  G%
  }}}
\newcommand{\earth}{\ensuremath{\text{%
  %\mdseries\scshape e%
  E%
  }}}% \Earth is defined elsewhere
\newcommand{\CMass}{\ensuremath{\text{%
  %\mdseries\scshape cm
  cm%
  }}}
\newcommand{\Body}{\ensuremath{\text{%
  %\mdseries\scshape b%
  B%
  }}}
\newcommand{\Fuselage}{\ensuremath{\text{%
  %\mdseries\scshape f
  F%
  }}}
\newcommand{\Wing}{\ensuremath{\text{%
   %\mdseries\scshape w%
   W%
   }}}
\newcommand{\Canard}{\ensuremath{\text{%
  %\mdseries\scshape c
  C%
  }}}
\newcommand{\Vertical}{\ensuremath{\text{%
  %\mdseries\scshape v%
  V%
  }}}
\newcommand{\VerticalI}{\ensuremath{\text{%
  %\mdseries\scshape i%
  I%
  }}}
\newcommand{\Inertial}{\ensuremath{\text{%
  %\mdseries\scshape i%
  I%
  }}}
\newcommand{\Interference}{\ensuremath{\text{%
  %\mdseries\scshape i%
  I%
  }}}

\newcommand{\Wind}{\ensuremath{\text{%
  %\mdseries\scshape w%
  wind%
  }}}
\newcommand{\Stability}{\ensuremath{\text{%
  %\mdseries\scshape s%
  S%
  }}}

\newcommand{\Stab}{\ensuremath{\mathrm{S}}}

\newcommand{\Constr}{\ensuremath{\text{
  %\mdseries\scshape c%
  C%
  }}}

\newcommand{\SeaLevel}{\ensuremath{\text{
  %\mdseries\scshape sl%
  SL%
  }}}
\newcommand{\GroundTrack}{\ensuremath{\text{%
  %\mdseries\scshape gt%
  GT%
  }}}
\newcommand{\elev}{\mathrm{e}}
\newcommand{\elevator}{\elev}

\newcommand{\rud}{\mathrm{r}}
\newcommand{\rudder}{\rud}

\newcommand{\ail}{\mathrm{a}}
\newcommand{\aileron}{\ail}

\newcommand{\stab}{\mathrm{s}}
\newcommand{\stabilator}{\stab}

\newcommand{\Htail}{\mathrm{H}}
\newcommand{\Vtail}{\mathrm{V}}

\newcommand{\flap}{\mathrm{f}}
\newcommand{\tab}{\mathrm{t}}

% pedici per X, Y, Z, ...

\newcommand{\aero}{\Aero}

\newcommand{\thrust}{\Thrust}

\newcommand{\grav}{\grav}

\newcommand{\XA}{\ensuremath{X_\Aero}}
\newcommand{\YA}{\ensuremath{Y_\Aero}}
\newcommand{\ZA}{\ensuremath{Z_\Aero}}
\newcommand{\LA}{\ensuremath{\mathcal{L}_\Aero}}
\newcommand{\MA}{\ensuremath{\mathcal{M}_\Aero}}
\newcommand{\NA}{\ensuremath{\mathcal{N}_\Aero}}

\newcommand{\wing}{\Wing}
\newcommand{\body}{\Body}
\newcommand{\BVH}{\ensuremath{\mathrm{BVH}}}
\newcommand{\WB}{\ensuremath{\mathrm{WB}}}
\newcommand{\WiB}{\ensuremath{\mathrm{W(B)}}}
\newcommand{\BiW}{\ensuremath{\mathrm{B(W)}}}
\newcommand{\WBV}{\ensuremath{\mathrm{WBV}}}
\newcommand{\WBH}{\ensuremath{\mathrm{WBH}}}
\newcommand{\Nose}{\ensuremath{\mathrm{N}}}

\newcommand{\HiB}{\ensuremath{\mathrm{H(B)}}}
\newcommand{\BiH}{\ensuremath{\mathrm{B(H)}}}



% Mach and Reynolds numbers
\newcommand{\Mach}{\ensuremath{M}\xspace}%
\newcommand{\Reynolds}{\ensuremath{\mathit{Re}}\xspace}


\newcommand{\sAlpha}{\ensuremath{S_{{\alpha}}}}
\newcommand{\cAlpha}{\ensuremath{C_{{\alpha}}}}
\newcommand{\sBeta}{\ensuremath{S_{{\beta}}}}
\newcommand{\cBeta}{\ensuremath{C_{{\beta}}}}
\newcommand{\sNu}{\ensuremath{S_{{\nu}}}}
\newcommand{\cNu}{\ensuremath{C_{{\nu}}}}
\newcommand{\sGamma}{\ensuremath{S_{{\gamma}}}}
\newcommand{\cGamma}{\ensuremath{C_{{\gamma}}}}
\newcommand{\sPsi}{\ensuremath{S_{{\psi}}}}
\newcommand{\cPsi}{\ensuremath{C_{{\psi}}}}
\newcommand{\sTheta}{\ensuremath{S_{{\theta}}}}
\newcommand{\cTheta}{\ensuremath{C_{{\theta}}}}
\newcommand{\sPhi}{\ensuremath{S_{{\phi}}}}
\newcommand{\cPhi}{\ensuremath{C_{{\phi}}}}


%% 2010

% Ixx, Ixy, etc (2013)
\newcommand{\Ixx}{\ensuremath{I_{\text{\textit{xx}}}\xspace}}
\newcommand{\Ixy}{\ensuremath{I_{\text{\textit{xy}}}\xspace}}
\newcommand{\Ixz}{\ensuremath{I_{\text{\textit{xz}}}\xspace}}
\newcommand{\Iyx}{\ensuremath{I_{\text{\textit{yx}}}\xspace}}
\newcommand{\Iyy}{\ensuremath{I_{\text{\textit{yy}}}\xspace}}
\newcommand{\Iyz}{\ensuremath{I_{\text{\textit{yz}}}\xspace}}
\newcommand{\Izx}{\ensuremath{I_{\text{\textit{zx}}}\xspace}}
\newcommand{\Izy}{\ensuremath{I_{\text{\textit{zy}}}\xspace}}
\newcommand{\Izz}{\ensuremath{I_{\text{\textit{zz}}}\xspace}}
\newcommand{\IxxB}{\ensuremath{I_{\text{\textit{x}}_\mathrm{B}\text{\textit{x}}_\mathrm{B}}\xspace}}
\newcommand{\IzzB}{\ensuremath{I_{\text{\textit{z}}_\mathrm{B}\text{\textit{z}}_\mathrm{B}}\xspace}}
\newcommand{\IxzB}{\ensuremath{I_{\text{\textit{x}}_\mathrm{B}\text{\textit{z}}_\mathrm{B}}\xspace}}
\newcommand{\IxxS}{\ensuremath{I_{\text{\textit{x}}_\mathrm{S}\text{\textit{x}}_\mathrm{S}}\xspace}}
\newcommand{\IzzS}{\ensuremath{I_{\text{\textit{z}}_\mathrm{S}\text{\textit{z}}_\mathrm{S}}\xspace}}
\newcommand{\IxzS}{\ensuremath{I_{\text{\textit{x}}_\mathrm{S}\text{\textit{z}}_\mathrm{S}}\xspace}}

\newcommand{\vXkin}{\vec{x}_{\mspace{-0.0mu}\textrm{k}}}
\newcommand{\vXdyn}{\vec{x}_{\mspace{-0.0mu}\textrm{d}}}
\newcommand{\dotvXkin}{\dot{\vec{x}}_{\mspace{-0.0mu}\textrm{k}}}
\newcommand{\dotvXdyn}{\dot{\vec{x}}_{\mspace{-0.0mu}\textrm{d}}}
\newcommand{\Xkin}{x_{\mspace{-0.0mu}\textrm{k}}}
\newcommand{\Xdyn}{x_{\mspace{-0.0mu}\textrm{d}}}
\newcommand{\dotXkin}{\dot{x}_{\mspace{-0.0mu}\textrm{k}}}
\newcommand{\dotXdyn}{\dot{x}_{\mspace{-0mu}\textrm{d}}}

% needs package "setspace" and Array macro definition (see Voss, Mathmode)
\makeatletter
\RequirePackage{setspace}
\makeatother
\newcommand*\Array[2][1]{\setstretch{#1}\array {#2}}
\let\endArray\endarray

\newcommand{\Transpose}{\ensuremath{^{\mathrm{T}}}}

% Environments for enlarged figures

\newcommand{\EnlargedFigure}[3]{%
\begin{figure}[tb]%
  \checkoddpage% needs changepage
  \ifoddpage% needs changepage, needs two laTeX passes
    % RHS page
    \makebox[\textwidth][l]{%
        \begin{minipage}[b]{\textwidth+\marginparwidth+\marginparsep}
            \includegraphics%
                [width=\linewidth]%
                {#1}%
        %
        % Note: caption goes here to have it well centered w.r.t. the whole enlarged minipage
        %
        \caption{\finalhyphendemerits=1000
           #2}
        %
        % Note: caption moved inside the enlarged minipage
        \label{#3}%
        \end{minipage}
    }
  \else%
    % LHS page
    \makebox[\textwidth][r]{%
        \begin{minipage}[b]{\textwidth+\marginparwidth+\marginparsep}
            \includegraphics%
                [width=\linewidth]%
                {#1}%
        %
        % Note: caption goes here to have it well centered w.r.t. the whole enlarged minipage
        %
        \caption{\finalhyphendemerits=1000
           #2}
        %
        % Note: caption moved inside the enlarged minipage
        \label{#3}%
        \end{minipage}
    }
  \fi%
        % Note: caption moved inside the enlarged minipage
%
\end{figure}%
}


\newcommand{\EnlargedFigureI}[4]{%
\begin{figure}[#1]%
  \checkoddpage% needs changepage
  \ifoddpage% needs changepage, needs two laTeX passes
    % RHS page
    \makebox[\textwidth][l]{%
        \begin{minipage}[b]{\textwidth+\marginparwidth+\marginparsep}
            \includegraphics%
                [width=\linewidth]%
                {#2}%
        %
        % Note: caption goes here to have it well centered w.r.t. the whole enlarged minipage
        %
        \caption{\finalhyphendemerits=1000
           #3}
        %
        % Note: caption moved inside the enlarged minipage
        \label{#4}%
        \end{minipage}
    }
  \else%
    % LHS page
    \makebox[\textwidth][r]{%
        \begin{minipage}[b]{\textwidth+\marginparwidth+\marginparsep}
            \includegraphics%
                [width=\linewidth]%
                {#2}%
        %
        % Note: caption goes here to have it well centered w.r.t. the whole enlarged minipage
        %
        \caption{\finalhyphendemerits=1000
           #3}
        %
        % Note: caption moved inside the enlarged minipage
        \label{#4}%
        \end{minipage}
    }
  \fi%
        % Note: caption moved inside the enlarged minipage
%
\end{figure}%
}

\newcommand{\EnlargedFigureII}[5]{%
\begin{figure}[#1]%
  \checkoddpage% needs changepage
  \ifoddpage% needs changepage, needs two laTeX passes
    % RHS page
    \makebox[\textwidth][l]{%
        \begin{minipage}[b]{\textwidth+\marginparwidth+\marginparsep}
        \makebox[\linewidth][c]{%
            \includegraphics%
                [width=#3\linewidth]%
                {#2}%
        }%
        %
        % Note: caption goes here to have it well centered w.r.t. the whole enlarged minipage
        %
        \caption{\finalhyphendemerits=1000
           #4}
        %
        % Note: caption moved inside the enlarged minipage
        \label{#5}%
        \end{minipage}
    }
  \else%
    % LHS page
    \makebox[\textwidth][r]{%
        \begin{minipage}[b]{\textwidth+\marginparwidth+\marginparsep}
        \makebox[\linewidth][c]{%
            \includegraphics%
                [width=#3\linewidth]%
                {#2}%
        }%
        %
        % Note: caption goes here to have it well centered w.r.t. the whole enlarged minipage
        %
        \caption{\finalhyphendemerits=1000
           #4}
        %
        % Note: caption moved inside the enlarged minipage
        \label{#5}%
        \end{minipage}
    }
  \fi%
        % Note: caption moved inside the enlarged minipage
%
\end{figure}%
}

\newcommand{\EnlargedFigureIV}[2]{%
\begin{figure}[#1]%
  \checkoddpage% needs changepage
  \ifoddpage% needs changepage, needs two laTeX passes
    % RHS page
    \makebox[\textwidth][l]{%
        \begin{minipage}[b]{\textwidth+\marginparwidth+\marginparsep}
        %\makebox[\linewidth][c]%
        \end{minipage}
    }
  \else%
    % LHS page
    \makebox[\textwidth][r]{%
        \begin{minipage}[b]{\textwidth+\marginparwidth+\marginparsep}
        %\makebox[\linewidth][c]%
        \end{minipage}
    }
  \fi%
\end{figure}%
}

%------------------------------------------------------------------------------------------
%% A macro to place figues over the entire \textwidth+\marginparwidth+\marginparsep
%% more general than \EnlargedFigure because the material inside the figure has to be coded
%% by the user
%
\newcommand{\EnlargedFigureX}[4]{%
\begin{figure}[#1]%
  \checkoddpage% needs changepage
  \ifoddpage% needs changepage, needs two laTeX passes
    % RHS page
    \makebox[\textwidth][l]{%
        \begin{minipage}[b]{\textwidth+\marginparwidth+\marginparsep}
            #2
        %
        % Note: caption goes here to have it well centered w.r.t. the whole enlarged minipage
        %
        \caption{%\finalhyphendemerits=1000
           #3}
        %
        % Note: caption moved inside the enlarged minipage
        \label{#4}%
        \end{minipage}
    }
  \else%
    % LHS page
    \makebox[\textwidth][r]{%
        \begin{minipage}[b]{\textwidth+\marginparwidth+\marginparsep}
            #2
        %
        % Note: caption goes here to have it well centered w.r.t. the whole enlarged minipage
        %
        \caption{%\finalhyphendemerits=1000
           #3}
        %
        % Note: caption moved inside the enlarged minipage
        \label{#4}%
        \end{minipage}
    }
  \fi%
        % Note: caption moved inside the enlarged minipage
%
\end{figure}%
}
% % Example of usage:
%
% \EnlargedFigureX% needs two latex passes
%     {<where>}% #1: t, b, p
%     {
%         \begin{tabular}{c}
%             \includegraphics[<opt>]{<file 1>}\\
%             \includegraphics[<opt>]{<file 2>}\\
%         \end{tabular}
%     }% #2: the image file included by \includegraphics
%     {<caption>}% #3: the caption text
%     {<label>}% #4: the label
%------------------------------------------------------------------------------------------

%------------------------------------------------------------------------------------------
% same as \EnlargedFigureX
%
\newcommand{\EnlargedTableX}[4]{%
\begin{table}[#1]%
  \checkoddpage% needs changepage
  \ifoddpage% needs changepage, needs two laTeX passes
    % RHS page
    \makebox[\textwidth][l]{%
        \begin{minipage}[b]{\textwidth+\marginparwidth+\marginparsep}
        %
        % Note: caption goes here to have it well centered w.r.t. the whole enlarged minipage
        %
        \caption{%\finalhyphendemerits=1000
           #3}
        \label{#4}%
            #2
        \end{minipage}
    }
  \else%
    % LHS page
    \makebox[\textwidth][r]{%
        \begin{minipage}[b]{\textwidth+\marginparwidth+\marginparsep}
        %
        % Note: caption goes here to have it well centered w.r.t. the whole enlarged minipage
        %
        \caption{%\finalhyphendemerits=1000
           #3}
        \label{#4}%
            #2
        \end{minipage}
    }
  \fi%
        % Note: caption moved inside the enlarged minipage
%
\end{table}%
}
% % Example of usage:
%
% \EnlargedTableX% needs two latex passes
%     {<where>}% #1: t, b, p
%     {
%         \begin{tabular}{c}
%             \includegraphics[<opt>]{<file 1>}\\
%             \includegraphics[<opt>]{<file 2>}\\
%         \end{tabular}
%     }% #2: the image file included by \includegraphics
%     {<caption>}% #3: the caption text
%     {<label>}% #4: the label
%------------------------------------------------------------------------------------------



%%%%% TIKZ-RELATED COMMANDS

\newcommand{\BgColorPage}[2][vshift=0.5mm,hshift=6.5mm]{%
\backgroundsetup{%
  % placement=bottom
  % position: (0,0) -> top-left corner of *textblock*
  %            x -> grows horizontally to right
  %            y -> grows horizontally to top
  anchor={below left}, position={current page.north west},
  #1,% vshift=0.5mm, hshift=6.5mm, % default
  % anchor to top-left corner of stock, 
  % x->grows vertically downwards, y->grows horizontally to right
  contents={%
    \tikz\fill[#2] (0,0) rectangle (22cm,22cm);
  }
}
\BgThispage
}

\newcommand{\PutPicture}[4]{%
% #1 : node options
% #2 : trim directives -> left  bottom  right  top
% #3 : width
% #4 : file
%
\begin{tikzpicture}[remember picture,overlay]
  %% Catch some significant points on the borders of the page
  \coordinate (p1) at (current page.south west); % lower left corner
  \coordinate (p2) at ($(p1)+(\stockwidth,0)$); % lower right
  \coordinate (p3) at ($(p2)+(0,\stockheight)$); % upper right
  \coordinate (p4) at ($(p1)+(0,\stockheight)$); % upper left
  \coordinate (m1) at ($(p1)!0.5!(p2)$); % midway of lower side
  \coordinate (m2) at ($(p2)!0.5!(p3)$); % midway of right side
  \coordinate (m3) at ($(p3)!0.5!(p4)$); % midway of upper side
  \coordinate (m4) at ($(p4)!0.5!(p1)$); % midway of left side

  \node[#1] at (p1) {% node options
    \includegraphics[%
      trim = #2,% left  bottom  right  top
      clip,% actually cut what trim says
      width=#3]{#4}%
  };
\end{tikzpicture}
}

\newcommand{\PutNode}[3]{%
% #1 : node options
% #2 : position, p1-4 | m1-4 or $...$ 
% #3 : content
%
\begin{tikzpicture}[remember picture,overlay]
  %% Catch some significant points on the borders of the page
  \coordinate (p1) at (current page.south west); % lower left corner
  \coordinate (p2) at ($(p1)+(\stockwidth,0)$); % lower right
  \coordinate (p3) at ($(p2)+(0,\stockheight)$); % upper right
  \coordinate (p4) at ($(p1)+(0,\stockheight)$); % upper left
  \coordinate (m1) at ($(p1)!0.5!(p2)$); % midway of lower side
  \coordinate (m2) at ($(p2)!0.5!(p3)$); % midway of right side
  \coordinate (m3) at ($(p3)!0.5!(p4)$); % midway of upper side
  \coordinate (m4) at ($(p4)!0.5!(p1)$); % midway of left side

  \node[#1] % node options
    at (#2) % position 
    {#3};   % content
\end{tikzpicture}
}

\newcommand{\PutStripe}[4]{%
% #1 : node options
% #2 : starting point
% #3 : width
% #4 : height
%
\begin{tikzpicture}[remember picture,overlay]
  %% Catch some significant points on the borders of the page
  \coordinate (p1) at (current page.south west); % lower left corner
  \coordinate (p2) at ($(p1)+(\stockwidth,0)$); % lower right
  \coordinate (p3) at ($(p2)+(0,\stockheight)$); % upper right
  \coordinate (p4) at ($(p1)+(0,\stockheight)$); % upper left
  \coordinate (m1) at ($(p1)!0.5!(p2)$); % midway of lower side
  \coordinate (m2) at ($(p2)!0.5!(p3)$); % midway of right side
  \coordinate (m3) at ($(p3)!0.5!(p4)$); % midway of upper side
  \coordinate (m4) at ($(p4)!0.5!(p1)$); % midway of left side

  %\fill[#1] #2 rectangle +(#3,#4);
  \node[rectangle] (rect) at #2 [#1,minimum width=#3,minimum height=#4] {};
\end{tikzpicture}
}
