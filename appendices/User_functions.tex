\documentclass[[12pt,twoside]{book}
\usepackage{_my_document_style}

\begin{document}
%
%\setcounter{chapter}{3}
%
\chapter%
   [Matlab functions for aircraft trajectory plots]%
   {Matlab functions for aircraft trajectory plots}
\label{chap:Appendix:Matlab:Trajectory}

\lstinputlisting[
    language=Matlab,
    label=list:fcn:0,
    caption={\mbox{}},
    % empty caption, only number
    inputpath={Matlab/appendices_functions},
    lineskip=0pt,
    %basicstyle={\ttfamily\small},%
    numbers=left,
    linerange={1-1,17-31},
    ]%
    {loadAircraftMAT.m}
    
    This function takes as input a .mat file which contains geometrical data regarding a specific aircraft and a shape scale factor used to fix dimensions of the object.
    
\lstinputlisting[
    language=Matlab,
    label=list:fcn:1,
    caption={\mbox{}},
    % empty caption, only number
    inputpath={Matlab/appendices_functions},
    lineskip=0pt,
    %basicstyle={\ttfamily\small},%
    numbers=left,
    linerange={1-1,25-80},
    ]%
    {plotBodyE.m}
  
  This function is used by \lstinline[basicstyle=\ttfamily]{plotTrajectoryAndBodyE} to show a given aircraft shape in Earth-axes using CoG coordinates and Euler angles.  
  
  \lstinputlisting[
    language=Matlab,
    label=list:fcn:2,
    caption={\mbox{}},
    % empty caption, only number
    inputpath={Matlab/appendices_functions},
    lineskip=0pt,
    %basicstyle={\ttfamily\small},%
    numbers=left,
    linerange={1-1,4-21},
    ]%
    {plotEarthAxes.m}
  
  This function is used in order to have a visual representation of the Earth-axes.
  
\lstinputlisting[
    language=Matlab,
    label=list:fcn:3,
    caption={\mbox{}},
    % empty caption, only number
    inputpath={Matlab/appendices_functions},
    lineskip=0pt,
    %basicstyle={\ttfamily\small},%
    numbers=left,
    %linerange={1-1,42-130,150-168},
    ]%
    {plotPoint3DHelperLines.m}
    
    This function is used by the previous one (see \lstlistingname~\ref{list:fcn:0}) to plot some helper lines that individuate time by time the exact position of the CoG using dotted lines which are the projection of the position vector along X, Y and Z axes. 
  
    
\lstinputlisting[
    language=Matlab,
    label=list:fcn:4,
    caption={\mbox{}},
    % empty caption, only number
    inputpath={Matlab/appendices_functions},
    lineskip=0pt,
    %basicstyle={\ttfamily\small},%
    numbers=left,
    linerange={1-1,42-130,150-168},
    ]%
    {plotTrajectoryAndBodyE.m}
    
    This function shows a given aircraft shape in a sequence of positions/attitudes in Earth-axes; trajectory is also shown using CoG coordinates and Euler angles.
    


    
    
\lstinputlisting[
    language=Matlab,
    label=list:fcn:5,
    caption={\mbox{}},
    % empty caption, only number
    inputpath={Matlab/appendices_functions},
    lineskip=0pt,
    %basicstyle={\ttfamily\small},%
    numbers=left,
    linerange={1-1,19-97},
    ]%
    {plotRibbon.m}
    
    This function allows the user to plot a ribbon surface defined by a sequence of aircraft positions in Earth-axes.
    
    

    
    
    
    \chapter%
   [Matlab functions for trim]%
   {Matlab functions for trim}
\label{chap:Appendix:Matlab:Trim}


\lstinputlisting[
    language=Matlab,
    label=list:fcn:6,
    caption={\mbox{}},
    % empty caption, only number
    inputpath={Matlab/appendices_functions},
    lineskip=0pt,
    %basicstyle={\ttfamily\small},%
    numbers=left,
    %linerange={1-1,25-80},
    ]%
    {costLongEquilibriumStaticStickFixed.m}

    This function allows the user to define a suitable trim cost function J of the aircraft, which will be minimized via the \lstinline[basicstyle=\ttfamily]{fmincon} Matlab function.

\lstinputlisting[
    language=Matlab,
    label=list:fcn:7,
    caption={\mbox{}},
    % empty caption, only number
    inputpath={Matlab/appendices_functions},
    lineskip=0pt,
    %basicstyle={\ttfamily\small},%
    numbers=left,
    %linerange={1-1,25-80},
    ]%
    {myNonLinearConstraint.m}

    This function implements any non~-~linear constraints related to the independent variables.


    
    
    
    
    
    
    \lstinputlisting[
    language=Matlab,
    label=list:fcn:8,
    caption={\mbox{}},
    % empty caption, only number
    inputpath={Matlab/appendices_functions},
    lineskip=0pt,
    %basicstyle={\ttfamily\small},%
    numbers=left,
    %linerange={1-1,25-80},
    ]%
    {model_datcom_longit.m} 
    
    This function import the aerodynamic model from a datcom output and allows the user to build the aerodynamic coefficient functions.



















\chapter%
   [Matlab functions for dynamics]%
   {Matlab functions for dynamics}
\label{chap:Appendix:Matlab:Dynamics}
\lstinputlisting[
    language=Matlab,
    label=list:fcn:9,
    caption={\mbox{}},
    % empty caption, only number
    inputpath={Matlab/appendices_functions},
    lineskip=0pt,
    %basicstyle={\ttfamily\small},%
    numbers=left,
    %linerange={1-1,25-80},
    ]%
    {ode_eqLongDynFree.m}
    
    This function implements the right-hand-side of the equations of the 3~-~DoF motion when the elevator is free to rotate, while stabilizer and throttle are fixed. 
    
    
    
    
    
    \lstinputlisting[
    language=Matlab,
    label=list:fcn:10,
    caption={\mbox{}},
    % empty caption, only number
    inputpath={Matlab/appendices_functions},
    lineskip=0pt,
    %basicstyle={\ttfamily\small},%
    numbers=left,
    %linerange={1-1,25-80},
    ]%
    {ode_eqLongDynFixed.m}
    
    This function implements the right-hand-side of the equations of the 3~-~DoF motion when the elevator, the stabilizer and the throttle are fixed. 
    
    
    
   
   
   
    \lstinputlisting[
    language=Matlab,
    label=list:fcn:11,
    caption={\mbox{}},
    % empty caption, only number
    inputpath={Matlab/appendices_functions},
    lineskip=0pt,
    %basicstyle={\ttfamily\small},%
    numbers=left,
    %linerange={1-1,25-80},
    ]%
    {aux_MassStickFree.m} 
    
    This function implements the mass matrix (inserire riferimento a (11.74) di Q11).


\chapter%
   [Matlab functions for turn dynamics]%
   {Matlab functions for turn dynamics}
\label{chap:Appendix:Matlab:Turn}

    
    \lstinputlisting[
    language=Matlab,
    label=list:fcn:12,
    caption={\mbox{}},
    % empty caption, only number
    inputpath={Matlab/appendices_functions},
    lineskip=0pt,
    %basicstyle={\ttfamily\small},%
    numbers=left,
    %linerange={1-1,25-80},
    ]%
    {FunCLlaw.m} 
    
    This function implements the right-hand-side of the system (inserire riferimento al sistema di equazioni 17.61).
    
    
     \lstinputlisting[
    language=Matlab,
    label=list:fcn:13,
    caption={\mbox{}},
    % empty caption, only number
    inputpath={Matlab/appendices_functions},
    lineskip=0pt,
    %basicstyle={\ttfamily\small},%
    numbers=left,
    %linerange={1-1,25-80},
    ]%
    {f_CL.m} 
    
    This function allows the user to find the equilibrium value for $C_L$ at $t=0$ (inserire riferimento alla 17.50).


   \lstinputlisting[
    language=Matlab,
    label=list:fcn:14,
    caption={\mbox{}},
    % empty caption, only number
    inputpath={Matlab/appendices_functions},
    lineskip=0pt,
    %basicstyle={\ttfamily\small},%
    numbers=left,
    %linerange={1-1,25-80},
    ]%
    {FunFZA.m} 
    
    This function implements the right-hand-side of the DAEs (inserire riferimento alle equazioni 17.64). The function limits the values of the lift coefficient; in particular it can not exceed 0.85 in order to prevent the numerical procedure from reaching too high values of the angle of attack, such as to invalidate the validity of the linear aerodynamic model.
    
    
     \lstinputlisting[
    language=Matlab,
    label=list:fcn:15,
    caption={\mbox{}},
    % empty caption, only number
    inputpath={Matlab/appendices_functions},
    lineskip=0pt,
    %basicstyle={\ttfamily\small},%
    numbers=left,
    %linerange={1-1,25-80},
    ]%
    {Fundelta_TFZA.m} 
    
    This function implements the right-hand-side of the governing differential-algebraic equations for a given $\delta_T$ and $f_{ZA}$ laws (inserire riferimento al sistema 17.67). The function limits the values of the lift coefficient; in particular it can not exceed 0.85 in order to prevent the numerical procedure from reaching too high values of the angle of attack, such as to invalidate the validity of the linear aerodynamic model. 
    
      \lstinputlisting[
    language=Matlab,
    label=list:fcn:16,
    caption={\mbox{}},
    % empty caption, only number
    inputpath={Matlab/appendices_functions},
    lineskip=0pt,
    %basicstyle={\ttfamily\small},%
    numbers=left,
    %linerange={1-1,25-80},
    ]%
    {AeroCoeff.m}
     
     \lstinputlisting[
    language=Matlab,
    label=list:fcn:17,
    caption={\mbox{}},
    % empty caption, only number
    inputpath={Matlab/appendices_functions},
    lineskip=0pt,
    %basicstyle={\ttfamily\small},%
    numbers=left,
    %linerange={1-1,25-80},
    ]%
    {correctedTurnCLAssigned.m} 
    


     \lstinputlisting[
    language=Matlab,
    label=list:fcn:18,
    caption={\mbox{}},
    % empty caption, only number
    inputpath={Matlab/appendices_functions},
    lineskip=0pt,
    %basicstyle={\ttfamily\small},%
    numbers=left,
    %linerange={1-1,25-80},
    ]%
    {ZeroAlpha.m}

   

\end{document}
