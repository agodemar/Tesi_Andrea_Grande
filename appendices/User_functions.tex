\documentclass[[12pt,twoside]{book}
\usepackage{_my_document_style}

\begin{document}
%
%\setcounter{chapter}{3}
%
\chapter%
   [Matlab functions for aircraft trajectory plots]%
   {Matlab functions for aircraft trajectory plots}
\label{chap:Appendix:Matlab:Trajectory}

\lstinputlisting[
    language=Matlab,
    label=list:fcn:0,
    caption={\mbox{}},
    % empty caption, only number
    inputpath={appendices/Matlab},
    lineskip=0pt,
    %basicstyle={\ttfamily\small},%
    numbers=left,
    linerange={1-1,17-31},
    ]%
    {loadAircraftMAT.m}
    
    This function takes as input a .mat file which contains geometrical data regarding a specific aircraft and a shape scale factor used to fix dimensions of the object.
    
\lstinputlisting[
    language=Matlab,
    label=list:fcn:1,
    caption={\mbox{}},
    % empty caption, only number
    inputpath={appendices/Matlab},
    lineskip=0pt,
    %basicstyle={\ttfamily\small},%
    numbers=left,
    linerange={1-1,25-80},
    ]%
    {plotBodyE.m}
  
  This function is used by \lstinline[basicstyle=\ttfamily]{plotTrajectoryAndBodyE} to show a given aircraft shape in Earth-axes using CoG coordinates and Euler angles.  
  
  \lstinputlisting[
    language=Matlab,
    label=list:fcn:2,
    caption={\mbox{}},
    % empty caption, only number
    inputpath={appendices/Matlab},
    lineskip=0pt,
    %basicstyle={\ttfamily\small},%
    numbers=left,
    linerange={1-1,4-21},
    ]%
    {plotEarthAxes.m}
  
  This function is used in order to have a visual representation of the Earth-axes.
 
\lstinputlisting[
    language=Matlab,
    label=list:fcn:0,
    caption={\mbox{}},
    % empty caption, only number
    inputpath={appendices/Matlab},
    lineskip=0pt,
    %basicstyle={\ttfamily\small},%
    numbers=left,
    linerange={1-1,1-75},
    ]%
    {zero_lift_angle_of_a_cranked_wing.m}
\end{document}
