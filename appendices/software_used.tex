\documentclass[[12pt,twoside]{book}
\usepackage{_my_document_style}
\begin{document}
\chapter%
   [Software used]%
   {Software used}
\label{chap:Appendix:Software:Used}
{\Large\textbf{Matlab}}
\bigskip
\noindent

 Matlab (an abbreviation of "matrix laboratory") is a high-performance language for technical computing. It integrates computation, visualization, and programming in an easy-to-use environment where problems and solutions are expressed in familiar mathematical notation.
 It allows matrix manipulations, plotting of functions and data, implementation of algorithms, creation of user interfaces, and interfacing with programs written in other languages. In this context it is used in conjunction with \textbf{GitHub}, which is a subsidiary of Microsoft which provides hosting for software development and version control using Git. It provides access control and several collaboration features such as bug tracking, feature requests, task management and continuous integration, in  particular it is used to host open-source projects.
\bigskip

\noindent
{\Large\textbf{\LaTeX{}}}
\bigskip

\LaTeX{} is a high-quality typesetting system; it includes features designed for the production of technical and scientific documentation. \LaTeX{} is de facto the most used typesetting system for the communication and publication of scientific documents; particular attention must be paid to the package  \textbf{pgfplot}: this package is a powerful tool, based on tikz, dedicated to create scientific graphs,in particulare it is a visualization tool to make simpler the inclusion of plots in a document drawing high-quality function plots in normal or logarithmic scaling with a user-friendly interface directly in \LaTeX{}. The user supplies axis labels, legend entries and the plot coordinates for one or more plots and pgfplot applies axis scaling, computes any logarithms and axis ticks and draws the plots, supporting line plots, scatter plots, piecewise constant plots, bar plots, area plots, mesh-- and surface plots and some more.





\end{document} 