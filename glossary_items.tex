%------------------------------------------------------------------------------------------
% Meta-commands for the TeXworks editor
%
% !TeX root = ./Tesi.tex
% !TEX encoding = UTF-8
% !TEX program = pdflatex
%

%--------------------------------------------------------------------------------
%                                                G L O S S A R Y    E N T R I E S

%%% ---------------------------------------------------------------- Main glossary
%\newglossaryentry{ACRF}{%
%   name=Aicraft Construction Reference Frame (ACRF),
%   description={%
%      the reference frame which has its origin in the fuselage forwardmost point, the x-axis pointing from the nose to the tail, the y-axis from fuselage plane of symmetry to the right wing (from the pilot's point of view) and the z-axis from pilot's feet to pilot's head}
%}

\newglossaryentry{Client:Code}{%
   name=client code,
   description={%
      the code where the code in question will be effectively exploited}
}

\newglossaryentry{Table}{%
   name=Table,
   description={%
      a collection that associates an ordered pair of keys, called a row key and a column key, with a single value. A table may be sparse, with only a small fraction of row key / column key pairs possessing a corresponding value. }
}

\newglossaryentry{User:Developer}{%
   name=user developer,
   description={%
      the term refers to the developer which will use a method without being interested in how the method performs the required action. This is the case of a utility method: the developer is the one who writes the method, while the user developer is who uses that method to accomplish some action which requires the functionality provided by the utility method. It has to be noticed that the user developer and the developer can be the same person}
}

\newglossaryentry{Mean:Sea:Level}{%
   name=livello del mare medio,
   description={%
      detto anche \emph{Mean Sea Level}, corrisponde alla superficie iso-geopotenziale a cui è assegnato il
   livello zero (a meno di variazioni locali della gravità terrestre);
   in aeronautica il MSL è usato come riferimento per la misura dell'altitudine}
}

\newglossaryentry{Body:Frame}{%
   name=assi velivolo,
   description={%
      terna standard di assi solidali al velivolo $\{G,x_\Body,y_\Body,z_\Body\}$, con origine nel baricentro $G$,
      asse $x_\Body$ diretto verso prua, asse $y_\Body$ diretto verso l'ala destra, asse $z_\Body$ diretto
      nel senso testa-piedi del pilota}
}

\newglossaryentry{Stability:Frame}{%
   name=assi stabilità,
   description={%
      un particolare sistema di assi velivolo $\{G,x_\Stability,y_\Stability,z_\Stability\}$ definito da una data
      condizione di volo anziché da considerazioni costruttive; è una terna di assi solidali al velivolo
      ed origine nel baricentro $G$, per cui l'orientamento dell'asse $x_\Stability$ rispetto alla fusoliera è 
      stabilito proiettando nel piano di simmetria il vettore velocità 
      del baricentro, in una prestabilita condizione di volo;
      tipicamente è usato in studi di stabilità statica e dinamica ed è riferito ad una condizione di volo 
      iniziale, simmetrica, equilibrata, a quota costante}
}

\newglossaryentry{Structural:Frame}{%
   name=assi costruttivi,
   description={%
      un sistema di assi $\{O_\Constr,x_\Constr,y_\Constr,z_\Constr\}$
      usato nei disegni costruttivi di un velivolo;
      tipicamente utilizzato per individuare punti particolari dell'aeromobile, come il centro di massa,
      i punti di contatto dei carrelli, la posizione del pilota, i punti di applicazione della spinta, eccetera;
      l'origine $O_\Constr$ è collocata tipicamente in prossimità della prua del velivolo (a volte in prossimità
      del pannello dei comandi, alle spalle del castello motore); l'asse $x_\Constr$ corre lungo la fusoliera,
      orientato verso poppa; l'asse $y_\Constr$ è orientato verso la destra del pilota;
      l'asse $z_\Constr$ è orientato nel senso piedi-testa del pilota; 
      il sistema di assi costruttivi si presenta ruotato rispetto alla terna di assi velivolo
      $\{G,x_\Body,y_\Body,z_\Body\}$ di \SI[round-precision=0]{180}{\degree} intorno a $y_\Body$}
}

\newglossaryentry{Aerodynamic:Frame}{%
   name=assi aerodinamici,
   description={%
      una terna di assi $\{G,x_\Aero,y_\Aero,z_\Aero\}$, con origine nel baricentro $G$, 
      in generale non solidale al velivolo;
      l'asse $x_\Aero$ ha per direzione la proiezione del vettore velocità $\vec{v}$ del baricentro sul
      piano di simmetria del velivolo ed è orientato nel verso del moto;
      l'asse $z_\Aero$ è ortogonale a $x_\Aero$, appartiene al piano di simmetria
      ed è orientato nel verso testa-piedi del pilota;
      l'asse $y_\Aero$, che completa la terna levogira, è orientato verso la destra del pilota e coincide con 
      l'asse velivolo $y_\Body$}
}

\newglossaryentry{Wind:Frame:Trajectory}{%
   name=assi vento (assi traiettoria),
   description={%
      una terna di assi $\{G,x_\Wind,y_\Wind,z_\Wind\}$, con origine nel baricentro $G$, 
      in generale non solidale al velivolo;
      l'asse $x_\Wind$ è diretto lungo il vettore velocità $\vec{v}$ del baricentro e orientato nel verso del moto;
      l'asse $z_\Wind$ è ortogonale a $x_\Wind$, appartiene a un piano verticale passante per $x_\Wind$
      ed è orientato nel verso testa-piedi del pilota;
      l'asse $y_\Wind$, che completa la terna levogira, è sempre ortogonale ed orientato verso la destra del pilota}
}

\newglossaryentry{Wind:Frame:American}{%
   name=assi vento (americani),
   description={%
      una terna di assi $\{G,x_\Wind,y_\Wind,z_\Wind\}$, con origine nel baricentro $G$, 
      in generale non solidale al velivolo;
      l'asse $x_\Wind$ è diretto lungo il vettore velocità $\vec{v}$ del baricentro e orientato nel verso del moto;
      l'asse $z_\Wind$ è ortogonale a $x_\Wind$, appartiene al piano di simmetria dell'aeromobile ed è
      orientato nel verso testa-piedi del pilota;
      l'asse $y_\Wind$ completa la terna levogira}
}

\newglossaryentry{Fuselage:Waterline}{%
   name=linea di galleggiamento della fusoliera,
   description={%
termine mutuato dal gergo degli architetti navali (in inglese \emph{waterline}) per indicare
sezioni di una carena con piani orizzontali;
per una fusoliera parzialmente immersa in acqua, lasciata galleggiare a riposo,
una linea di galleggiamento corrisponde con la curva intersezione del pelo libero del liquido
con la superficie esterna del solido}
}

\newglossaryentry{Earth:Frame}{%
   name=assi terra,
   description={%
      una terna di assi $\{O_\earth,x_\earth,y_\earth,z_\earth\}$, con origine $O_\earth$
      un punto convenientemente scelto sulla superficie terrestre al livello del mare medio (MSL);
      l'asse $z_\earth$ è rivolto per convenzione verso il centro terrestre, l'asse $x_\earth$ verso
      il nord geografico e l'asse $y_\earth$ verso est (convenzione degli orientamenti
      \emph{North-East-Down}, NED)}
}

\newglossaryentry{Moto:di:Regime}{%
   name=moto di regime,
   description={%
            un moto equilibrato in cui alcuni parametri di volo cararatteristici rimangono stazionari}
}
\newglossaryentry{Virosbandometro}{%
   name=virosbandometro,
   description={%
      uno strumento composto da un virometro e da uno sbandometro. Il virometro è uno strumento giroscopico,
      misura la velocità angolare e fornisce informazioni circa il rateo di virata dell'aeromobile.
      Sullo strumento è disegnata la sagoma dell'aeromobile e due tacche etichettate con ``L'' (\emph{left}) ed 
      ``R'' (\emph{right}). Se il pilota effettua una virata mantenendo la sagoma su una delle due tacche
      ottiene una velocità angolare pari a \SI[round-precision=0]{3}{deg/s}, cioè un cambiamento di
      angolo di prua di \SI[round-precision=0]{360}{\degree} compiuto in due minuti. 
      Lo sbandometro è costituito da una sfera metallica soggetta alle forze d'inerzia laterali
      e, se mantenuta al centro dell'indicatore, indica se l'aeromobile effettua una virata coordinata. 
      Se nella virata la pallina non è centrata l'aeromobile possiede un angolo di derapata non nullo}
}
\newglossaryentry{Orizzonte:Artificiale}{%
   name=orizzonte artificiale,
   description={%
      uno strumento che indica l'orientamento del velivolo nello spazio. Fornisce informazioni sull'inclinazione 
      della fusoliera rispetto all'orizzontale e sull'inclinazione laterale delle ali}
}
\newglossaryentry{Indicatore:Velocita}{%
   name=indicatore di velocità,
   description={%
      o anemometro, è lo strumento che consente di misurare la velocità dell'aereo relativa alla massa
      d'aria che lo circonda. Tale strumento risale al valore dela velocità tramite misure di pressione statica
      e totale effettuate con un tubo di Pitot ed eventuali prese statiche aggiuntive}
}
\newglossaryentry{Altimetro}{%
   name=altimetro,
   description={%
      è un barometro aneroide che converte le misurazioni di pressione atmosferica in letture di quota e
      generalmente in piedi. Sono presenti due lancette: la più corta è quella delle migliaia di piedi, 
      la più lunga è quella delle centinaia di piedi. \`E presente anche uno specchietto che indica
      le decine di migliaia di piedi. La lettura della quota può essere tarata in funzione ad un valore di 
      pressione a livello del mare diverso da quello standard}
}
\newglossaryentry{Variometro}{%
   name=variometro,
   description={%
      strumento che misura la velocità verticale dell'aereo, cioè il rateo di salita,
      a partire dalla lettura della variazioni di pressione nel volo in salita. Le pressioni sono misurate
      attraverso una presa statica posizionata in un punto tale da non essere influenzato da particolari 
      depressioni o sovrapressioni del campo aerodinamico che circonda il velivolo}
}
\newglossaryentry{Indicatore:di:Direzione}{%
   name=indicatore di direzione,
   description={%
      o indicatore di prua, è costituito da un giroscopio dall'aspetto
      di una bussola con su disegnata la sagoma dell'aereo. Lo scopo dello strumento è di
      mostrare al pilota l'angolo di prua. L'indicatore direzionale,
      al contrario della normale bussola, non è soggetto agli errori causati da moti di cabrata
      o picchiata o dalla turbolenza atmosferica}
}
\newglossaryentry{Punto:Neutro:Comandi:Bloccati}{%
   name=punto neutro a comandi bloccati,
   description={%
      punto rispetto al quale il momento di beccheggio del velivolo a comandi bloccati
      risulta costante con l'angolo d'attacco. Corrisponde alla posizione massima arretrata
      del baricentro, oltre la quale il velivolo presenta un gradiente  
      \smash{$C_{\mathcal{M}_\mathlarger{\alpha}}$} negativo del
      momento baricentrale, cioè una instabilità statica al beccheggio}
}
\newglossaryentry{Punto:Neutro:Comandi:Liberi}{%
   name=punto neutro a comandi liberi,
   description={%
      equivalente del punto neutro a comandi bloccati nel caso di comando longitudinale
      reversibile lasciato libero di ruotare intorno all'asse di cerniera per
      effetto della pressione dinamica della corrente}
}
\newglossaryentry{Asse:Spinta}{%
   name=asse di spinta,
   description={%
      direzione lungo la quale agisce la spinta totale $T$}
}


%%% --------------------------------------------------------------- List of symbols
%%% see _local_macros.tex

\newglossaryentry{vec:F}{%
   type=symbols,
   name={\ensuremath{\vec{F}}},
   sort=F,
   description={vettore forza esterna risultante}
}

\newglossaryentry{vec:M}{%
   type=symbols,
   name={\ensuremath{\vec{\mathcal{M}}}},
   sort=M,
   description={vettore momento baricentrico esterno risultante}
}

\newglossaryentry{vec:W}{%
   type=symbols,
   name={\ensuremath{\vec{W}}},
   sort=W,
   description={vettore peso del velivolo, applicato al baricentro $G$, detto anche $\vec{F}_{\mspace{-6mu}\Grav}$}
}

\newglossaryentry{Weight}{%
   type=symbols,
   name={\ensuremath{W}},
   sort=W,
   description={peso del velivolo, pari a $mg$}
}

\newglossaryentry{vec:g}{%
   type=symbols,
   name={\ensuremath{\vec{g}}},
   sort=g,
   description={vettore accelerazione gravitazionale}
}

\newglossaryentry{mass}{%
   type=symbols,
   name={\ensuremath{m}},
   sort=m,
   description={massa del velivolo}
}

\newglossaryentry{mass:Wing}{%
   type=symbols,
   name={\ensuremath{m_\mathrm{W}}},
   sort=mW,
   description={wing mass}
}

\newglossaryentry{rho}{%
   type=symbols,
   name={\ensuremath{\rho}},
   sort=zz:rho,
   description={densità dell'aria alla quota di volo}
}

\newglossaryentry{phi}{%
   type=symbols,
   name={\ensuremath{\phi}},
   sort=zz:phi,
   description={angolo d'inclinazione laterale delle ali.\glspar
                Terzo angolo della terna di angoli di Eulero $(\psi,\theta,\phi)$
                dell'orientamento del velivolo rispetto a un riferimento fisso}
}

\newglossaryentry{psi}{%
   type=symbols,
   name={\ensuremath{\psi}},
   sort=zz:psi,
   description={angolo di azimuth dell'asse velivolo $x_\Body$.\glspar
                Primo angolo della terna di angoli di Eulero $(\psi,\theta,\phi)$
                dell'orientamento del velivolo rispetto a un riferimento fisso}
}

\newglossaryentry{psi:GT}{%
   type=symbols,
   name={\ensuremath{\psi_\GroundTrack}},
   sort=zz:psi:GT,
   description={\textit{ground-track heading}, detto anche angolo di virata $\delta$.\glspar
                Angolo che la proiezione a terra della velocità $\vec{V}$ del baricentro del velivolo
                forma con il nord}
}

\newglossaryentry{theta}{%
   type=symbols,
   name={\ensuremath{\theta}},
   sort=zz:theta,
   description={angolo d'inclinazione sull'orizzontale dell'asse velivolo $x_\Body$.\glspar
                Secondo angolo della terna di angoli di Eulero $(\psi,\theta,\phi)$
                dell'orientamento del velivolo rispetto a un riferimento fisso}
}

\newglossaryentry{alpha}{%
   type=symbols,
   name={\ensuremath{\alpha}},
   sort=zz:alpha,
   description={angolo d'attacco riferito a una retta del piano di simmetria del velivolo passante per
      il baricentro, inclinata di un angolo $\mu_x$ rispetto all'asse $x_\Body$}
}

\newglossaryentry{alpha:Body}{%
   type=symbols,
   name={\ensuremath{\alpha_\Body}},
   sort=zz:alpha:B,
   description={angolo d'attacco riferito all'asse $x_\Body$}
}

\newglossaryentry{alpha:Wing}{%
   type=symbols,
   name={\ensuremath{\alpha_\Wing}},
   sort=zz:alpha:W,
   description={angolo d'attacco riferito alla corda di radice dell'ala}
}

\newglossaryentry{beta}{%
   type=symbols,
   name={\ensuremath{\beta}},
   sort=zz:beta,
   description={angolo di derapata (o di \emph{sideslip})}
}

\newglossaryentry{gamma}{%
   type=symbols,
   name={\ensuremath{\gamma}},
   sort=zz:gamma,
   description={angolo di salita (o di rampa o di volta)}
}

\newglossaryentry{Drag}{%
   type=symbols,
   name={\ensuremath{D}},
   sort=D,
   description={resistenza aerodinamica (\emph{drag})}
}

\newglossaryentry{Lift}{%
   type=symbols,
   name={\ensuremath{L}},
   sort=L,
   description={portanza aerodinamica (\emph{lift})}
}

\newglossaryentry{vec:Omega:Body:wrt:Inertial}{%
   type=symbols,
   name={\ensuremath{\vec{\Omega}}},
   sort=zz:Omega,
   description={vettore velocità angolare istantanea del velivolo rispetto a un riferimento inerziale}
}

\newglossaryentry{vec:V:Body:wrt:Inertial}{%
   type=symbols,
   name={\ensuremath{\vec{V}}},
   sort=V,
   description={vettore velocità del baricentro del velivolo (rispetto all'osservatore inerziale)}
}

\newglossaryentry{V:true}{%
   type=symbols,
   name={\ensuremath{V_\mathrm{t}}},
   sort=Vt,
   description={velocità vera (\emph{true airspeed})}
}

\newglossaryentry{V:equiv}{%
   type=symbols,
   name={\ensuremath{V_\mathrm{e}}},
   sort=Ve,
   description={velocità equivalente (\emph{equivalent airspeed})}
}

\newglossaryentry{u:Body}{%
   type=symbols,
   name={\ensuremath{u}},
   sort=u,
   description={componente del vettore velocità $\vec{V}$ lungo l'asse velivolo $x_\Body$}
}

\newglossaryentry{v:Body}{%
   type=symbols,
   name={\ensuremath{v}},
   sort=v,
   description={componente del vettore velocità $\vec{V}$ lungo l'asse velivolo $y_\Body$}
}

\newglossaryentry{w:Body}{%
   type=symbols,
   name={\ensuremath{w}},
   sort=w,
   description={componente del vettore velocità $\vec{V}$ lungo l'asse velivolo $z_\Body$}
}

\newglossaryentry{p:Body}{%
   type=symbols,
   name={\ensuremath{p}},
   sort=p,
   description={componente del vettore velocità angolare $\vec{\Omega}$ lungo l'asse velivolo $x_\Body$}
}

\newglossaryentry{q:Body}{%
   type=symbols,
   name={\ensuremath{q}},
   sort=q,
   description={componente del vettore velocità angolare $\vec{\Omega}$ lungo l'asse velivolo $y_\Body$}
}

\newglossaryentry{r:Body}{%
   type=symbols,
   name={\ensuremath{r}},
   sort=r,
   description={componente del vettore velocità angolare $\vec{\Omega}$ lungo l'asse velivolo $z_\Body$}
}

\newglossaryentry{q:bar}{%
   type=symbols,
   name={\ensuremath{\bar{q}}},
   sort=q,
   description={pressione dinamica di volo}
}

\newglossaryentry{Mach}{%
   type=symbols,
   name={\ensuremath{\Mach}},
   sort=M,
   description={numero di Mach di volo}
}

\newglossaryentry{Reynolds}{%
   type=symbols,
   name={\ensuremath{\Reynolds}},
   sort=Re,
   description={numero di Reynolds di volo (basato tipicamente sulla corda di riferimento $\bar{c}$)}
}

%\newglossaryentry{var:Ti2ec}{%
%   type=symbols,
%   name={\mdseries\texttt{Ti2ec}},
%   sort=Ti2ec,
%   description={variable of class \texttt{FGMatrix33}, a 3$\times$3 matrix implementing
%   the ECI (inertial) to ECEF frame transformation matrix}
%}

\newglossaryentry{x:Aero}{%
   type=symbols,
   name={\ensuremath{x_\Aero}},
   sort=xB,
   description={asse aerodinamico longitudinale}
}

\newglossaryentry{x:Body}{%
   type=symbols,%                      goes in the list of symbols
   name={\ensuremath{x_\Body}},
   sort=xB,
   description={asse velivolo longitudinale o asse di rollio}
}

\newglossaryentry{X:Body:Force}{%
   type=symbols,
   name={\ensuremath{X}},
   sort=X,
   description={componente della forza esterna risultante lungo l'asse velivolo $x_\Body$}
}

\newglossaryentry{x:Stability}{%
   type=symbols,
   name={\ensuremath{x_\Stability}},
   sort=xS,
   description={asse di stabilità longitudinale}
}

\newglossaryentry{YA}{%
   type=symbols,
   name={\ensuremath{Y_\Aero}},
   sort=YA,
   description={componente laterale della forza aerodinamica}
}

\newglossaryentry{y:Aero}{%
   type=symbols,
   name={\ensuremath{y_\Aero}},
   sort=xB,
   description={asse aerodinamico laterale}
}

\newglossaryentry{y:Body}{%
   type=symbols,
   name={\ensuremath{y_\Body}},
   sort=yB,
   description={asse velivolo laterale o asse di beccheggio}
}

\newglossaryentry{Y:Body:Force}{%
   type=symbols,
   name={\ensuremath{Y}},
   sort=Y,
   description={componente laterale della forza esterna risultante, lungo l'asse velivolo $y_\Body$}
}

\newglossaryentry{z:Aero}{%
   type=symbols,
   name={\ensuremath{z_\Aero}},
   sort=xB,
   description={asse aerodinamico di portanza}
}

\newglossaryentry{z:Body}{%
   type=symbols,
   name={\ensuremath{z_\Body}},
   sort=zB,
   description={asse velivolo di imbardata}
}

\newglossaryentry{z:Stability}{%
   type=symbols,
   name={\ensuremath{z_\Stability}},
   sort=xS,
   description={asse di stabilità trasversale}
}

\newglossaryentry{z:Vertical}{%
   type=symbols,
   name={\ensuremath{z_\Vertical}},
   sort=zV,
   description={asse verticale locale, con origine nel baricentro del velivolo e orientato positivamente verso il basso}
}

\newglossaryentry{Z:Body:Force}{%
   type=symbols,
   name={\ensuremath{Z}},
   sort=Y,
   description={componente della forza esterna risultante lungo l'asse velivolo $z_\Body$}
}

\newglossaryentry{CG}{%
   type=symbols,
   name={\ensuremath{G}},
   sort=G,
   description={baricentro del velivolo}
}

\newglossaryentry{i:Wing}{%
   type=symbols,
   name={\ensuremath{i_\Wing}},
   sort=iW,
   description={angolo di calettamento dell'ala rispetto alla fusoliera, formato dalla corda di radice alare
      con l'asse longitudinale $x_\Body$}
}

\newglossaryentry{i:Htail}{%
   type=symbols,
   name={\ensuremath{i_\Htail}},
   sort=iH,
   description={angolo di calettamento dell'impennaggio orizzontale rispetto alla fusoliera, formato dalla corda di 
      radice del piano di coda con l'asse longitudinale $x_\Body$}
}

\newglossaryentry{sub:B}{%
   type=symbols,
   name={\ensuremath{(\;)_\Body}},
   sort=aaB,
   description={grandezza o componente nel riferimento di assi velivolo; termine relativo alla fusoliera (\emph{body})}
}

\newglossaryentry{sub:W}{%
   type=symbols,
   name={\ensuremath{(\;)_\Wind}},
   sort=aaW,
   description={grandezza o componente nel riferimento di assi vento; termine relativo all'ala (\emph{wing})}
}

\newglossaryentry{sub:WB}{%
   type=symbols,
   name={\ensuremath{(\;)_{\Wing\Body}}},
   sort=aaWB,
   description={termine relativo al velivolo parziale (configurazione ala-fusoliera o \emph{wing-body})}
}

\newglossaryentry{sub:H}{%
   type=symbols,
   name={\ensuremath{(\;)_{\Htail}}},
   sort=aaH,
   description={termine relativo all'impennaggio orizzontale (\emph{horizontal tail})}
}

\newglossaryentry{sub:V}{%
   type=symbols,
   name={\ensuremath{(\;)_{\Vtail}}},
   sort=aaV,
   description={termine relativo all'impennaggio verticale (\emph{vertical tail})}
}

\newglossaryentry{sub:A}{%
   type=symbols,
   name={\ensuremath{(\;)_\Aero}},
   sort=aaA,
   description={grandezza o componente nel riferimento di assi aerodinamici; azione di natura aerodinamica}
}

\newglossaryentry{sub:T}{%
   type=symbols,
   name={\ensuremath{(\;)_\Thrust}},
   sort=aaT,
   description={azione di natura propulsiva (\emph{thrust}); termine collegato alla spinta $T$}
}

\newglossaryentry{sub:G}{%
   type=symbols,
   name={\ensuremath{(\;)_\Grav}},
   sort=aaG,
   description={azione di natura gravitazionale}
}

\newglossaryentry{sub:SL}{%
   type=symbols,
   name={\ensuremath{(\;)_\SeaLevel}},
   sort=aaSL,
   description={grandezza calcolata al livello del mare (\emph{sea level})}
}
\newglossaryentry{mu:Thrust}{%
   type=symbols,
   name={\ensuremath{\mu_\Thrust}},
   sort=zz:mu,
   description={inclinazione della spinta rispetto all'asse velivolo longitudinale}
}
\newglossaryentry{Thrust}{%
   type=symbols,
   name={\ensuremath{T}},
   sort=T,
   description={spinta propulsiva}
}
\newglossaryentry{vec:Thrust}{%
   type=symbols,
   name={\ensuremath{\vec{T}}},
   sort=T,
   description={vettore spinta, indicato come forza esterna anche con $\vec{F}_{\mspace{-6mu}\Thrust}$}
}

%% TO DO:
%\item[$\bar{q} =$] pressione dinamica di volo, anche detta $q_{\infty}$ in Aerodinamica;
%per la (\ref{eq:Vequivalente})
%\item[$=$]$\dfrac{1}{2} \rho (u^2+v^2+w^2) = \dfrac{1}{2}\,\gamma\,p\,\Mach^2$;
%\item[$\rho =$] densità dell'aria alla quota effettiva di volo;
%\item[$p =$] pressione  statica alla quota effettiva di volo;
%\item[$\Mach =$] numero di Mach di volo;
%\item[$\gamma =$] rapporto dei calori specifici dell'aria $(=\SI{1.4}{})$;
%\item[$S =$] superficie di riferimento, tipicamente la superficie della forma in pianta dell'ala;
%\item[$b =$] apertura alare di riferimento;
%\item[$c =$] corda alare di riferimento, tipicamente la corda media aerodinamica dell'ala, detta anche $\bar{c}$.


