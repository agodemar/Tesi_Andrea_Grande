
%%% --------------------------------------------------------------- List of symbols
%%% see _local_macros.tex

\newglossaryentry{vec:F}{%
   type=symbols,
   name={\ensuremath{\vec{F}}},
   sort=F,
   description={resulting external force vector}
}

\newglossaryentry{vec:M}{%
   type=symbols,
   name={\ensuremath{\vec{\mathcal{M}}}},
   sort=M,
   description={resulting external barycentric moment vector}
}

\newglossaryentry{vec:W}{%
   type=symbols,
   name={\ensuremath{\vec{W}}},
   sort=W,
   description={weight vector of the aircraft, applied to the center of gravity $G$, also said $\vec{F}_{\mspace{-6mu}\Grav}$}
}

\newglossaryentry{Weight}{%
   type=symbols,
   name={\ensuremath{W}},
   sort=W,
   description={weight of the aircraft, equal to $mg$}
}

\newglossaryentry{vec:g}{%
   type=symbols,
   name={\ensuremath{\vec{g}}},
   sort=g,
   description={gravitational acceleration vector}
}

\newglossaryentry{mass}{%
   type=symbols,
   name={\ensuremath{m}},
   sort=m,
   description={aircraft mass}
}

\newglossaryentry{mass:Wing}{%
   type=symbols,
   name={\ensuremath{m_\mathrm{W}}},
   sort=mW,
   description={wing mass}
}

\newglossaryentry{rho}{%
   type=symbols,
   name={\ensuremath{\rho}},
   sort=zz:rho,
   description={air density at flight altitude}
}

\newglossaryentry{phi}{%
   type=symbols,
   name={\ensuremath{\phi}},
   sort=zz:phi,
   description={lateral tilt angle of the wings.\glspar
                Third angle of the three Euler angles $(\psi,\theta,\phi)$
                of orientation of the aircraft relative to a fixed reference}
}

\newglossaryentry{psi}{%
   type=symbols,
   name={\ensuremath{\psi}},
   sort=zz:psi,
   description={Azimuth angle of the aircraft axis $x_\Body$.\glspar
                First angle of the three angles of Euler $(\psi,\theta,\phi)$
                of orientation of the aircraft relative to a fixed reference}
}

\newglossaryentry{psi:GT}{%
   type=symbols,
   name={\ensuremath{\psi_\GroundTrack}},
   sort=zz:psi:GT,
   description={\textit{ground-track heading} $\delta$.\glspar
                Angle that the ground projection of the speed $\vec{V}$ of the aircraft center of gravity makes with the north}
}

\newglossaryentry{theta}{%
   type=symbols,
   name={\ensuremath{\theta}},
   sort=zz:theta,
   description={angle of inclination on the horizontal axis of the aircraft $x_\Body$.\glspar
                Second angle of the three angles of Euler $(\psi,\theta,\phi)$
                of orientation of the aircraft relative to a fixed reference}
}

\newglossaryentry{alpha}{%
   type=symbols,
   name={\ensuremath{\alpha}},
   sort=zz:alpha,
   description={angle of attack referred to a straight line of the plane of symmetry of the aircraft passing through 
       the center of gravity, inclined at an angle $\mu_x$ with respect to the axis $x_\Body$}
}

\newglossaryentry{alpha:Body}{%
   type=symbols,
   name={\ensuremath{\alpha_\Body}},
   sort=zz:alpha:B,
   description={angle of attack referred to the axis $x_\Body$}
}

\newglossaryentry{alpha:Wing}{%
   type=symbols,
   name={\ensuremath{\alpha_\Wing}},
   sort=zz:alpha:W,
   description={angle of attack referred to the root chord of the wing}
}

\newglossaryentry{beta}{%
   type=symbols,
   name={\ensuremath{\beta}},
   sort=zz:beta,
   description={\emph{sideslip} angle}
}

\newglossaryentry{gamma}{%
   type=symbols,
   name={\ensuremath{\gamma}},
   sort=zz:gamma,
   description={climb angle}
}

\newglossaryentry{Drag}{%
   type=symbols,
   name={\ensuremath{D}},
   sort=D,
   description={\emph{drag}}
}

\newglossaryentry{Lift}{%
   type=symbols,
   name={\ensuremath{L}},
   sort=L,
   description={\emph{lift}}
}

\newglossaryentry{vec:Omega:Body:wrt:Inertial}{%
   type=symbols,
   name={\ensuremath{\vec{\Omega}}},
   sort=zz:Omega,
   description={instantaneous angular velocity vector of the aircraft with respect to an inertial reference}
}

\newglossaryentry{vec:V:Body:wrt:Inertial}{%
   type=symbols,
   name={\ensuremath{\vec{V}}},
   sort=V,
   description={velocity vector of the center of gravity of the aircraft (with respect to the inertial observer)}
}

\newglossaryentry{V:true}{%
   type=symbols,
   name={\ensuremath{V_\mathrm{t}}},
   sort=Vt,
   description={\emph{true airspeed}}
}

\newglossaryentry{V:equiv}{%
   type=symbols,
   name={\ensuremath{V_\mathrm{e}}},
   sort=Ve,
   description={\emph{equivalent airspeed}}
}

\newglossaryentry{u:Body}{%
   type=symbols,
   name={\ensuremath{u}},
   sort=u,
   description={component of the velocity vector $\vec{V}$ along the aircraft axis $x_\Body$}
}

\newglossaryentry{v:Body}{%
   type=symbols,
   name={\ensuremath{v}},
   sort=v,
   description={component of the velocity vector $\vec{V}$ along the aircraft axis $y_\Body$}
}

\newglossaryentry{w:Body}{%
   type=symbols,
   name={\ensuremath{w}},
   sort=w,
   description={component of the velocity vector $\vec{V}$ along the aircraft axis $z_\Body$}
}

\newglossaryentry{p:Body}{%
   type=symbols,
   name={\ensuremath{p}},
   sort=p,
   description={component of the angular velocity vector $\vec{\Omega}$ along the aircraft axis $x_\Body$}
}

\newglossaryentry{q:Body}{%
   type=symbols,
   name={\ensuremath{q}},
   sort=q,
   description={component of the angular velocity vector $\vec{\Omega}$ along the aircraft axis $y_\Body$}
}

\newglossaryentry{r:Body}{%
   type=symbols,
   name={\ensuremath{r}},
   sort=r,
   description={component of the angular velocity vector $\vec{\Omega}$ along the aircraft axis $z_\Body$}
}

\newglossaryentry{q:bar}{%
   type=symbols,
   name={\ensuremath{\bar{q}}},
   sort=q,
   description={dynamic flight pressure}
}

\newglossaryentry{Mach}{%
   type=symbols,
   name={\ensuremath{\Mach}},
   sort=M,
   description={flight Mach number}
}

\newglossaryentry{Reynolds}{%
   type=symbols,
   name={\ensuremath{\Reynolds}},
   sort=Re,
   description={Reynolds number of flight (typically based on the reference chord $\bar{c}$)}
}

%\newglossaryentry{var:Ti2ec}{%
%   type=symbols,
%   name={\mdseries\texttt{Ti2ec}},
%   sort=Ti2ec,
%   description={variable of class \texttt{FGMatrix33}, a 3$\times$3 matrix implementing
%   the ECI (inertial) to ECEF frame transformation matrix}
%}

\newglossaryentry{x:Aero}{%
   type=symbols,
   name={\ensuremath{x_\Aero}},
   sort=xB,
   description={longitudinal aerodynamic axis}
}

\newglossaryentry{x:Body}{%
   type=symbols,%                      goes in the list of symbols
   name={\ensuremath{x_\Body}},
   sort=xB,
   description={longitudinal aircraft axis or roll axis}
}

\newglossaryentry{X:Body:Force}{%
   type=symbols,
   name={\ensuremath{X}},
   sort=X,
   description={component of the resulting external force along the aircraft axis $x_\Body$}
}

\newglossaryentry{x:Stability}{%
   type=symbols,
   name={\ensuremath{x_\Stability}},
   sort=xS,
   description={longitudinal stability axis}
}

\newglossaryentry{YA}{%
   type=symbols,
   name={\ensuremath{Y_\Aero}},
   sort=YA,
   description={lateral component of the aerodynamic force}
}

\newglossaryentry{y:Aero}{%
   type=symbols,
   name={\ensuremath{y_\Aero}},
   sort=xB,
   description={lateral aerodynamic axis}
}

\newglossaryentry{y:Body}{%
   type=symbols,
   name={\ensuremath{y_\Body}},
   sort=yB,
   description={aircraft lateral axis or pitch axis}
}

\newglossaryentry{Y:Body:Force}{%
   type=symbols,
   name={\ensuremath{Y}},
   sort=Y,
   description={lateral component of the resulting external force, along the aircraft axis $y_\Body$}
}

\newglossaryentry{z:Aero}{%
   type=symbols,
   name={\ensuremath{z_\Aero}},
   sort=xB,
   description={aerodynamic lift axis}
}

\newglossaryentry{z:Body}{%
   type=symbols,
   name={\ensuremath{z_\Body}},
   sort=zB,
   description={aircraft axis of yaw}
}

\newglossaryentry{z:Stability}{%
   type=symbols,
   name={\ensuremath{z_\Stability}},
   sort=xS,
   description={transverse stability axis}
}

\newglossaryentry{z:Vertical}{%
   type=symbols,
   name={\ensuremath{z_\Vertical}},
   sort=zV,
   description={local vertical axis, originating in the center of gravity of the aircraft and positively oriented downwards}
}

\newglossaryentry{Z:Body:Force}{%
   type=symbols,
   name={\ensuremath{Z}},
   sort=Y,
   description={component of the resulting external force along the aircraft axis $z_\Body$}
}

\newglossaryentry{CG}{%
   type=symbols,
   name={\ensuremath{G}},
   sort=G,
   description={center of gravity of the aircraft}
}

\newglossaryentry{i:Wing}{%
   type=symbols,
   name={\ensuremath{i_\Wing}},
   sort=iW,
   description={twist angle of the horizontal tail with respect to the fuselage, formed by the root chord of
      the wing plane with the longitudinal axis $x_\Body$}
}

\newglossaryentry{i:Htail}{%
   type=symbols,
   name={\ensuremath{i_\Htail}},
   sort=iH,
   description={twist angle of the horizontal tail with respect to the fuselage, formed by the root chord of
      the tail plane with the longitudinal axis $x_\Body$}
}

\newglossaryentry{sub:B}{%
   type=symbols,
   name={\ensuremath{(\;)_\Body}},
   sort=aaB,
   description={quantity or component in the aircraft axis reference; term related to the  \emph{body}}
}

\newglossaryentry{sub:W}{%
   type=symbols,
   name={\ensuremath{(\;)_\Wind}},
   sort=aaW,
   description={quantity or component in the reference of wind axes; term related to the \emph{wing}}
}

\newglossaryentry{sub:WB}{%
   type=symbols,
   name={\ensuremath{(\;)_{\Wing\Body}}},
   sort=aaWB,
   description={term related to the \emph{wing-body} configuration}
}

\newglossaryentry{sub:H}{%
   type=symbols,
   name={\ensuremath{(\;)_{\Htail}}},
   sort=aaH,
   description={term related to the \emph{horizontal tail}}
}

\newglossaryentry{sub:V}{%
   type=symbols,
   name={\ensuremath{(\;)_{\Vtail}}},
   sort=aaV,
   description={term related to the \emph{vertical tail}}
}

\newglossaryentry{sub:A}{%
   type=symbols,
   name={\ensuremath{(\;)_\Aero}},
   sort=aaA,
   description={quantity or component in the reference of aerodynamic axes; aerodynamic action}
}

\newglossaryentry{sub:T}{%
   type=symbols,
   name={\ensuremath{(\;)_\Thrust}},
   sort=aaT,
   description={action of a propulsive nature (\emph{thrust}); term related to thrust $T$}
}

\newglossaryentry{sub:G}{%
   type=symbols,
   name={\ensuremath{(\;)_\Grav}},
   sort=aaG,
   description={gravitational action}
}

\newglossaryentry{sub:SL}{%
   type=symbols,
   name={\ensuremath{(\;)_\SeaLevel}},
   sort=aaSL,
   description={variable calculated at \emph{sea level}}
}
\newglossaryentry{mu:Thrust}{%
   type=symbols,
   name={\ensuremath{\mu_\Thrust}},
   sort=zz:mu,
   description={thrust inclination with respect to the longitudinal aircraft axis}
}
\newglossaryentry{Thrust}{%
   type=symbols,
   name={\ensuremath{T}},
   sort=T,
   description={thrust}
}
\newglossaryentry{vec:Thrust}{%
   type=symbols,
   name={\ensuremath{\vec{T}}},
   sort=T,
   description={thrust vector, also referred to as external force with $\vec{F}_{\mspace{-6mu}\Thrust}$}
}

%% TO DO:
%\item[$\bar{q} =$] pressione dinamica di volo, anche detta $q_{\infty}$ in Aerodinamica;
%per la (\ref{eq:Vequivalente})
%\item[$=$]$\dfrac{1}{2} \rho (u^2+v^2+w^2) = \dfrac{1}{2}\,\gamma\,p\,\Mach^2$;
%\item[$\rho =$] densità dell'aria alla quota effettiva di volo;
%\item[$p =$] pressione  statica alla quota effettiva di volo;
%\item[$\Mach =$] numero di Mach di volo;
%\item[$\gamma =$] rapporto dei calori specifici dell'aria $(=\SI{1.4}{})$;
%\item[$S =$] superficie di riferimento, tipicamente la superficie della forma in pianta dell'ala;
%\item[$b =$] apertura alare di riferimento;
%\item[$c =$] corda alare di riferimento, tipicamente la corda media aerodinamica dell'ala, detta anche $\bar{c}$.

%%% -------------------------------------------------------------------- Acronyms

%\newacronym[long={Aicraft Construction Reference Frame}]{acr:ACRF}{ACRF}{%
%  Aicraft Construction Reference Frame.\glspar
%      A reference frame which has its origin in the fuselage forwardmost point, the x-axis pointing from the nose to the tail, the y-axis from fuselage plane of symmetry to the right wing (from the pilot's point of view) and the z-axis from pilot's feet to pilot's head%
%}%

\newacronym{acr:Mean:Sea:Level}{MSL}{%
   livello del mare medio o \emph{Mean Sea Level}%
}

\newacronym{acr:NED}{NED}{%
   \emph{North-East-Down}%
}

\newacronym{acr:AGL}{AGL}{%
   Above Ground Level.\glspar
   Used to describe an aircraft's altitude above the ground.
   Aircraft instruments do not provide the pilot with this information. 
   Altitude AGL must be deduced from a reading of the altimeter and charts or
   knowledge of the terrain over which the aircraft is operating
}
\newacronym{acr:AIAA}{AIAA}{American Institute of Aeronautics and Astronautics}
\newacronym{acr:DAVE-ML}{DAVE-ML}{Dynamic Aerospace Vehicle Exchange Markup Language}
\newacronym{acr:DME}{DME}{%
   Distance Measuring Equipment.\glspar
   Provides the pilot with a panel readout of the aircraft's distance from a VOR station, in nautical miles
}
\newacronym{acr:FAA}{FAA}{Federal Aviation Administration}
\newacronym{acr:GNC}{GNC}{Guidance Navigation and Control}
\newacronym{acr:JSBSim-ML}{JSBSim-ML}{JSBSim Markup Language}
\newacronym{acr:MST}{MSTC}{AIAA Modelling and Simulation Technical Commitee}

\newdualentry{DC} % label
  {DC}            % abbreviation
  {Direction Cosines}  % long form
  {The cosines of the angles between a vector and the three coordinate axes of a given reference frame.} % description

\newdualentry{DCM}% label
   {DCM}% abbreviation
   {Direction Cosine Matrix}  % long form
   {the matrix containing the direction cosines}% description

\newdualentry{ACRF} % label
  {ACRF}            % abbreviation
  {Aicraft Construction Reference Frame}  % long form
  {The reference frame which has its origin in the fuselage forwardmost point, the x-axis pointing from the nose to the tail, the y-axis from fuselage plane of symmetry to the right wing (from the pilot's point of view) and the z-axis from pilot's feet to pilot's head} % description
%

